% Created 2025-04-24 Thu 00:24
% Intended LaTeX compiler: pdflatex
\documentclass{article}
\newcommand{\use}[2][]{\usepackage[#1]{#2}}
% PACCHETTI FONDAMENTLAI
\use[utf8]{inputenc}
\use[T1]{fontenc}
\use{graphicx}
\use{longtable}
\use{wrapfig}
\use{rotating}
\use[normalem]{ulem}
\use{amsmath}
\use{amsthm}
\use{amssymb}
\use{capt-of}
\use[italian]{babel}
\use[babel]{csquotes}
\use[style=numeric, hyperref]{biblatex}
\use{microtype}
\use{lmodern}
\use{subfig} % sottofigure
\use{multicol} % due colonne
\use{lipsum} % lorem ipsum
\use{color} % colori in latex
\use{parskip} % rimuove l'indentazione dei nuovi paragrafi %% Add parbox=false to all new tcolorbox
\use{centernot}
\use[outline]{contour}\contourlength{3pt}
\use{fancyhdr}
\use{layout}
\use[most]{tcolorbox} % Riquadri colorati
\use{ifthen} % IFTHEN
\use{geometry}

% pacchetti matematica
\use{yhmath}
\use{dsfont}
\use{mathrsfs}
\use{cancel} % semplificare
\use{polynom} %divisione tra polinomi
\use{forest} % grafi ad albero
\use{booktabs} % tabelle
\use{commath} %simboli e differenziali
\use{bm} %bold
\use[fulladjust]{marginnote} %to use marginnote for date notes
\use{arrayjobx}%array
\use[intlimits]{empheq} % Riquadri colorati attorno alle equazioni
\use{mathtools}
\use{circuitikz} % Disegnare i circuiti
%%%%%%%%%%%%%


%%%% QUIVER
\newcommand{\duepunti}{\,\mathchar\numexpr"6000+`:\relax\,}
% A TikZ style for curved arrows of a fixed height, due to AndréC.
\tikzset{curve/.style={settings={#1},to path={(\tikztostart)
    .. controls ($(\tikztostart)!\pv{pos}!(\tikztotarget)!\pv{height}!270:(\tikztotarget)$)
    and ($(\tikztostart)!1-\pv{pos}!(\tikztotarget)!\pv{height}!270:(\tikztotarget)$)
    .. (\tikztotarget)\tikztonodes}},
    settings/.code={\tikzset{quiver/.cd,#1}
        \def\pv##1{\pgfkeysvalueof{/tikz/quiver/##1}}},
    quiver/.cd,pos/.initial=0.35,height/.initial=0}

% TikZ arrowhead/tail styles.
\tikzset{tail reversed/.code={\pgfsetarrowsstart{tikzcd to}}}
\tikzset{2tail/.code={\pgfsetarrowsstart{Implies[reversed]}}}
\tikzset{2tail reversed/.code={\pgfsetarrowsstart{Implies}}}
% TikZ arrow styles.
\tikzset{no body/.style={/tikz/dash pattern=on 0 off 1mm}}
%%%%%%%%%%


%% DEFINIZIONI COMANDI MATEMATICI
\let\sin\relax %TOGLIE LA DEFINIZIONE SU "\sin"

% cambia la definizione di empty set
% ---
\let\oldemptyset\emptyset
% ---
% \let\emptyset\varnothing
% ---
% \let\emptyset\relax
% \newcommand{\emptyset}{\text{\textnormal{\O}}}
% ---

\DeclareMathOperator{\bounded}{bd}
\DeclareMathOperator{\sin}{sen}
\DeclareMathOperator{\epi}{Epi}
\DeclareMathOperator{\cl}{cl}
\DeclareMathOperator{\graph}{graph}
\DeclareMathOperator{\arcsec}{arcsec}
\DeclareMathOperator{\arccot}{arccot}
\DeclareMathOperator{\arccsc}{arccsc}
\DeclareMathOperator{\spettro}{Spettro}
\DeclareMathOperator{\nulls}{nullspace}
\DeclareMathOperator{\dom}{dom}
\DeclareMathOperator{\ar}{ar}
\DeclareMathOperator{\const}{Const}
\DeclareMathOperator{\fun}{Fun}
\DeclareMathOperator{\rel}{Rel}
\DeclareMathOperator{\altezza}{ht}
\let\det\relax %TOGLIE LA DEFINIZIONE SU "\det"
\DeclareMathOperator{\det}{det}
\DeclareMathOperator{\End}{End}
\DeclareMathOperator{\gl}{GL}
\DeclareMathOperator{\Id}{Id}
\DeclareMathOperator{\id}{Id}
\DeclareMathOperator{\I}{\mathds{1}}
\DeclareMathOperator{\II}{II}
\DeclareMathOperator{\rank}{rank}
\DeclareMathOperator{\tr}{tr}
\DeclareMathOperator{\tc}{t.c.}
\DeclareMathOperator{\T}{T}
\DeclareMathOperator{\var}{Var}
\DeclareMathOperator{\cov}{Cov}
\DeclareMathOperator{\st}{st}
\DeclareMathOperator{\mon}{Mon}
\newcommand{\card}[1]{\left\vert #1 \right\vert}
\newcommand{\trasposta}[1]{\prescript{\text{T}}{}{#1}}
\newcommand{\1}{\mathds{1}}
\newcommand{\R}{\mathds{R}}
\newcommand{\diesis}{\#}
\newcommand{\bemolle}{\flat}
\newcommand{\starR}{\nonstandard{\R}}
\newcommand{\borel}{\mathscr{B}}
\newcommand{\lebesgue}[1]{\mathscr{L}\left(#1\right)}
\newcommand{\media}{\mathds{E}}
\newcommand{\K}{\mathds{K}}
\newcommand{\A}{\mathds{A}}
\newcommand{\Q}{\mathds{Q}}
\newcommand{\N}{\mathds{N}}
\newcommand{\C}{\mathds{C}}
\newcommand{\Z}{\mathds{Z}}
\newcommand{\qo}{\hspace{1em}\text{q.o.}\,}
\renewcommand{\tilde}[1]{\widetilde{#1}}
\renewcommand{\parallel}{\mathrel{/\mkern-5mu/}}
\newcommand{\parti}[2][]{\wp_{#1}(#2)}
\newcommand{\diff}[1]{\operatorname{d}_{#1}}
\renewcommand{\vec}[1]{\overrightarrow{\vphantom{i}#1}}
\newcommand{\floor}[1]{\left\lfloor #1 \right\rfloor}
\newcommand{\cat}[1]{\mathbf{#1}}
\newcommand{\dfreccia}[1]{\xrightarrow{\ #1 \ }}
\newcommand{\sfreccia}[1]{\xleftarrow{\ #1 \ }}
\newcommand{\formalsum}[2]{{\sum_{#1}^{#2}}{\vphantom{\sum}}'}
\newcommand{\minim}[2]{\mu_{#1}\, \left(#2\right)}
\newcommand{\concat}{\null^{\frown}} % concatenazione di stringe

%% Definizione di \dotminus

\makeatletter
\newcommand{\dotminus}{\mathbin{\text{\@dotminus}}}

\newcommand{\@dotminus}{%
  \ooalign{\hidewidth\raise1ex\hbox{.}\hidewidth\cr$\m@th-$\cr}%
}
\makeatother

%tramite i prossimi due comandi posso decidere come scrivere i logaritmi naturali in tutti i documenti: ho infatti eliminato qualsiasi differenza tra "ln" e "log": se si vuole qualcosa di diverso bisogna inserire manualmente il tutto
\let\ln\relax
\DeclareMathOperator{\ln}{ln}
\let\log\relax
\DeclareMathOperator{\log}{log}
%%%%%%

%% NUOVI COMANDI
\newcommand{\straniero}[1]{\textit{#1}} %parole straniere
\newcommand{\titolo}[1]{\textsc{#1}} %titoli
\newcommand{\qedd}{\tag*{$\blacksquare$}} %qed per ambienti matemastici
\renewcommand{\qedsymbol}{$\blacksquare$} %modifica colore qed
\newcommand{\ooverline}[1]{\overline{\overline{#1}}}
\newcommand{\circoletto}[1]{\left(#1\right)^{\text{o}}}
%
\newcommand{\qmatrice}[1]{\begin{pmatrix}
#1_{11} & \cdots & #1_{1n}\\
\vdots & \ddots & \vdots \\
#1_{m1} & \cdots & #1_{mn}
\end{pmatrix}}
%
\newcommand{\parentesi}[2]{%
\underset{#1}{\underbrace{#2}}%
}
%
\newcommand{\norma}[1]{% Norma
\left\lVert#1\right\rVert%
}
\newcommand{\scalare}[2]{% Scalare
\left\langle #1, #2\right\rangle
}
%%%%%

%% RESTRIZIONI
\newcommand{\referenze}[2]{
	\phantomsection{}#2\textsuperscript{\textcolor{blue}{\textbf{#1}}}
}

\let\restriction\relax

\def\restriction#1#2{\mathchoice
              {\setbox1\hbox{${\displaystyle #1}_{\scriptstyle #2}$}
              \restrictionaux{#1}{#2}}
              {\setbox1\hbox{${\textstyle #1}_{\scriptstyle #2}$}
              \restrictionaux{#1}{#2}}
              {\setbox1\hbox{${\scriptstyle #1}_{\scriptscriptstyle #2}$}
              \restrictionaux{#1}{#2}}
              {\setbox1\hbox{${\scriptscriptstyle #1}_{\scriptscriptstyle #2}$}
              \restrictionaux{#1}{#2}}}
\def\restrictionaux#1#2{{#1\,\smash{\vrule height .8\ht1 depth .85\dp1}}_{\,#2}}
%%%%%%%%%%%

%% SEZIONE GRAFICA
\use{tikz}
\usetikzlibrary{matrix, patterns, calc, decorations.pathreplacing, hobby, decorations.markings, decorations.pathmorphing, babel}
\use{tikz-3dplot}
\use{mathrsfs} %per geogebra
\use{tikz-cd}
\tikzset
{
  %surface/.style={fill=black!10, shading=ball,fill opacity=0.4},
  plane/.style={black,pattern=north east lines},
  curve/.style={black,line width=0.5mm},
  dritto/.style={decoration={markings,mark=at position 0.5 with {\arrow{Stealth}}}, postaction=decorate},
  rovescio/.style={decoration={markings,mark=at position 0.5 with {\arrow{Stealth[reversed]}}}, postaction=decorate}
}
\use{pgfplots} % stampare le funzioni
	\pgfplotsset{/pgf/number format/use comma,compat=1.15}
	%\pgfplotsset{compat=1.15} %per geogebra
	\usepgfplotslibrary{fillbetween, polar}
%%%%%%

%% CITAZIONI
\use{lineno}

\newcommand{\citazione}[1]{%
  \begin{quotation}
  \begin{linenumbers}
  \modulolinenumbers[5]
  \begingroup
  \setlength{\parindent}{0cm}
  \noindent #1
  \endgroup
  \end{linenumbers}
  \end{quotation}\setcounter{linenumber}{1}
  }
%%%%%%


\use{hyperref}
\hypersetup{%
	pdfauthor={Davide Peccioli},
	pdfsubject={},
	allcolors=black,
	citecolor=black,
	colorlinks=true,
	bookmarksopen=true}
\pagestyle{empty}

\renewcommand{\href}[2]{#2}
\renewcommand{\theenumi}{\alph{enumi}}
\author{Davide Peccioli}
\date{24 aprile 2025}
\title{Esercizi TDI - Foglio 2}
\begin{document}

\maketitle
\section{Esercizio 1}
\label{sec:orgc42a61e}

Prove that for every \href{../../../../../../org/roam/20250103145124-topologia.org}{topological space} \(X\) and every \(A \subseteq X\), the following are equivalent:

\begin{enumerate}
\item The set \(A\) is \href{../../../../../../org/roam/20250417180515-insieme_mai_denso.org}{nowhere dense}, i.e. there is no open set \(U \subseteq X\) such that \(A \cap U\) is dense in \(U\).
\item The closure of \(A\) has empty \href{../../../../../../org/roam/20250122181431-parte_interna.org}{interior}.
\item There is an open dense set \(V \subseteq X\) such that \(A \cap V = \emptyset\).
\end{enumerate}

Conclude that \(B \subseteq X\) is \href{../../../../../../org/roam/20250419122752-insieme_magro.org}{comeager} if and only if it contains a countable intersection of dense open sets.
\subsection{Soluzione}
\label{sec:org0c27de0}

\subsubsection{a implica b}
\label{sec:org06f12d4}

Sia \(B\coloneqq \operatorname{Cl}_{X}(A)\), e sia per assurdo \(b \in \mathring{B}\). Allora esiste \(U \subseteq B\) aperto di \(X\) tale che \(b \in U\).

\uline{Claim}: \(A\cap U\) è denso in \(U\), ovvero \(U \subseteq \operatorname{Cl}_{X}(A\cap U)\).

Sia \(x \in U\) e sia \(V\) un intorno aperto di \(x\) in \(X\). Si vuole dimostrare che \(V\cap(A\cap U)\neq \emptyset\).
\begin{itemize}
\item L'insieme \(W\coloneqq U\cap V\) è un intorno aperto di \(x\).
\item Siccome \(x \in U \subseteq \operatorname{Cl}_{X}(A)\), allora \(A\cap W \neq \emptyset\).
\item Allora
\begin{equation*}
  	\emptyset\neq W\cap A = (V\cap U)\cap A = V\cap(A\cap U)
\end{equation*}
\end{itemize}

Per l'arbitrarietà di \(V\), si è dimostrato che \(x \in \operatorname{Cl}_{X}(A\cap U)\), ovvero che \(A\cap U\) è denso in \(U\). Questo contraddice l'ipotesi.
\subsubsection{b implica c}
\label{sec:org9ba7aac}

Sia \(V\coloneqq X\setminus \operatorname{Cl}_{X}(A)\). Allora \(V\) è \uline{denso}, in quanto il suo complementare \(\operatorname{Cl}_{X}(A)\) ha parte interna vuota (per ipotesi).

L'insieme \(V\) è aperto poiché complementare di un chiuso, e inoltre \(A\cap V=\emptyset\).
\subsubsection{c implica a}
\label{sec:org06ebdd5}

Sia per assurdo \(U \subseteq X\) un aperto non vuoto tale che \(\operatorname{Cl}_{U}(A\cap U) = U\).

Poiché \(V\) è denso in \(X\), \(U\cap V\neq \emptyset\) aperto di \(X\) e quindi aperto di \(U\). Ma
\begin{equation*}
(U\cap V)\cap (A\cap U) = U\cap (V\cap A) = \emptyset
\end{equation*}
poiché \(V\cap A=\emptyset\).

Assurdo, poiché se \(A\cap U\) è denso in \(U\), allora \(A\cap U\) incontra ogni aperto di \(U\).
\subsubsection{Caratterizzazione degli insiemi comagri}
\label{sec:org6340613}
\begin{itemize}
\item Se \(B\) è comagro, allora si può scrivere:
\begin{equation*}
B \coloneqq X\setminus \left(\bigcup_{n \in \omega} A_{n}\right)
\end{equation*}
dove \(A_{n}\) è un insieme mai denso:
\begin{equation*}
B = \bigcap_{n \in \omega} (X\setminus A_{n}).
\end{equation*}
Per ogni \(n \in \omega\) esiste \(V_{n}\) aperto denso di \(X\) tale che \(A_{n}\cap V_{n} = \emptyset\), ovvero \(V_{n} \subseteq X\setminus A_{n}\):
\begin{equation*}
  	B= \bigcap_{n \in \omega} (X\setminus A_{n}) \supseteq \bigcap_{n \in \omega} V_{n}
\end{equation*}

\item Viceversa, siano \(V_{n} \subseteq X\) insiemi aperti e densi tali che
\begin{equation*}
  B\supseteq \bigcap_{n \in \omega} V_{n}
\end{equation*}
Si ha quindi \(X\setminus B \subseteq X\setminus \left(\bigcap_{n \in \omega} V_{n}\right) = \bigcup_{n \in \omega} (X\setminus V_{n})\).

Allora \(A_{n} \coloneqq X\setminus V_{n}\) è mai denso per la caratterizzazione di cui sopra, e pertanto
\begin{equation*}
  C\coloneqq \bigcup_{n \in \omega} A_{n}
\end{equation*}
è un insieme magro. Pertanto \(X\setminus B \subseteq C\) è un insieme magro, e quindi \(B\) è un insieme comagro.\qed
\end{itemize}
\section{Esercizio 2}
\label{sec:org4799bf6}

Prove that for every topological space \(X\), the following are equivalent:
\begin{enumerate}
\item Every nonempty open subset of \(X\) is non-meager.
\item Every comeager set in \(X\) is dense.
\item The intersection of countably many dense open subsets of \(X\) is dense.
\end{enumerate}
\subsection{Soluzione}
\label{sec:org713d1db}

\subsubsection{a. implica b.}
\label{sec:orgaba9acf}

Sia \(A \subseteq X\) un insieme comagro: pertanto \(X\setminus A\) è magro. Se per assurdo \(A\) non è denso, allora esiste \(U \subseteq X\) aperto tale che \(A\cap U = \emptyset\), ovvero \(U \subseteq X\setminus A\).

Dunque \(U\) è sottoinsieme di un magro, e pertanto è magro. Assurdo.
\subsubsection{b. implica c.}
\label{sec:orgf5e16db}

Sia \(\set{U_{n}\mid n \in \omega}\) una collezione di aperti densi di \(X\), e sia, per ogni \(n \in\omega\), \(F_{n} \coloneqq X\setminus U_{n}\).

Per la caratterizzazione di cui sopra, gli \(F_{n}\) sono mai densi, e pertanto \(\bigcup_{n \in \omega} F_{n}\) è magro per definizione

Siccome
\[
X \setminus \bigcap_{n \in \omega} U_n = \bigcup_{n \in \omega} F_n
\]
allora \(\bigcap_{n \in \omega} U_{n}\) è comagro, e quindi è denso.
\subsubsection{c. implica a.}
\label{sec:orgf3adb0d}

Sia \(U \subseteq X\) aperto, magro. Per definizione, allora, esistono, per ogni \(n \in\omega\), \(A_{n} \subseteq X\) mai densi tali che
\begin{equation*}
U = \bigcup_{n \in \omega} A_{n}.
\end{equation*}

Sia quindi \(B_{n} \coloneqq X\setminus \operatorname{Cl}_{X}(A_{n})\): questo è aperto poiché complementare di un chiuso, ed è denso, in quanto il suo complementare ha interno vuoto (per la caratterizzazione dell'esercizio precedente).

Pertanto \(\bigcap_{n \in \omega} B_{n}\) è denso. Inoltre
\begin{equation*}
\bigcap_{n \in \omega} B_{n} = X\setminus \bigcup_{n \in\omega} \operatorname{Cl}_{X}(A_{n}) \subseteq X\setminus A_{n} = X\setminus U
\end{equation*}

Pertanto \(U\cap \left(\bigcap_{n \in \omega} B_{n}\right) = \emptyset\). Siccome \(\bigcap_{n \in\omega} B_{n}\) è denso, allora \(U=\emptyset\).

Segue che ogni aperto non vuoto di \(X\) è non magro.\qed
\section{Esercizio 3}
\label{sec:org624d793}

Let \(X\) be a metrizable topological space. Prove by induction on \(1 \leq \alpha < \omega_1\) that:
\begin{enumerate}
\item \(\bm{\Sigma}^0_\alpha(X)\) is closed under countable unions and finite intersections;
\item \(\bm{\Pi}^0_\alpha(X)\) is closed under countable intersections and finite unions;
\item \(\bm{\Delta}^0_\alpha(X)\) is a Boolean algebra, i.e., it is closed under complements, finite unions, and finite intersections.
\end{enumerate}
\subsection{Soluzione}
\label{sec:orgc24262c}

\subsubsection{Caso base: \(\alpha=1\)}
\label{sec:org65744de}

\begin{enumerate}
\item Unione di aperti è aperta e intersezione finita di aperti è aperta.
\item Intersezione di chiusi è chiusa e unione finita di chiusi è chiusa.
\item Il complementare di un clopen è ancora un clopen, così come unioni e intersezioni finite.
\end{enumerate}
\subsubsection{Passo induttivo}
\label{sec:orgc6b42db}

Sia l'enunciato vero per ogni \(\beta<\alpha\).
\begin{enumerate}
\item Classi additive
\label{sec:orge7d0112}

\begin{itemize}
\item Siano, per ogni \(n \in\omega\), \(A_{n} \in \bm{\Sigma}^0_\alpha(X)\). Per definizione, per ogni \(n \in\omega\), esistono degli \(A_{n}^{m} \in \bm{\Pi}^0_{\beta_{n}^{m}}(X)\), con \(\beta_{n}^{m}<\alpha\), tali che
\begin{equation*}
  A_{n} = \bigcup_{m \in \omega} A_{n}^{m}
\end{equation*}
Allora si ha che
\begin{equation*}
  \bigcup_{n \in \omega} A_{n} = \bigcup_{n,m \in \omega} A_{n}^{m}
\end{equation*}
che è ancora una unione numerabile, ed è quindi un elemento di \(\bm{\Sigma}^0_\alpha(X)\).

\item Siano \(U, V \in \bm{\Sigma}^0_\alpha(X)\). Per definizione esistono degli \(U_{n} \in \bm{\Pi}^0_{\beta^{U}_{n}}(X)\) e degli \(V_{m} \in \bm{\Pi}^0_{\beta^{V}_{m}}(X)\), con \(\beta_{n}^{U}, \beta_{m}^{V} <\alpha\) tali che
\begin{equation*}
  U = \bigcup_{n \in \omega} U_{n},\quad V = \bigcup_{m \in \omega} V_{m}
\end{equation*}

Detto \(\beta_{n,m} \coloneqq \max\set{\beta_{n}^{V},\beta_{m}^{U}}<\alpha\), si ha che
\begin{equation*}
  U\cap V = \left(\bigcup_{n \in\omega} U_{n}\right)\cap\left(\bigcup_{m \in \omega} V_{m}\right) = \bigcup_{n,m \in \omega} (U_{n}\cap V_{m})
\end{equation*}
Per ipotesi induttiva, per ogni \(n,m\) si ha \(U_{n}\cap V_{m} \in \bm{\Pi}^0_{\beta_{n,m}}(X)\) e pertanto \(U\cap V \in \bm{\Sigma}^0_\alpha(X)\)
\end{itemize}
\item Classi moltiplicative
\label{sec:orga716f9c}

\begin{itemize}
\item Siano, per ogni \(n \in\omega\), \(A_{n} \in \bm{\Pi}^0_\alpha(X)\). Per definizione, per ogni \(n \in\omega\), esistono degli \(A_{n}^{m} \in \bm{\Sigma}^0_{\beta_{n}^{m}}(X)\), con \(\beta_{n}^{m}<\alpha\), tali che
\begin{equation*}
  A_{n} = \bigcap_{m \in \omega} A_{n}^{m}
\end{equation*}
Allora si ha che
\begin{equation*}
  \bigcap_{n \in \omega} A_{n} = \bigcap_{n,m \in \omega} A_{n}^{m}
\end{equation*}
che è ancora una intersezione numerabile, ed è quindi un elemento di \(\bm{\Pi}^0_\alpha(X)\).

\item Siano \(U, V \in \bm{\Pi}^0_\alpha(X)\). Allora \((X\setminus U), (X\setminus V) \in \bm{\Sigma}^0_\alpha(X)\)
\begin{align*}
  X\setminus (U\cup V) = (X\setminus U)\cap (X\setminus V)
\end{align*}
e siccome \(\bm{\Sigma}^0_\alpha(X)\) è chiuso per intersezioni finite, allora \(X\setminus (U\cup V)\) è un elemento di \(\bm{\Sigma}^0_\alpha(X)\), ovvero
\begin{equation*}
  U\cup V \in \bm{\Pi}^0_\alpha(X).
\end{equation*}
\end{itemize}
\item Classi ambigue
\label{sec:orgdcf6ba8}

\begin{itemize}
\item Sia \(U \in \bm{\Delta}^0_\alpha(X)\). Allora \(U \in \bm{\Sigma}^0_\alpha(X)\cap \bm{\Pi}^0_\alpha(X)\), ovvero esistono
\begin{equation*}
  A_{n} \in \bm{\Pi}^0_{\beta_{n}}(X),\qquad B_{m} \in \bm{\Sigma}^0_{\beta^{m}}(X)
\end{equation*}
con \(\beta_{n},\beta^{m} <\alpha\) tali che
\begin{equation*}
  U=\bigcup_{n \in \omega} A_{n},\qquad U = \bigcap_{m \in \omega} B_{m}.
\end{equation*}

Pertanto si ha che
\begin{align*}
  X\setminus U &= X \setminus \left(\bigcup_{n \in \omega} A_{n}\right) = \bigcap_{n \in\omega} (X\setminus A_{n})\\
  X\setminus U &= X\setminus \left(\bigcap_{m \in \omega} B_{m}\right) = \bigcup_{m \in \omega} (X\setminus B_{m})
\end{align*}

Se \(A_{n} \in \bm{\Pi}^0_{\beta_{n}}(X)\) allora \(X\setminus A_{n} \in \bm{\Sigma}^0_{\beta_{n}}(X)\), e pertanto \(X\setminus U \in \bm{\Pi}^0_\alpha(X)\).

Se \(B_{m} \in \bm{\Sigma}^0_{\beta^{m}}(X)\) allora \(X\setminus B_{m} \in \bm{\Pi}^0_{\beta^{m}}(X)\), e pertanto \(X\setminus U \in \bm{\Sigma}^0_\alpha(X)\).

Dunque \(X\setminus U \in \bm{\Sigma}^0_\alpha(X)\cap \bm{\Pi}^0_\alpha(X) = \bm{\Delta}^0_\alpha(X)\).

\item Siccome sia \(\bm{\Pi}^0_\alpha(X)\) che \(\bm{\Sigma}^0_\alpha(X)\) sono chiusi per unioni e intersezioni finite, allora
\[
  \bm{\Pi}^0_\alpha(X)\cap \bm{\Sigma}^0_\alpha(X) = \bm{\Delta}^0_\alpha(X)
  \]
è chiuso per unioni e intersezioni finite.\qed
\end{itemize}
\end{enumerate}
\section{Esercizio 4}
\label{sec:org8e08f1b}

Let \(Y \subseteq X\) be Polish spaces. Show that for every \(\alpha \geq 3\),
\[
\bm{\Delta}^0_\alpha(Y) = \bm{\Delta}^0_\alpha(X) \upharpoonright Y,
\]
where as usual \(\bm{\Delta}^0_\alpha(X) \upharpoonright Y = \{ A \cap Y \mid A \in \Delta^0_\alpha(X) \}\).
\subsection{Soluzione}
\label{sec:orgada97fe}

Si richiama il Lemma 2.1.5(vi):
\begin{align*}
\bm{\Sigma}^{0}_{\alpha}(Y) &= \bm{\Sigma}^{0}_{\alpha}(X)\upharpoonright Y;\\
\bm{\Pi}^{0}_{\alpha}(Y) &= \bm{\Pi}^{0}_{\alpha}(X)\upharpoonright Y.
\end{align*}
\subsubsection{Inclusione ``\(\subseteq\)''}
\label{sec:org5dee351}

Sia \(A \in \bm{\Delta}^{0}_{\alpha}(Y) = \bm{\Sigma}^{0}_{\alpha}(Y) \cap \bm{\Pi}^{0}_{\alpha}(Y)\). Allora esistono \(B \in \bm{\Sigma}^{0}_{\alpha}(X)\) e \(C \in \bm{\Pi}^{0}_{\alpha}(X)\) tali che
\begin{equation*}
A = B\cap Y,\quad A= C\cap Y
\end{equation*}

Siccome \(Y \subseteq X\) è polacco, \href{../../../../../../org/roam/20250306134632-caratterizzazione_dei_sottoinsiemi_polacchi_di_uno_spazio_polacco.org}{allora} \(Y\) è un \href{../../../../../../org/roam/20250304152026-sottoinsiemi_gdelta_e_fsigma.org}{sottoinsieme \(\bm{G}_{\delta}\)} di \(X\), ovvero \(Y \in \bm{\Pi}^{0}_{2}(X)\). Poiché \(\alpha\ge 3\), \(\bm{\Pi}^{0}_{2}(X) \subseteq \bm{\Pi}^{0}_{\alpha}(X), \bm{\Sigma}^{0}_{\alpha}(X)\): \(Y \in \bm{\Sigma}^{0}_{\alpha}(X)\) e \(Y \in \bm{\Pi}^{0}_{\alpha}(X)\), e quindi, poiché entrambe le classi \(\bm{\Sigma}_{\alpha}^{0}(X), \bm{\Pi}^{0}_{\alpha}(X)\) sono chiuse per intersezioni finite:
\begin{equation*}
A=B\cap Y \in \bm{\Sigma}^{0}_{\alpha}(X),\qquad A=C\cap Y \in \bm{\Pi}^{0}_{\alpha}(X)
\end{equation*}
ovvero \(A \in \bm{\Delta}^{0}_{\alpha}(X)\). Inoltre \(A \subseteq Y\), e pertanto
\begin{equation*}
A = A\cap Y \in \bm{\Delta}^{0}_{\alpha}(X)\upharpoonright Y = \set{V\cap Y\mid V \in \bm{\Delta}^{0}_{\alpha}(X)}.
\end{equation*}
\subsubsection{Inclusione ``\(\supseteq\)''}
\label{sec:org72a6ff9}

Sia \(A \in \bm{\Delta}^{0}_{\alpha}(X)\), ovvero \(A\cap Y \in \bm{\Delta}^{0}_{\alpha}(X)\upharpoonright Y\).

Allora
\begin{itemize}
\item \(A \in \bm{\Sigma}^{0}_{\alpha}(X)\), e quindi \(A\cap Y \in \bm{\Sigma}^{0}_{\alpha}(Y)\);
\item \(A \in \bm{\Pi}^{0}_{\alpha}(X)\), e quindi \(A\cap Y \in \bm{\Pi}^{0}_{\alpha}(Y)\).
\end{itemize}

Pertanto
\begin{equation*}
(A\cap Y) \in \bm{\Sigma}^{0}_{\alpha}(Y)\cap \bm{\Pi}^{0}_{\alpha}(Y) = \bm{\Delta}^{0}_{\alpha}(Y).\qedd
\end{equation*}
\section{Esercizio 5}
\label{sec:org27d4296}

Given a continuous function \(f : [0, 1] \to \mathbb{R}\), let
\[
D_f = \set{ x \in [0, 1] \mid f' \text{ exists}}.
\]
(At endpoints we take one-sided derivatives.) Prove that \(D_f \in \bm{\Pi}^{0}_{3}\left([0,1]\right)\).
\subsection{Dimostrazione}
\label{sec:orgf68e587}

Si osserva che \(x \in D_{f}\) se e solo se \(x \in [0,1]\) e per ogni \(\varepsilon \in \R^{+}\) esiste \(\delta \in \R^{ +}\) tale che per ogni \(p, q \in [0,1]\):
\begin{equation*}
0<|p-x|,|q-x|< \delta
\quad\implies\quad
\left|\frac{f(p)-f(x)}{p-x}-\frac{f(q)-f(x)}{q-x}\right|\le\varepsilon
\end{equation*}
se e solo se \(x \in [0,1]\) e per ogni \(\varepsilon \in \Q^{+}\) esiste \(\delta \in \Q^{ +}\) tale che per ogni \(p, q \in [0,1]\cap \Q\):
\begin{equation*}
0<|p-x|,|q-x|< \delta
\quad\implies\quad
\left|\frac{f(p)-f(x)}{p-x}-\frac{f(q)-f(x)}{q-x}\right|\le\varepsilon
\end{equation*}
ovvero
\begin{equation*}
\left(
\begin{aligned}
0<|p-x| \,&\mathord{\wedge}\,|p-x|< \delta\\
&\mathord{\wedge}\\
0<|q-x| \,&\mathord{\wedge}\,|q-x|<\delta
\end{aligned}
\right)
\quad\implies\quad
\left|\frac{f(p)-f(x)}{p-x}-\frac{f(q)-f(x)}{q-x}\right|\le\varepsilon
\end{equation*}
ovvero
\begin{equation*}
\lnot\,\left(
\begin{aligned}
0<|p-x| \,&\mathord{\wedge}\,|p-x|< \delta\\
&\mathord{\wedge}\\
0<|q-x| \,&\mathord{\wedge}\,|q-x|<\delta
\end{aligned}
\right)
\,\lor\,
\left|\frac{f(p)-f(x)}{p-x}-\frac{f(q)-f(x)}{q-x}\right|\le\varepsilon
\end{equation*}
ovvero
\begin{equation*}
0\ge |p-x| \,\lor\, |p-x|\ge \delta \,\lor\, 0\ge|q-x| \,\lor\, |q-x|\ge \delta
\,\lor\, \left|\frac{f(p)-f(x)}{p-x}-\frac{f(q)-f(x)}{q-x}\right|\le\varepsilon.
\end{equation*}

Siano quindi
\begin{align*}
A_{p} &\coloneqq \set{x \in [0,1]\,\tc\, |p-x|=0} = \set{p}\\
B_{p,\delta} &\coloneqq \set{x \in [0,1]\,\tc\, |p-x|\ge \delta}\\
C_{q} &\coloneqq \set{x \in [0,1]\,\tc\, |q-x|=0} = \set{q}\\
D_{q,\delta} &\coloneqq \set{x \in [0,1]\,\tc\, |q-x|\ge \delta}\\
E_{p,q}^{\varepsilon} &\coloneqq \set{x \in [0,1]\,\tc\,\left|\frac{f(p)-f(x)}{p-x}-\frac{f(q)-f(x)}{q-x}\right|\le\varepsilon}
\end{align*}

Vale dunque l'uguaglianza
\begin{equation*}
D_{f} = \bigcap_{\varepsilon \in \Q^{+}} \bigcup_{\delta \in \Q^{ +}} \bigcap_{p,q \in [0,1]\cap \Q } A_{p}\cup B_{p,\delta}\cup C_{q}\cup D_{q,\delta}\cup E_{p,q}^{\varepsilon},
\end{equation*}
e pertanto:
\begin{itemize}
\item l'insieme \(V_{p,q}^{\varepsilon,\delta} \coloneqq A_{p}\cup B_{p,\delta}\cup C_{q}\cup D_{q,\delta}\cup E_{p,q}^{\varepsilon}\) è chiuso, in quanto unione di tre chiusi:
\begin{itemize}
\item \(B_{p,\delta}\) e \(D_{q,\delta}\) sono chiusi;
\item si consideri ora la funzione continua:
\begin{align*}
	F: [0,1]\setminus\set{p,q} &\longrightarrow \R\\
	x&\longmapsto \frac{f(p)-f(x)}{p-x}-\frac{f(q)-f(x)}{q-x}
\end{align*}
pertanto \(E_{p,q}^{\varepsilon} = F^{-1}\left([-\varepsilon,\varepsilon]\right)\) è un chiuso di \([0,1]\setminus\set{p,q}\); esiste quindi un \textbf{\textbf{chiuso}} \(W\) di \([0,1]\) tale che
\begin{equation*}
	E_{p,q}^{\varepsilon} = \left([0,1]\setminus\set{p,q}\right) \cap W = W\setminus\set{p,q}
\end{equation*}
per cui vale questa uguaglianza
\begin{equation*}
	W= E_{p,q}^{\varepsilon} \cup\set{p,q} = E_{p,q}^{\varepsilon} \cup A_{p}\cup C_{q};
\end{equation*}
\end{itemize}
\item l'insieme \(\bigcap_{p,q \in [0,1]\cap \Q} V_{p,q}^{\varepsilon,\delta}\) è chiuso, poiché intersezione di chiusi;
\item l'insieme \(\bigcup_{\delta \in \Q^{ +}} \bigcap_{p,q \in [0,1]\cap \Q } V_{p,q}^{\varepsilon,\delta}\) è un \(\bm{\Sigma}^0_2(X)\), poiché unione numerabile di chiusi;
\item l'insieme \(D_{f}\) è un \(\bm{\Pi}^0_3(X)\) poiché è intersezione numerabile di \(\bm{\Sigma}^0_2(X)\).\qed
\end{itemize}
\end{document}
