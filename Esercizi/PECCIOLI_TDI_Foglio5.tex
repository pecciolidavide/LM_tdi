% Created 2025-05-23 Fri 13:46
% Intended LaTeX compiler: pdflatex
\documentclass{article}
\newcommand{\use}[2][]{\usepackage[#1]{#2}}
% PACCHETTI FONDAMENTLAI
\use[utf8]{inputenc}
\use[T1]{fontenc}
\use{graphicx}
\use{longtable}
\use{wrapfig}
\use{rotating}
\use[normalem]{ulem}
\use{amsmath}
\use{amsthm}
\use{amssymb}
\use{capt-of}
\use[italian]{babel}
\use[babel]{csquotes}
\use[style=numeric, hyperref]{biblatex}
\use{microtype}
\use{lmodern}
\use{subfig} % sottofigure
\use{multicol} % due colonne
\use{lipsum} % lorem ipsum
\use{color} % colori in latex
\use{parskip} % rimuove l'indentazione dei nuovi paragrafi %% Add parbox=false to all new tcolorbox
\use{centernot}
\use[outline]{contour}\contourlength{3pt}
\use{fancyhdr}
\use{layout}
\use[most]{tcolorbox} % Riquadri colorati
\use{ifthen} % IFTHEN
\use{geometry}

% pacchetti matematica
\use{yhmath}
\use{dsfont}
\use{mathrsfs}
\use{cancel} % semplificare
\use{polynom} %divisione tra polinomi
\use{forest} % grafi ad albero
\use{booktabs} % tabelle
\use{commath} %simboli e differenziali
\use{bm} %bold
\use[fulladjust]{marginnote} %to use marginnote for date notes
\use{arrayjobx}%array
\use[intlimits]{empheq} % Riquadri colorati attorno alle equazioni
\use{mathtools}
\use{circuitikz} % Disegnare i circuiti
%%%%%%%%%%%%%


%%%% QUIVER
\newcommand{\duepunti}{\,\mathchar\numexpr"6000+`:\relax\,}
% A TikZ style for curved arrows of a fixed height, due to AndréC.
\tikzset{curve/.style={settings={#1},to path={(\tikztostart)
    .. controls ($(\tikztostart)!\pv{pos}!(\tikztotarget)!\pv{height}!270:(\tikztotarget)$)
    and ($(\tikztostart)!1-\pv{pos}!(\tikztotarget)!\pv{height}!270:(\tikztotarget)$)
    .. (\tikztotarget)\tikztonodes}},
    settings/.code={\tikzset{quiver/.cd,#1}
        \def\pv##1{\pgfkeysvalueof{/tikz/quiver/##1}}},
    quiver/.cd,pos/.initial=0.35,height/.initial=0}

% TikZ arrowhead/tail styles.
\tikzset{tail reversed/.code={\pgfsetarrowsstart{tikzcd to}}}
\tikzset{2tail/.code={\pgfsetarrowsstart{Implies[reversed]}}}
\tikzset{2tail reversed/.code={\pgfsetarrowsstart{Implies}}}
% TikZ arrow styles.
\tikzset{no body/.style={/tikz/dash pattern=on 0 off 1mm}}
%%%%%%%%%%


%% DEFINIZIONI COMANDI MATEMATICI
\let\sin\relax %TOGLIE LA DEFINIZIONE SU "\sin"

% cambia la definizione di empty set
% ---
\let\oldemptyset\emptyset
% ---
% \let\emptyset\varnothing
% ---
% \let\emptyset\relax
% \newcommand{\emptyset}{\text{\textnormal{\O}}}
% ---

\DeclareMathOperator{\bounded}{bd}
\DeclareMathOperator{\sin}{sen}
\DeclareMathOperator{\epi}{Epi}
\DeclareMathOperator{\cl}{cl}
\DeclareMathOperator{\graph}{graph}
\DeclareMathOperator{\arcsec}{arcsec}
\DeclareMathOperator{\arccot}{arccot}
\DeclareMathOperator{\arccsc}{arccsc}
\DeclareMathOperator{\spettro}{Spettro}
\DeclareMathOperator{\nulls}{nullspace}
\DeclareMathOperator{\dom}{dom}
\DeclareMathOperator{\ar}{ar}
\DeclareMathOperator{\const}{Const}
\DeclareMathOperator{\fun}{Fun}
\DeclareMathOperator{\rel}{Rel}
\DeclareMathOperator{\altezza}{ht}
\let\det\relax %TOGLIE LA DEFINIZIONE SU "\det"
\DeclareMathOperator{\det}{det}
\DeclareMathOperator{\End}{End}
\DeclareMathOperator{\gl}{GL}
\DeclareMathOperator{\Id}{Id}
\DeclareMathOperator{\id}{Id}
\DeclareMathOperator{\I}{\mathds{1}}
\DeclareMathOperator{\II}{II}
\DeclareMathOperator{\rank}{rank}
\DeclareMathOperator{\tr}{tr}
\DeclareMathOperator{\tc}{t.c.}
\DeclareMathOperator{\T}{T}
\DeclareMathOperator{\var}{Var}
\DeclareMathOperator{\cov}{Cov}
\DeclareMathOperator{\st}{st}
\DeclareMathOperator{\mon}{Mon}
\newcommand{\card}[1]{\left\vert #1 \right\vert}
\newcommand{\trasposta}[1]{\prescript{\text{T}}{}{#1}}
\newcommand{\1}{\mathds{1}}
\newcommand{\R}{\mathds{R}}
\newcommand{\diesis}{\#}
\newcommand{\bemolle}{\flat}
\newcommand{\starR}{\nonstandard{\R}}
\newcommand{\borel}{\mathscr{B}}
\newcommand{\lebesgue}[1]{\mathscr{L}\left(#1\right)}
\newcommand{\media}{\mathds{E}}
\newcommand{\K}{\mathds{K}}
\newcommand{\A}{\mathds{A}}
\newcommand{\Q}{\mathds{Q}}
\newcommand{\N}{\mathds{N}}
\newcommand{\C}{\mathds{C}}
\newcommand{\Z}{\mathds{Z}}
\newcommand{\qo}{\hspace{1em}\text{q.o.}\,}
\renewcommand{\tilde}[1]{\widetilde{#1}}
\renewcommand{\parallel}{\mathrel{/\mkern-5mu/}}
\newcommand{\parti}[2][]{\wp_{#1}(#2)}
\newcommand{\diff}[1]{\operatorname{d}_{#1}}
\let\oldvec\vec
\renewcommand{\vec}[1]{\overrightarrow{\vphantom{i}#1}}
\newcommand{\floor}[1]{\left\lfloor #1 \right\rfloor}
\newcommand{\cat}[1]{\mathbf{#1}}
\newcommand{\dfreccia}[1]{\xrightarrow{\ #1 \ }}
\newcommand{\sfreccia}[1]{\xleftarrow{\ #1 \ }}
\newcommand{\formalsum}[2]{{\sum_{#1}^{#2}}{\vphantom{\sum}}'}
\newcommand{\minim}[2]{\mu_{#1}\, \left(#2\right)}
\newcommand{\concat}{\null^{\frown}} % concatenazione di stringe

%% Definizione di \dotminus

\makeatletter
\newcommand{\dotminus}{\mathbin{\text{\@dotminus}}}

\newcommand{\@dotminus}{%
  \ooalign{\hidewidth\raise1ex\hbox{.}\hidewidth\cr$\m@th-$\cr}%
}
\makeatother

%tramite i prossimi due comandi posso decidere come scrivere i logaritmi naturali in tutti i documenti: ho infatti eliminato qualsiasi differenza tra "ln" e "log": se si vuole qualcosa di diverso bisogna inserire manualmente il tutto
\let\ln\relax
\DeclareMathOperator{\ln}{ln}
\let\log\relax
\DeclareMathOperator{\log}{log}
%%%%%%

%% NUOVI COMANDI
\newcommand{\straniero}[1]{\textit{#1}} %parole straniere
\newcommand{\titolo}[1]{\textsc{#1}} %titoli
\newcommand{\qedd}{\tag*{$\blacksquare$}} %qed per ambienti matemastici
\renewcommand{\qedsymbol}{$\blacksquare$} %modifica colore qed
\newcommand{\ooverline}[1]{\overline{\overline{#1}}}
\newcommand{\circoletto}[1]{\left(#1\right)^{\text{o}}}
%
\newcommand{\qmatrice}[1]{\begin{pmatrix}
#1_{11} & \cdots & #1_{1n}\\
\vdots & \ddots & \vdots \\
#1_{m1} & \cdots & #1_{mn}
\end{pmatrix}}
%
\newcommand{\parentesi}[2]{%
\underset{#1}{\underbrace{#2}}%
}
%
\newcommand{\norma}[1]{% Norma
\left\lVert#1\right\rVert%
}
\newcommand{\scalare}[2]{% Scalare
\left\langle #1, #2\right\rangle
}
%%%%%

%%%% Change footnote appearance
%%%%

\makeatletter
% ---- marker nel TESTO: (nota 1)
\renewcommand\@makefnmark{%
  \hbox{\normalfont\footnotesize(nota~\@thefnmark)}%
}

% ---- layout e marker diverso in PIÉ DI PAGINA: (1)
\renewcommand\@makefntext[1]{%
  \parindent 1em
  \noindent
  % qui non chiamo \@makefnmark, ma uso direttamente (\@thefnmark)
  \hb@xt@1.8em{\hss\normalfont\footnotesize(\@thefnmark)} %
  #1%
}
\makeatother
%%%%
%%%%

%% RESTRIZIONI
\newcommand{\referenze}[2]{
	\phantomsection{}#2\textsuperscript{\textcolor{blue}{\textbf{#1}}}
}

\let\restriction\relax

\def\restriction#1#2{\mathchoice
              {\setbox1\hbox{${\displaystyle #1}_{\scriptstyle #2}$}
              \restrictionaux{#1}{#2}}
              {\setbox1\hbox{${\textstyle #1}_{\scriptstyle #2}$}
              \restrictionaux{#1}{#2}}
              {\setbox1\hbox{${\scriptstyle #1}_{\scriptscriptstyle #2}$}
              \restrictionaux{#1}{#2}}
              {\setbox1\hbox{${\scriptscriptstyle #1}_{\scriptscriptstyle #2}$}
              \restrictionaux{#1}{#2}}}
\def\restrictionaux#1#2{{#1\,\smash{\vrule height .8\ht1 depth .85\dp1}}_{\,#2}}
%%%%%%%%%%%

%% SEZIONE GRAFICA
\use{tikz}
\usetikzlibrary{matrix, patterns, calc, decorations.pathreplacing, hobby, decorations.markings, decorations.pathmorphing, babel}
\use{tikz-3dplot}
\use{mathrsfs} %per geogebra
\use{tikz-cd}
\tikzset
{
  %surface/.style={fill=black!10, shading=ball,fill opacity=0.4},
  plane/.style={black,pattern=north east lines},
  curve/.style={black,line width=0.5mm},
  dritto/.style={decoration={markings,mark=at position 0.5 with {\arrow{Stealth}}}, postaction=decorate},
  rovescio/.style={decoration={markings,mark=at position 0.5 with {\arrow{Stealth[reversed]}}}, postaction=decorate}
}
\use{pgfplots} % stampare le funzioni
	\pgfplotsset{/pgf/number format/use comma,compat=1.15}
	%\pgfplotsset{compat=1.15} %per geogebra
	\usepgfplotslibrary{fillbetween, polar}
%%%%%%

%% CITAZIONI
\use{lineno}

\newcommand{\citazione}[1]{%
  \begin{quotation}
  \begin{linenumbers}
  \modulolinenumbers[5]
  \begingroup
  \setlength{\parindent}{0cm}
  \noindent #1
  \endgroup
  \end{linenumbers}
  \end{quotation}\setcounter{linenumber}{1}
  }
%%%%%%


\use{hyperref}
\hypersetup{%
	pdfauthor={Davide Peccioli},
	pdfsubject={},
	allcolors=black,
	citecolor=black,
	colorlinks=true,
	bookmarksopen=true}
\pagestyle{empty}

\renewcommand{\href}[2]{#2}
\renewcommand{\theenumi}{\alph{enumi}}
\author{Davide Peccioli}
\date{24 maggio 2025}
\title{Esercizi TDI - Foglio 5}
\begin{document}

\maketitle
\section{Esercizio 1}
\label{sec:org46ec7ec}

Let \(E\) be an equivalence relation on a Polish space \(X\). A set \(A \subseteq X\) is called \textbf{\(E\)-invariant} if \(x \in A\) and \(y \mathrel{E} x\) implies \(y \in A\), for all \(x, y \in X\). Suppose that \(E\) is analytic, that is, \(E \in \bm{\Sigma}^1_1(X^2)\). Show that if \(A, B \subseteq X\) are disjoint analytic \(E\)-invariant sets, then there is a Borel \(E\)-invariant set \(C \subseteq X\) separating \(A\) from \(B\), that is, \(A \subseteq C\) and \(C \cap B = \emptyset\).

\emph{Hint}: Recursively define sets \(A_n\), \(C_n \subseteq X\) so that \(A_0 = A\), \(C_n\) is a Borel set separating \(A_n\) from \(B\), and \(A_{n+1} \supseteq C_n\) is \(E\)-invariant, analytic, and disjoint from \(B\).
\subsection{Soluzione}
\label{sec:org437d64a}

\uline{Claim}: Esistono due famiglie \((A_{n})_{n \in \omega}, (C_{n})_{n \in \omega}\) di sottoinsiemi di \(X\), tali che
\begin{itemize}
\item \(A_{0}=A\);
\item \(\forall\, n \in\omega\): \(A_{n} \subseteq C_{n} \subseteq A_{n+1}\);
\item \(\forall\, n \in\omega\): \(C_{n} \in \bm{{\operatorname{Bor}}}(X)\) e \(C_{n}\cap B = \emptyset\)
\item \(\forall\, n \in \omega\): \(A_{n}\) è \(E\)-invariante, analitico.
\end{itemize}

Se tali famiglie esistono, sia \(C\coloneqq\bigcup_{n \in\omega} C_{n}\).
\begin{itemize}
\item \(C\) è \(E\)-invariante. Infatti, siano \(x,y \in X\), con  \(x\mathrel{E}y\). Se \(x \in C\), allora esiste \(n \in\omega\) tale che \(x \in C_{n} \subseteq A_{n+1}\); poiché \(A_{n+1}\) è \(E\)-invariante, allora \(y \in A_{n+1} \subseteq C_{n+1} \subseteq C\), e pertanto \(y \in C\).
\item \(C \in \bm{{\operatorname{Bor}}}(X)\), poiché unione numerabile di Boreliani.
\item \(A \subseteq C\); infatti \(A = A_{0} \subseteq C_{0} \subseteq C\).
\item \(C\cap B = \emptyset\), poiché ciascun \(C_{n}\) è disgiunto ta \(B\).
\end{itemize}

\uline{Dimostazione del claim}: si procede per induzione.

\begin{enumerate}
\item Sia \(A_{0}\coloneqq A\), \(E\)-invariante e analitico. Allora \(A_{0}, B \subseteq X\) sono due insiemi analitici disgiunti, e pertanto esiste, per il Teorema 3.2.1, un Boreliano \(C_{0} \subseteq X\) tale che
\begin{equation*}
 A_{0} \subseteq C_{0};\quad C_{0}\cap B = \emptyset.
\end{equation*}

\item Per il passo induttivo, si supponga di aver costruito \((A_{i})_{i\le n}\) e \((C_{i})_{i\le n}\). Si costruiscono \(A_{n+1}, C_{n+1}\).

\uline{L'insieme \(A_{n+1}\)} è definito chiudendo \(C_{n}\) rispetto alla relazione di equivalenza \(E\), ovvero
\begin{equation*}
 C_{n} \subseteq A_{n+1} \coloneqq \set{x \in X\mid \exists\,y \in C_{n}\ (x\mathrel{E}y)}.
\end{equation*}
\begin{itemize}
\item Ovviamente \(A_{n} \subseteq C_{n} \subseteq A_{n+1}\), poiché \(E\) è riflessiva.
\item \(A_{n+1}\) è \(E\)-invariante per definizione, poiché \(E\) è transitiva e simmetrica.
\item \(A_{n+1}\) è analitico, poiché \((X\times C_{n})\cap E\) è analitico, e \(A_{n+1}\) è
\begin{equation*}
   \pi_{1}\left((X\times C_{n})\cap E\right)
\end{equation*}
dove \(\pi_{1}:X\times X\to X\) è la proiezione sul primo fattore (per la proposizione 3.1.5).

L'insieme \((X\times C_{n})\cap E\) è analitico
poiché \(\bm{\Sigma}_{1}^{1}\) è chiusa per intersezioni finite e:
\begin{itemize}
\item \(E\) è analitico per ipotesi;
\item \(C_{n}\) è Boreliano per ipotesi, dunque analitico, e, detta \(\pi_{2}: X\times X\to X\) la proiezione sul secondo fattore,
\begin{equation*}
 X\times C_{n}= \pi_{2}^{-1}(C_{n})
\end{equation*}
e siccome \(\bm{\Sigma}_{1}^{1}\) è chiusa per retroimmagini continue, anche \(X\times C_{n}\) è analitico.
\end{itemize}
\item Si nota che \(A_{n+1}\cap B=\emptyset\) poiché, se per assurdo esistesse \(x \in A_{n+1}\cap B\) allora ci sarebbe \(y \in C_{n}\) tale che
\begin{equation*}
   x\mathrel{E}y
\end{equation*}
e siccome \(B\) è \(E\)-invariante, allora \(y \in B\). Dunque \(y \in B\cap C_{n} \neq\emptyset\). Assurdo.
\end{itemize}

Dunque gli insiemi \(A_{n+1}, B \subseteq X\) sono analitici e disgiunti, e pertanto esiste, per il Teorema 3.2.1, un Boreliano \(C_{n+1} \subseteq X\) tale che
\begin{equation*}
 A_{n+1} \subseteq C_{n+1}; \quad C_{n+1}\cap B =\emptyset\qedd
\end{equation*}
\end{enumerate}
\section{Esercizio 2}
\label{sec:org427ea94}

Let \(E\) be an equivalence relation on a Polish space \(X\). A \textbf{partial transversal} for \(E\) is a set \(T \subseteq X\) meeting each \(E\)-equivalence class in at most one point. Show that the following are equivalent:
\begin{enumerate}
\item \(E\) admits an uncountable analytic partial transversal;
\item \(E\) admits an uncountable Borel partial transversal;
\item there is a Borel function \(f: \R \to X\) such that \(f(r_0) \not\mathrel{E} f(r_1)\) for all distinct \(r_0, r_1 \in \R\).
\end{enumerate}
\subsection{Soluzione}
\label{sec:orgf1df743}

\subsubsection{a. implica b.}
\label{sec:org5c34a71}

\uline{Osservazione}: se \(T \subseteq X\) è un insieme trasversale parziale, allora ogni \(T' \subseteq T\) è ancora un insieme trasversale parziale.

Inoltre, \href{../../../../../../org/roam/20250522113216-ogni_insieme_analitico_non_numerabile_ammette_un_sottoinsieme_boreliano_non_numerabile.org}{ogni insieme analitico \(A\) non numerabile ammette un sottoinsieme Boreliano \(B\) non numerabile}, in quanto:
\begin{itemize}
\item siccome \(A\) è analitico, allora \(A\) ha la PSP (per il Teorema 3.4.1);
\item siccome \(A\) è non numerabile, allora esiste
\begin{equation*}
  \iota: 2^{\omega}\to A
\end{equation*}
una immersione topologica, ovvero \(\iota\) continua e iniettiva;
\item pertanto, per il Corollario 3.2.7, \(B\coloneqq\iota(2^{\omega}) \subseteq T\) è Boreliano (poiché \(2^{\omega} \in \bm{{\operatorname{Bor}}}(2^{\omega})\) e \(\iota\) iniettiva) ed è ovviamente non numerabile, poiché ha cardinalità \(2^{\aleph_{0}}>\aleph_{0}\).
\end{itemize}

Pertanto l'insieme analitico trasversale parziale \(T\) ammette un sottoinsieme Boreliano non numerabile \(T' \subseteq T\), e per l'Osservazione iniziale, \(T'\) è un insieme trasversale parziale.
\subsubsection{b. implica a.}
\label{sec:org386d496}

Questo è ovvio, poiché \(\bm{{\operatorname{Bor}}}(X) \subseteq \bm{\Sigma}_{1}^{1}(X)\) per il Corollario 3.1.4.
\subsubsection{b. implica c.}
\label{sec:org650759c}

Sia \(T' \subseteq X\) un insieme Boreliano trasversale parziale. Allora, per il Corollario 3.2.7 esiste un chiuso \(F \subseteq \omega^{\omega}\) e una funzione continua e iniettiva
\begin{equation*}
g: F \subseteq \omega^{\omega}\to X
\end{equation*}
tale che \(g(F)=T'\).

Inoltre, per il Teorema 1.3.17, esiste una biiezione continua
\begin{equation*}
h: F \subseteq \omega^{\omega}\to \R.
\end{equation*}
In particolare, per il Corollario 3.2.6, \(h\) è un Borel-isomorfismo, e pertanto \(h^{-1}: \R\to F\) è una funzione Boreliana.

Si pone quindi \(f\coloneqq g\circ h^{-1}\). Questa è una funzione Boreliana iniettiva (poiché composizione di funzioni iniettive)
\begin{equation*}
f: \R\to X.
\end{equation*}

Siano dunque \(r_{0}\neq r_{1} \in \R\). Allora \(f(r_{0})\neq f(r_{1})\), e \(f(r_{0}), f(r_{1}) \in T'\). Se per assurdo
\begin{equation*}
f(r_{0})\mathrel{E} f(r_{1})
\end{equation*}
si avrebbe che \(T'\) contiene due elementi distinti della stessa classe di \(E\)-equivalenza. Assurdo.

Pertanto, se \(r_{0}\neq r_{1} \in \R\), allora \(f(r_{0})\not\mathrel{E}f(r_{1})\).
\subsubsection{c. implica b.}
\label{sec:orgc543721}

La funzione \(f\) è necessariamente \uline{iniettiva}, poiché se per assurdo esistessero \(r_{0}\neq r_{1} \in \R\) tali che \(f(r_{0})=f(r_{1})\), allora per la \uline{riflessività} di \(E\):
\begin{equation*}
f(r_{0})\mathrel{E}f(r_{1})
\end{equation*}
e questo contraddice l'ipotesi.

Si consideri dunque \(A \subseteq \R\) non numerabile, \(A \in \bm{{\operatorname{Bor}}}(\R)\): allora \(f(A) \subseteq X\) è Boreliano per il Corollario 3.2.7, ed è inoltre un insieme trasversale parziale per \(E\): infatti se per assurdo vi fossero \(x\neq y \in f(A)\) tali che \(x\mathrel{E}y\) allora, siccome \(f\) è iniettiva, esistono \(x_{0}\neq y_{0} \in A\) tali che \(x=f(x_{0})\), \(y=f(y_{0})\), ovvero
\begin{equation*}
f(x_{0})\mathrel{E}f(y_{0}).
\end{equation*}
Questo contraddice l'ipotesi.\qed
\section{Esercizio 3}
\label{sec:org15bc569}

Let \(E\) be an equivalence relation on a Polish space \(X\). A \textbf{transversal} for \(E\) is a set \(T \subseteq X\) meeting every \(E\)-equivalence class in exactly one point. A \textbf{selector} for \(E\) is a map \(s: X \to X\) selecting one element from each \(E\)-equivalence class, that is, \(s(x) \in [x]_E\) and \(s(x) = s(y)\) if \(x \mathrel{E} y\). Show that if \(E\) is analytic, then the following are equivalent:

\begin{enumerate}
\item \(E\) admits an analytic transversal;
\item \(E\) admits a Borel transversal;
\item \(E\) admits a Borel selector.
\end{enumerate}
\subsection{Soluzione}
\label{sec:org529990d}

\subsubsection{c. implica b.}
\label{sec:org0801096}

Sia \(s:X\to X\) un selettore Boreliano per \(E\) e sia \(T\coloneqq s(X)\).

Allora \uline{\(T\) è trasversale}. Infatti incontra ogni classe di \(E\)-equivalenza esattamente una volta.
\begin{itemize}
\item \uline{Almeno una volta}: Per ogni \(x \in X\) esiste \(t \in T\) tale che \(x\mathrel{R}t\): \(t=s(x)\).
\item \uline{Al più una volta}: Siano \(x\neq y \in T\) e siano \(x_{0},y_{0} \in X\) tali che
\begin{equation*}
  s(x_{0})=x,\quad s(y_{0})=y.
\end{equation*}
Per definizione \(x\mathrel{E}x_{0}\) e \(y\mathrel{E}y_{0}\). Se per assurdo \(x\mathrel{E}y\) allora \(x_{0}\mathrel{E}y_{0}\) per transitività di \(E\). Per definizione, allora
\begin{equation*}
  s(x_{0}) = s(y_{0})
\end{equation*}
ovvero \(x=y\). Assurdo.
\end{itemize}

Inoltre, sia
\begin{align*}
f: X &\longrightarrow X\times X\\
x &\longmapsto \left(x,s(x)\right)
\end{align*}
Questa è una funzione Boreliana, poiché \(s\) è Boreliana: \(f=\operatorname{Id}_{X}\times s\) e per le proprietà di pag. 54, \(f\) è Boreliana.

Allora, detta \(D \subseteq X\times X\) la diagonale,
\begin{equation*}
D\coloneqq\set{(x,x)\mid x \in X}
\end{equation*}
si ha che \(D\) è chiuso, poiché \(X\) è metrizzabile e quindi Haussdorf. Inoltre \(T=f^{-1}(D)\)
\begin{itemize}
\item (\(\subseteq\)): Se \(t \in T\), allora \(s(t)=t\), poiché altrimenti \(s(t) \in T\) sarebbe un elemento distinto da \(t\) della classe \([t]_{E}\). Pertanto \(f(t) = \left(t,s(t)\right) = (t,t) \in D\).
\item (\(\supseteq\)): Se \(t \in f^{-1}(D)\) allora \(s(t)=t\) e quindi \(t \in s(X) = T\).
\end{itemize}

Dunque, siccome \(f\) è Boreliana e \(D\) è chiuso, \uline{\(T\) è un Boreliano}.
\subsubsection{b. implica a.}
\label{sec:orgda48351}

Questo è ovvio, poiché \(\bm{{\operatorname{Bor}}}(X) \subseteq \bm{\Sigma}_{1}^{1}(X)\) per il Corollario 3.1.4.
\subsubsection{a. implica c.}
\label{sec:org034dabe}

Sia \(T \subseteq X\) un insieme analitico trasversale per \(E\).

Siccome \(T\) è trasversale per \(E\), allora è ben definita la funzione
\begin{align*}
\varphi: X/E &\longrightarrow T\\
[x]_{E} &\longmapsto t \in [x]_{E}.
\end{align*}
poiché per ogni classe di \(E\)-equivalenza esiste un unico elemento \(t \in T\) tale che \(t \in [x]_{E}\).

Si definisce dunque la funzione \(s: X\to T: x\mapsto \varphi\left([x]_{E}\right)\). Questa è un \uline{selettore}, poiché:
\begin{itemize}
\item per ogni \(x \in X\): \(s(x) = \varphi\left([x]_{E}\right) = t \in [x]_{E}\);
\item se \(x\mathrel{E}y\) allora \([x]_{E}= [y]_{E}\) e pertanto
\begin{equation*}
  s(x) = \varphi\left([x]_{E}\right) = \varphi\left([y]_{E}\right) = s(y).
\end{equation*}
\end{itemize}

Resta da dimostrare che \(s\) sia Boreliana. Sfruttando il Teorema 3.2.4 è sufficiente dimostrare che \(\operatorname{graph}(s) \subseteq X\times X\) sia analitico. Si ha che
\begin{equation*}
\operatorname{graph}(s) = E\cap (X\times T)
\end{equation*}
infatti:
\begin{itemize}
\item se \((x,y) \in \operatorname{graph}(s)\) allora \(y=s(x)\), e poiché \(s\) è un selettore: \(x\mathrel{E} s(x)\) e quindi \((x,y) \in E\); inoltre \(x \in X\) e \(y=s(x) \in T\);
\item viceversa, se \((x,y) \in E\cap (X\times T)\) allora \(y \in T\) e \(x\mathrel{E} y\); inoltre \(y\) è l'unico elemento di \(T\) tale che \(x\mathrel{E}y\), e pertanto, per definizione \(y=s(x)\).
\end{itemize}

Sia \(T\) che \(E\) sono analitici per ipotesi. Inoltre \(X\times T = \pi_{2}^{-1}(T)\) è analitico, in quanto retroimmagine continua di un analitico (per la Proposizione 3.1.5), e dunque \(E\cap (X\times T) = \operatorname{graph}(s)\) è analitico.\qed
\section{Esercizio 4}
\label{sec:orga30ffb9}

Prove the following theorem:
\begin{quote}
Let \(X\) be a Polish space. Then every \(A \in \bm{\Pi}^1_1(X)\) can be written as \(A = \bigcup_{\xi < \omega_1} A_\xi\), where \(A_\xi\) is Borel for every \(\xi < \omega_1\).
\end{quote}
by completing the details of the following steps:

\begin{enumerate}
\item First prove the theorem for \(X = \mathrm{LO}\) and \(A = \mathrm{WO}\) as follows:
\begin{itemize}
\item Given \(\omega \leq \xi < \omega_1\), let \(\mathrm{WO}_\xi\) be the set of codes for well-orders of \(\omega\) with order type \(\leq \xi\). Show that each \(\mathrm{WO}_\xi\) is analytic.
\item Argue that there is a Borel set \(A_\xi\) such that \(\mathrm{WO}_\xi \subseteq A_\xi \subseteq \mathrm{WO}\).

\emph{Optional}: Show that \(\mathrm{WO}_\xi\) itself is Borel by showing that its complement is analytic as well.
\item Conclude that \(\mathrm{WO} = \bigcup_{\xi < \omega_1} A_\xi\).
\end{itemize}

\item Use the fact that \(\mathrm{WO}\) is \(\bm{\Pi}^1_1\)-complete to prove the theorem for \(X = \omega^\omega\) and an arbitrary \(A \in \bm{\Pi}^1_1(\omega^\omega)\).

\item Use the Borel isomorphism theorem for Polish spaces to transfer the result to an arbitrary uncountable Polish space \(X\).

\item What happens if \(X\) is a countable Polish space?
\end{enumerate}
\subsection{Soluzione}
\label{sec:org30a8dac}

\subsubsection{Parte a.}
\label{sec:org6d3ff7c}

Si consideri lo spazio polacco \(X\coloneqq\mathrm{LO} \subseteq 2^{\omega\times\omega}\) e si adotti la notazione dell'Esempio 3.1.8: l'insieme \(\mathrm{NWO}\) è analitico, mentre l'insieme \(\mathrm{WO}\) è coanalitico. È dunque possibile porre
\begin{equation*}
A\coloneqq \mathrm{WO} \in \bm{\Pi}_{1}^{1}(\mathrm{LO}).
\end{equation*}


\begin{itemize}
\item Sia \(\omega\le\xi< \omega_{1}\) fissato. Sia \(\mathrm{WO}_{\xi}\) l'insieme di tutti gli elementi di \(\mathrm{WO}\) con order type \(\le \xi\): un buon ordine \(\langle A, \preceq\rangle\) ha order type \(\xi'\) se e solo se esiste una biiezione \(f:A\to \xi'\) tale che, per ogni \(a,b \in A\)
\begin{equation*}
  	a\preceq b\quad \iff\quad f(a)< f(b)
\end{equation*}

Dunque \(x \in \mathrm{WO}\) ha order type \(\xi'\) se e solo se esiste una funzione biiettiva \(f:\omega \to\xi'\) tale che per ogni \(m,n \in \omega\):
\begin{equation*}
  	x(m,n) = 1\quad\iff\quad f(m)< f(n)
\end{equation*}

Si consideri quindi \(\mathrm{WO}^{=\xi'}\) l'insieme di tutti gli elementi di \(\mathrm{WO}\) con order type \uline{esattamente} \(\xi'\): per ogni \(x \in \mathrm{WO}\):
\begin{equation*}
  	x \in \mathrm{WO}^{=\xi'} \quad \iff \quad\exists\, f \in (\xi')^{\omega}\text{ biiettiva}\ \forall\, m,n \in\omega\ \left(x(m,n)=1\,\iff\, f(m)<f(n)\right).
\end{equation*}

Inoltre, se \(x \in \mathrm{LO}\), la condizione di destra garantisce che \(x \in \mathrm{WO}\), poiché la biiezione \(f\) è un isomorfismo di ordini e \(\xi'\) è ben ordinato (in quanto ordinale). Pertanto, per ogni \(x \in \mathrm{LO}\):
\begin{equation*}
  	x \in \mathrm{WO}^{=\xi'} \quad \iff \quad\exists\, f \in (\xi')^{\omega}\text{ biiettiva}\ \forall\, m,n \in\omega\ \left(x(m,n)=1\,\iff\, f(m)<f(n)\right).
\end{equation*}

\uline{Osservazione 1}: per ogni \(\xi' < \omega_{1}=\omega^{+}\), si ha che \(\card{\xi} =\aleph_{0}\), e pertanto \(\xi'\) è numerabile.

\uline{Osservazione 2}: per ogni \(\xi'<\omega_{1}\), \(\xi'\) è uno spazio polacco; infatti ogni ordinale numerabile è omeomorfo ad un sottoinsieme chiuso e numerabile di \(\R\) e pertanto è polacco. Siccome prodotto numerabile di spazi polacchi è ancora polacco, \((\xi')^{\omega}\) è uno spazio polacco.

Si definisce quindi:
\begin{equation*}
  	A_{m,n} \coloneqq \set{(x, f) \in \mathrm{LO}\times (\xi')^{\omega }\mid \left(x(m,n)=1 \,\iff\, f(m)<f(n)\right) \,\land\, f\text{ biiettiva}}
\end{equation*}
Questo è un insieme \(\bm{{\operatorname{Bor}}}\left(\mathrm{LO}\times(\xi')^{\omega}\right)\), poiché tutte le condizioni sono Boreliane:
\begin{align*}
  	(x,f) \in A_{m,n}\quad \iff\quad &\left[x(m,n)=1 \,\iff\, f(m)<f(n)\right] \,\land\\
  	&\land\, \left[\forall\, \lambda,\mu \in \omega\ \left(f(\lambda)= f(\mu)\right) \,\implies\,(\lambda = \mu)\right] \,\land\\
  	&\land\, \left[\forall\,\lambda<\xi'\ \exists\, k \in \omega\ \left(f(k)=\lambda\right)\right]
\end{align*}
Le quantificazioni sono tutte numerabili in virtù dell'Osservazione 1.

Pertanto
\begin{equation*}
A_{m,n} \in \bm{{\operatorname{Bor}}}\left(\mathrm{LO}\times(\xi')^{\omega}\right) \subseteq \bm{\Sigma}_{1}^{1}\left(\mathrm{LO}\times(\xi')^{\omega}\right),
\end{equation*}
e dunque anche \(\bigcap_{m,n \in \omega} A_{m,n}\) è \(\bm{\Sigma}_{1}^{1}\left(\mathrm{LO}\times(\xi')^{\omega}\right)\).

Definita
\begin{equation*}
  	\pi_{\mathrm{LO}}: \mathrm{LO} \times (\xi')^{\omega} \to \mathrm{LO}
\end{equation*}
la proiezione sul primo fattore, allora
\begin{equation*}
  	\mathrm{WO}^{=\xi'} = \pi_{\mathrm{LO}}\left(\bigcap_{m,n \in \omega} A_{m,n}\right).
\end{equation*}
Dunque applicando la Proposizione 3.1.5 (per l'osservazione precedente \((\xi')^{\omega}\) è Polacco) si ottiene che \(\mathrm{WO}^{=\xi'}\) è \(\bm{\Sigma}_{1}^{1}(\mathrm{LO})\).

Inoltre,
\begin{equation*}
  	\mathrm{WO}_{\xi} = \bigcup_{\xi'\le \xi} \mathrm{WO}^{=\xi'}
\end{equation*}
e pertanto \uline{questo dimostra che \(\mathrm{WO}_{\xi} \in \bm{\Sigma}_{1}^{1}(\mathrm{LO})\)}, poiché \(\bm{\Sigma}_{1}^{1}\) è chiuso per unioni numerabili (per la Proposizione 3.1.5) e \(\xi\) numerabile per l'Osservazione 1.

\item Sia \(\omega\le\xi< \omega_{1}\) fissato. È possibile applicare il Teorema 3.2.1 a \(\mathrm{WO}_{\xi}\)  e \(\mathrm{NWO}\) (infatti sono entrambi analitici e \(\mathrm{WO}_{\xi} \cap \mathrm{NWO} \subseteq \mathrm{WO} \cap \mathrm{NWO} =\emptyset\)): esiste \(A_{\xi}\) \uline{Boreliano} tale che:
\begin{equation*}
  \mathrm{WO}_{\xi} \subseteq A_{\xi}, \qquad A_{\xi} \cap \mathrm{NWO} = \emptyset
\end{equation*}
Siccome \(\mathrm{NWO} = X\setminus\mathrm{WO}\) si ha che \(A_{\xi} \subseteq \mathrm{WO}\):
\begin{equation*}
  \mathrm{WO}_{\xi} \subseteq A_{\xi} \subseteq \mathrm{WO}.
\end{equation*}

Per ogni \(\xi<\omega\) si pone \(A_{\xi} =\emptyset \in \bm{{\operatorname{Bor}}}(\mathrm{LO})\).
\item Vale la seguente uguaglianza: \(\mathrm{WO} = \bigcup_{\omega\le \xi<\omega_{1}} \mathrm{WO}_{\xi}\). (\(\supseteq\)): è ovvio, poiché per ogni \(\omega\le\xi<\omega_{1}\) si ha \(\mathrm{WO}_{\xi} \subseteq \mathrm{WO}\). (\(\subseteq\)): ciascun buon ordine lineare ha order type minore di \(\omega_{1}\), e pertanto se \(x \in \mathrm{WO}\) allora esiste \(\xi<\omega_{1}\) tale che \(x \in \mathrm{WO}_{\xi}\).

Pertanto si ha che
\begin{equation*}
  	\mathrm{WO} = \bigcup_{\omega\le \xi<\omega_{1}} \mathrm{WO}_{\xi} \subseteq \bigcup_{\omega\le \xi<\omega_{1}} A_{\xi} = \bigcup_{\xi<\omega_{1}} A_{\xi}
\end{equation*}
ed inoltre, per ogni \(\xi<\omega_{1}\), \(A_{\xi} \subseteq \mathrm{WO}\) e dunque
\begin{equation*}
  	\bigcup_{\xi<\omega_{1}} A_{\xi} \subseteq \mathrm{WO}
\end{equation*}

Per doppia inclusione si ha proprio \(\mathrm{WO} = \bigcup_{\xi<\omega_{1}} A_{\xi}\).
\end{itemize}
\subsubsection{Parte b.}
\label{sec:org1770568}

Sia \(X\coloneqq\omega^{\omega}\) e \(A \in \bm{\Pi}_{1}^{1}(X)\).

Siccome \(\mathrm{WO}\) è \(\bm{\Pi}_{1}^{1}\)-completo, allora esiste una funzione continua
\begin{equation*}
f: \omega^{\omega}\to \mathrm{LO}
\end{equation*}
tale che \(f^{-1}(\mathrm{WO}) = A\).

Per il punto precedente è possibile scrivere \(\mathrm{WO} = \bigcup_{\xi<\omega_{1}} B_{\xi}\) con \(B_{\xi} \in \bm{{\operatorname{Bor}}}(\mathrm{LO})\), e quindi
\begin{equation*}
A = f^{-1}(\mathrm{WO}) = f^{-1}\left(\bigcup_{\xi<\omega_{1}} B_{\xi}\right) = \bigcup_{\xi<\omega_{1}} f^{-1}(B_{\xi}).
\end{equation*}
Posto \(A_{\xi}\coloneqq f^{-1}(B_{\xi})\), si ha che \(A_{\xi} \in \bm{{\operatorname{Bor}}}(X)\) poiché \(B_{\xi} \in \bm{{\operatorname{Bor}}}(\mathrm{LO})\) e \(f\) continua. Pertanto
\begin{equation*}
A=\bigcup_{\xi<\omega_{1}} A_{\xi}
\end{equation*}
con \(A_{\xi}\) Boreliani.
\subsubsection{Parte c.}
\label{sec:org8cce25d}

Sia \(X\) uno spazio polacco non numerabile, e sia \(A \in \bm{\Pi}_{1}^1(X)\). Per il Teorema 3.2.9 esiste un isomorfismo Boreliano:
\begin{equation*}
F: \omega^{\omega}\to X
\end{equation*}

In particolare \(B\coloneqq F^{-1}(A) \in \bm{\Pi}_{1}^{1}(X)\) per il Corollario 3.1.16, poiché \(F\) è Boreliana. Per il punto precedente,
\begin{equation*}
B=\bigcup_{\xi<\omega_{1}} B_{\xi}
\end{equation*}
con \(B_{\xi} \in \bm{{\operatorname{Bor}}}(\omega^{\omega})\)

Siccome \(F\) è una biiezione, allora \(A=F(B)\):
\begin{equation*}
A= F(B) = F\left(\bigcup_{\xi<\omega_{1}} B_{\xi}\right) = \bigcup_{\xi<\omega_{1}} F(B_{\xi}).
\end{equation*}

Posto ora \(A_{\xi} \coloneqq F(B_{\xi})\), questi sono Boreliani per il Corollario 3.2.7, poiché \(F\) Boreliana iniettiva e \(B_{\xi}\) Boreliano.
\subsubsection{Parte d.}
\label{sec:org7b2d0ec}

Se \(X\) è numerabile allora il teorema è banale: ogni sottoinsieme di \(X\) è unione numerabile di singoletti, che sono chiusi, e pertanto ogni sottoinsieme di \(X\) è un Boreliano.\qed
\end{document}
