% Created 2025-05-12 Mon 13:28
% Intended LaTeX compiler: pdflatex
\documentclass{article}
\newcommand{\use}[2][]{\usepackage[#1]{#2}}
% PACCHETTI FONDAMENTLAI
\use[utf8]{inputenc}
\use[T1]{fontenc}
\use{graphicx}
\use{longtable}
\use{wrapfig}
\use{rotating}
\use[normalem]{ulem}
\use{amsmath}
\use{amsthm}
\use{amssymb}
\use{capt-of}
\use[italian]{babel}
\use[babel]{csquotes}
\use[style=numeric, hyperref]{biblatex}
\use{microtype}
\use{lmodern}
\use{subfig} % sottofigure
\use{multicol} % due colonne
\use{lipsum} % lorem ipsum
\use{color} % colori in latex
\use{parskip} % rimuove l'indentazione dei nuovi paragrafi %% Add parbox=false to all new tcolorbox
\use{centernot}
\use[outline]{contour}\contourlength{3pt}
\use{fancyhdr}
\use{layout}
\use[most]{tcolorbox} % Riquadri colorati
\use{ifthen} % IFTHEN
\use{geometry}

% pacchetti matematica
\use{yhmath}
\use{dsfont}
\use{mathrsfs}
\use{cancel} % semplificare
\use{polynom} %divisione tra polinomi
\use{forest} % grafi ad albero
\use{booktabs} % tabelle
\use{commath} %simboli e differenziali
\use{bm} %bold
\use[fulladjust]{marginnote} %to use marginnote for date notes
\use{arrayjobx}%array
\use[intlimits]{empheq} % Riquadri colorati attorno alle equazioni
\use{mathtools}
\use{circuitikz} % Disegnare i circuiti
%%%%%%%%%%%%%


%%%% QUIVER
\newcommand{\duepunti}{\,\mathchar\numexpr"6000+`:\relax\,}
% A TikZ style for curved arrows of a fixed height, due to AndréC.
\tikzset{curve/.style={settings={#1},to path={(\tikztostart)
    .. controls ($(\tikztostart)!\pv{pos}!(\tikztotarget)!\pv{height}!270:(\tikztotarget)$)
    and ($(\tikztostart)!1-\pv{pos}!(\tikztotarget)!\pv{height}!270:(\tikztotarget)$)
    .. (\tikztotarget)\tikztonodes}},
    settings/.code={\tikzset{quiver/.cd,#1}
        \def\pv##1{\pgfkeysvalueof{/tikz/quiver/##1}}},
    quiver/.cd,pos/.initial=0.35,height/.initial=0}

% TikZ arrowhead/tail styles.
\tikzset{tail reversed/.code={\pgfsetarrowsstart{tikzcd to}}}
\tikzset{2tail/.code={\pgfsetarrowsstart{Implies[reversed]}}}
\tikzset{2tail reversed/.code={\pgfsetarrowsstart{Implies}}}
% TikZ arrow styles.
\tikzset{no body/.style={/tikz/dash pattern=on 0 off 1mm}}
%%%%%%%%%%


%% DEFINIZIONI COMANDI MATEMATICI
\let\sin\relax %TOGLIE LA DEFINIZIONE SU "\sin"

% cambia la definizione di empty set
% ---
\let\oldemptyset\emptyset
% ---
% \let\emptyset\varnothing
% ---
% \let\emptyset\relax
% \newcommand{\emptyset}{\text{\textnormal{\O}}}
% ---

\DeclareMathOperator{\bounded}{bd}
\DeclareMathOperator{\sin}{sen}
\DeclareMathOperator{\epi}{Epi}
\DeclareMathOperator{\cl}{cl}
\DeclareMathOperator{\graph}{graph}
\DeclareMathOperator{\arcsec}{arcsec}
\DeclareMathOperator{\arccot}{arccot}
\DeclareMathOperator{\arccsc}{arccsc}
\DeclareMathOperator{\spettro}{Spettro}
\DeclareMathOperator{\nulls}{nullspace}
\DeclareMathOperator{\dom}{dom}
\DeclareMathOperator{\ar}{ar}
\DeclareMathOperator{\const}{Const}
\DeclareMathOperator{\fun}{Fun}
\DeclareMathOperator{\rel}{Rel}
\DeclareMathOperator{\altezza}{ht}
\let\det\relax %TOGLIE LA DEFINIZIONE SU "\det"
\DeclareMathOperator{\det}{det}
\DeclareMathOperator{\End}{End}
\DeclareMathOperator{\gl}{GL}
\DeclareMathOperator{\Id}{Id}
\DeclareMathOperator{\id}{Id}
\DeclareMathOperator{\I}{\mathds{1}}
\DeclareMathOperator{\II}{II}
\DeclareMathOperator{\rank}{rank}
\DeclareMathOperator{\tr}{tr}
\DeclareMathOperator{\tc}{t.c.}
\DeclareMathOperator{\T}{T}
\DeclareMathOperator{\var}{Var}
\DeclareMathOperator{\cov}{Cov}
\DeclareMathOperator{\st}{st}
\DeclareMathOperator{\mon}{Mon}
\newcommand{\card}[1]{\left\vert #1 \right\vert}
\newcommand{\trasposta}[1]{\prescript{\text{T}}{}{#1}}
\newcommand{\1}{\mathds{1}}
\newcommand{\R}{\mathds{R}}
\newcommand{\diesis}{\#}
\newcommand{\bemolle}{\flat}
\newcommand{\starR}{\nonstandard{\R}}
\newcommand{\borel}{\mathscr{B}}
\newcommand{\lebesgue}[1]{\mathscr{L}\left(#1\right)}
\newcommand{\media}{\mathds{E}}
\newcommand{\K}{\mathds{K}}
\newcommand{\A}{\mathds{A}}
\newcommand{\Q}{\mathds{Q}}
\newcommand{\N}{\mathds{N}}
\newcommand{\C}{\mathds{C}}
\newcommand{\Z}{\mathds{Z}}
\newcommand{\qo}{\hspace{1em}\text{q.o.}\,}
\renewcommand{\tilde}[1]{\widetilde{#1}}
\renewcommand{\parallel}{\mathrel{/\mkern-5mu/}}
\newcommand{\parti}[2][]{\wp_{#1}(#2)}
\newcommand{\diff}[1]{\operatorname{d}_{#1}}
\renewcommand{\vec}[1]{\overrightarrow{\vphantom{i}#1}}
\newcommand{\floor}[1]{\left\lfloor #1 \right\rfloor}
\newcommand{\cat}[1]{\mathbf{#1}}
\newcommand{\dfreccia}[1]{\xrightarrow{\ #1 \ }}
\newcommand{\sfreccia}[1]{\xleftarrow{\ #1 \ }}
\newcommand{\formalsum}[2]{{\sum_{#1}^{#2}}{\vphantom{\sum}}'}
\newcommand{\minim}[2]{\mu_{#1}\, \left(#2\right)}
\newcommand{\concat}{\null^{\frown}} % concatenazione di stringe

%% Definizione di \dotminus

\makeatletter
\newcommand{\dotminus}{\mathbin{\text{\@dotminus}}}

\newcommand{\@dotminus}{%
  \ooalign{\hidewidth\raise1ex\hbox{.}\hidewidth\cr$\m@th-$\cr}%
}
\makeatother

%tramite i prossimi due comandi posso decidere come scrivere i logaritmi naturali in tutti i documenti: ho infatti eliminato qualsiasi differenza tra "ln" e "log": se si vuole qualcosa di diverso bisogna inserire manualmente il tutto
\let\ln\relax
\DeclareMathOperator{\ln}{ln}
\let\log\relax
\DeclareMathOperator{\log}{log}
%%%%%%

%% NUOVI COMANDI
\newcommand{\straniero}[1]{\textit{#1}} %parole straniere
\newcommand{\titolo}[1]{\textsc{#1}} %titoli
\newcommand{\qedd}{\tag*{$\blacksquare$}} %qed per ambienti matemastici
\renewcommand{\qedsymbol}{$\blacksquare$} %modifica colore qed
\newcommand{\ooverline}[1]{\overline{\overline{#1}}}
\newcommand{\circoletto}[1]{\left(#1\right)^{\text{o}}}
%
\newcommand{\qmatrice}[1]{\begin{pmatrix}
#1_{11} & \cdots & #1_{1n}\\
\vdots & \ddots & \vdots \\
#1_{m1} & \cdots & #1_{mn}
\end{pmatrix}}
%
\newcommand{\parentesi}[2]{%
\underset{#1}{\underbrace{#2}}%
}
%
\newcommand{\norma}[1]{% Norma
\left\lVert#1\right\rVert%
}
\newcommand{\scalare}[2]{% Scalare
\left\langle #1, #2\right\rangle
}
%%%%%

%% RESTRIZIONI
\newcommand{\referenze}[2]{
	\phantomsection{}#2\textsuperscript{\textcolor{blue}{\textbf{#1}}}
}

\let\restriction\relax

\def\restriction#1#2{\mathchoice
              {\setbox1\hbox{${\displaystyle #1}_{\scriptstyle #2}$}
              \restrictionaux{#1}{#2}}
              {\setbox1\hbox{${\textstyle #1}_{\scriptstyle #2}$}
              \restrictionaux{#1}{#2}}
              {\setbox1\hbox{${\scriptstyle #1}_{\scriptscriptstyle #2}$}
              \restrictionaux{#1}{#2}}
              {\setbox1\hbox{${\scriptscriptstyle #1}_{\scriptscriptstyle #2}$}
              \restrictionaux{#1}{#2}}}
\def\restrictionaux#1#2{{#1\,\smash{\vrule height .8\ht1 depth .85\dp1}}_{\,#2}}
%%%%%%%%%%%

%% SEZIONE GRAFICA
\use{tikz}
\usetikzlibrary{matrix, patterns, calc, decorations.pathreplacing, hobby, decorations.markings, decorations.pathmorphing, babel}
\use{tikz-3dplot}
\use{mathrsfs} %per geogebra
\use{tikz-cd}
\tikzset
{
  %surface/.style={fill=black!10, shading=ball,fill opacity=0.4},
  plane/.style={black,pattern=north east lines},
  curve/.style={black,line width=0.5mm},
  dritto/.style={decoration={markings,mark=at position 0.5 with {\arrow{Stealth}}}, postaction=decorate},
  rovescio/.style={decoration={markings,mark=at position 0.5 with {\arrow{Stealth[reversed]}}}, postaction=decorate}
}
\use{pgfplots} % stampare le funzioni
	\pgfplotsset{/pgf/number format/use comma,compat=1.15}
	%\pgfplotsset{compat=1.15} %per geogebra
	\usepgfplotslibrary{fillbetween, polar}
%%%%%%

%% CITAZIONI
\use{lineno}

\newcommand{\citazione}[1]{%
  \begin{quotation}
  \begin{linenumbers}
  \modulolinenumbers[5]
  \begingroup
  \setlength{\parindent}{0cm}
  \noindent #1
  \endgroup
  \end{linenumbers}
  \end{quotation}\setcounter{linenumber}{1}
  }
%%%%%%


\use{hyperref}
\hypersetup{%
	pdfauthor={Davide Peccioli},
	pdfsubject={},
	allcolors=black,
	citecolor=black,
	colorlinks=true,
	bookmarksopen=true}
\pagestyle{empty}

\renewcommand{\href}[2]{#2}
\renewcommand{\theenumi}{\alph{enumi}}
\author{Davide Peccioli}
\date{12 maggio 2025}
\title{Esercizi TDI - Foglio 4}
\begin{document}

\maketitle
\section{Esercizio 1}
\label{sec:orga85bf6d}

Prove that for any \href{../../../../../../org/roam/20250301194013-spazio_polacco.org}{Polish space} \(X\) and \(x \in X\), the singleton \(\{x\}\) is \(\bm{\Pi}^0_1\)-complete if and only if \(x\) is not \href{../../../../../../org/roam/20250403131856-punto_isolato.org}{isolated} in \(X\). Conclude that the set
\[
  C_1 = \{x \in 2^\omega \mid \exists n \ (x(n) = 0)\}
\]
from Proposition 2.1.31 of the notes is \(\bm{\Sigma}^0_1\)-complete.
\subsection{Soluzione}
\label{sec:org0c1a504}

Siccome \(X\) è uno spazio metrizzabile, allora \(\set{x} \subseteq X\) è chiuso, e pertanto \(\set{x} \in \bm{\Pi_{1}}^{0}(X)\). Bisogna quindi dimostrare che \(\set{x}\) è \(\bm{\Pi}^{0}_{1}\)-hard sse \(x\) è \textbf{non isolato} in \(X\).
\subsubsection{Implicazione ``\(\implies\)''}
\label{sec:org02a916b}

Sia \(C \in \bm{\Pi}_{1}^{0}(\omega^{\omega})\), e sia \(f: \omega^{\omega}\to X\) continua tale che
\begin{equation*}
f^{-1}(x) = C.
\end{equation*}

Si supponga per assurdo che \(x\) sia isolato. Allora \(\set{x} \subseteq X\) è aperto, e quindi \(C \subseteq \omega^{\omega}\) è aperto (retroimmagine continua di un aperto).

Per l'arbitrarietà di \(C\), questo implica che ogni chiuso di \(\omega^{\omega}\) è un clopen. Inoltre, se \(A \subseteq \omega^{\omega}\) è aperto, allora \(\omega^{\omega}\setminus A\) è chiuso e quindi clopen, e pertanto \(A\) è un chiuso:
\begin{equation*}
\bm{\Sigma}_{1}^{0}(\omega^{\omega}) = \bm{\Delta}_{1}^{0}(\omega^{\omega}) = \bm{\Pi}_{1}^{0}(\omega^{\omega}).
\end{equation*}
Questo contraddice il Theorem 2.1.17 delle note.
\subsubsection{Implicazione ``\(\impliedby\)''\hfill{}\textsc{Modificato}}
\label{sec:org13774d8}

Sia \(x \in X\) un punto non isolato, ovvero \(x\) un punto di accumulazione di \(X\), e sia \(B \in \bm{\Pi}_{1}^{0}(\omega^{\omega})\).

\begin{itemize}
\item Si fissi \(d:X\to \R\) una metrica completa su \(X\).
\item Siccome \(x\) è un punto di accumulazione di \(X\), allora esiste una successione \((y_{n})_{n \in \omega} \subseteq X\setminus\set{x}\) tale che \(y_{n}\to x\), ovvero, per ogni intorno \(U\) di \(x\) esiste \(N \in \N\) tale che, per ogni \(j\ge N\), \(y_{j} \in U\).

\item Si costruisce \(\set{U_{n}}_{n \in \omega}\) una famiglia di aperti di \(X\) tali che
\begin{itemize}
\item per ogni \(n \in \omega\): \(U_{n}\setminus \set{x}\neq \emptyset\);
\item l'intersezione \(\bigcap_{n \in \omega} U_{n} = \set{x}\);
\item \(\operatorname{diam}(U_{n})\to 0\);
\item per ogni \(n \in \omega\): \(\operatorname{Cl}(U_{n+1}) \subsetneqq U_{n}\)
\end{itemize}
e una successione \(v_{n} \subseteq X\setminus \set{x}\) tale che \(v_{n} \in U_{n}\setminus \operatorname{Cl}(U_{n+1})\).

Sia \(U_{0} = X\). Si supponga di aver costruito \(U_{n}\), e sia \(\alpha \in U_{n}\setminus\set{x}\). Tale \(\alpha\) esiste, poiché esistono infiniti elementi della successione \((y_{j})_{j \in \omega}\) dentro \(U_{n}\) intorno di \(x\).

Detto \(r\coloneqq\min\set{2^{-n-1}, d(x,\alpha)/2}>0\), sia \(U_{n}'\coloneqq B_{d}(x,r)\). Necessariamente \(\alpha\notin U_{n}'\) e \(U_{n}' \subsetneqq U_{n}\).

È quindi possibile porre \(U_{n+1}\coloneqq B_{d}(x,r/2)\):
\begin{equation*}
  	\operatorname{Cl}(U_{n+1}) = \operatorname{Cl} \left(B_{d}(x,r/2)\right) \subseteq B_{d}^{\text{cl}}(x,r/2) \subseteq B_{d}(x,r) = U_{n}' \subsetneqq U_{n}.
\end{equation*}

Si ponga inoltre \(v_{n} \coloneqq \alpha\), \(v_{n} \in U_{n}\setminus \operatorname{Cl}(U_{n+1})\).

Questa famiglia soddisfa tutte le proprietà elencate.
\end{itemize}

\begin{itemize}
\item Siccome \(B\) è un chiuso di \(\omega^{\omega}\), allora esiste un albero potato \(T \subseteq \omega^{<\omega}\) tale che \(B=[T]\), i.e.
\begin{equation*}
  	B = \set{\alpha \in \omega^{\omega}\mid \forall\,n \in \omega\ (\alpha\upharpoonright n \in T)}
\end{equation*}
\item Si costruisce un \(\omega\)-schema \(\set{B_{s}\mid s \in \omega^{<\omega}}\) su \(X\):
\begin{itemize}
\item se \(s \in T\), allora \(B_{s} \coloneqq U_{\operatorname{lh}(s)}\); in particolare, quindi \(\emptyset \in T\) e \(B_{\emptyset} = U_{0} = X\);
\item se \(s\notin T\), sia \(j_{s}\) il più grande indice tale che \(s\upharpoonright j_{s} \in T\); si pone \(B_{s} \coloneqq \set{v_{j_{s}}}\).
\end{itemize}
\item Questo definisce effettivamente uno schema tale che \(\operatorname{Cl}(B_{s\concat a}) \subseteq B_{s}\) e ciascun \(B_{s}\neq \emptyset\): pertanto è indotta una funzione continua totale (per il Lemma 1.3.6)
\begin{equation*}
  	F:\omega^{\omega}\to X
\end{equation*}
\item Resta da mostrare che \(F^{-1}(x) = B\). Questo per definizione garantisce che \(\set{x}\) sia un \(\bm{\Pi}_{1}^{0}\)-hard.

Per ogni \(\beta \in B\),
\begin{equation*}
  	F(\beta) \in \bigcap_{n \in \omega} B_{\beta\upharpoonright n}
\end{equation*}
dove \(\beta\upharpoonright n \in T\). Quindi \(B_{\beta\upharpoonright n} = U_{n}\). Quindi
\begin{equation*}
  	F(\beta) \in \bigcap_{n \in \omega} U_{n} = \set{x}.
\end{equation*}

Viceversa, se \(\beta \notin B\), allora esiste \(n_{0} \in \omega\) tale che \(\beta\upharpoonright n_{0} \notin T\) e pertanto
\begin{equation*}
  	F(\beta) \in \bigcap_{n \in\omega} B_{\beta\upharpoonright n} \subseteq B_{\beta\upharpoonright n_{0}}
\end{equation*}
e per costruzione \(x\notin B_{\beta\upharpoonright n_{0}}\).
\end{itemize}
\subsubsection{Insieme \(C_{1}\)}
\label{sec:org7d4fb60}

Dal momento che \(\bm{\Sigma}_{1}^{0} = \check{\bm{\Pi}}_{1}^{0}\) segue che \(C_{1}\) è \(\bm{\Sigma}_{1}^{0}\)-completo se e solo se \(2^{\omega}\setminus C_1\) è \(\bm{\Pi}_{1}^{0}\)-completo.

Si ha che \(x \in 2^{\omega}\setminus C_{1}\) se e solo se per ogni \(n \in \omega\), \(x(n)\neq 0\), ovvero \(x(n)=1\).

Pertanto \(2^{\omega}\setminus C_{1} =  \set{u}\), dove
\begin{align*}
u: \omega &\longrightarrow 2\\
n &\longmapsto 1
\end{align*}

Per la caratterizzazione di cui sopra, \(C_{1}\) è \(\bm{\Sigma}_{1}^{0}\)-completo se e solo se \(u\) non è un punto isolato di \(2^{\omega}\).

Si consideri ora la successione \((x_{n})_{n \in \omega} \subseteq 2^{\omega}\):
\begin{equation*}
x_{n}(j) = \begin{cases}
1 & j<n\\
0 &j\ge n
\end{cases}
\end{equation*}
Si ha che \(x_{n}\to u\), e pertanto \(u\) non è un punto isolato di \(2^{\omega}\) (per ogni intorno \(I\) di \(u\) esiste \(N \in \omega\) tale che \(x_{N} \in I\setminus\set{u}\)). \qed
\section{Esercizio 2}
\label{sec:org8ad8b62}

Prove that for any Polish space and \(A \subseteq X\), if \(A\) is not open then it is \(\bm{\Pi}_{1}^{0}\)-hard. Conclude that a set \(A\) is truly closed (i.e. closed but not open) if and only if it is \(\bm{\Pi}_{1}^{0}\)-complete, and similarly for \(\bm{\Sigma}_{1}^{0}\).
\subsection{Soluzione}
\label{sec:orgac76c22}

\subsubsection{Non aperti sono \(\mathbf{\Pi}_{1}^{0}\)-hard\hfill{}\textsc{Modificato}}
\label{sec:org2319ee5}

Sia \(A\) un insieme non aperto, e sia \(C \subseteq \omega^{\omega}\) un chiuso fissato.

Sia dunque \(a_{0} \in A\setminus \operatorname{Int}(A)\). In particolare, quindi \(a_{0} \in \operatorname{Cl}(X\setminus A) \supseteq A\setminus \operatorname{Int}(A)\).

\begin{itemize}
\item Si fissi \(d:X\to \R\) una metrica completa su \(X\).
\item Siccome \(a_{0} \in \operatorname{Cl}(X\setminus A)\), allora esiste una successione \((y_{n})_{n \in \omega} \subseteq X\setminus A\) tale che \(y_{n}\to a_{0}\), ovvero, per ogni intorno \(U\) di \(a_{0}\) esiste \(N \in \N\) tale che, per ogni \(j\ge N\), \(y_{j} \in U\).
\item Si costruisce \(\set{U_{n}}_{n \in \omega}\) una famiglia di aperti di \(X\) tali che
\begin{itemize}
\item per ogni \(n \in \omega\): \(U_{n}\setminus \set{a_{0}}\neq \emptyset\);
\item l'intersezione \(\bigcap_{n \in \omega} U_{n} = \set{a_{0}}\);
\item \(\operatorname{diam}(U_{n})\to 0\);
\item per ogni \(n \in \omega\): \(\operatorname{Cl}(U_{n+1}) \subsetneqq U_{n}\)
\end{itemize}
e una successione \(v_{n} \subseteq X\setminus A\) tale che \(v_{n} \in U_{n}\setminus \operatorname{Cl}(U_{n+1})\).

Sia \(U_{0} = X\). Si supponga di aver costruito \(U_{n}\), e sia \(\alpha \in U_{n}\setminus A\). Tale \(\alpha\) esiste, poiché esistono infiniti elementi della successione \((y_{j})_{j \in \omega}\) dentro \(U_{n}\) intorno di \(a_{0}\), e tutti questi elementi \uline{non appartengono ad \(A\)}.

Detto \(r\coloneqq\min\set{2^{-n-1}, d(a_{0},\alpha)/2}>0\), sia \(U_{n}'\coloneqq B_{d}(a_{0},r)\). Necessariamente \(\alpha\notin U_{n}'\) e \(U_{n}' \subsetneqq U_{n}\).

È quindi possibile porre \(U_{n+1}\coloneqq B_{d}(a_{0},r/2)\):
\begin{equation*}
  	\operatorname{Cl}(U_{n+1}) = \operatorname{Cl} \left(B_{d}(a_{0},r/2)\right) \subseteq B_{d}^{\text{cl}}(a_{0},r/2) \subseteq B_{d}(a_{0},r) = U_{n}' \subsetneqq U_{n}.
\end{equation*}

Si ponga inoltre \(v_{n} \coloneqq \alpha\), \(v_{n} \in U_{n}\setminus \operatorname{Cl}(U_{n+1})\), \(v_{n}\notin A\).

Questa famiglia soddisfa tutte le proprietà elencate.

\item Siccome \(C\) è un chiuso di \(\omega^{\omega}\), allora esiste un albero potato \(T \subseteq \omega^{<\omega}\) tale che \(C=[T]\), i.e.
\begin{equation*}
  	C= \set{\alpha \in \omega^{\omega}\mid \forall\,n \in \omega\ (\alpha\upharpoonright n \in T)}
\end{equation*}
\item Si costruisce un \(\omega\)-schema \(\set{B_{s}\mid s \in \omega^{<\omega}}\) su \(X\):
\begin{itemize}
\item se \(s \in T\), allora \(B_{s} \coloneqq U_{\operatorname{lh}(s)}\); in particolare, quindi \(\emptyset \in T\) e \(B_{\emptyset} = U_{0} = X\);
\item se \(s\notin T\), sia \(j_{s}\) il più grande indice tale che \(s\upharpoonright j_{s} \in T\); si pone \(B_{s} \coloneqq \set{v_{j_{s}}}\).
\end{itemize}
\item Questo definisce effettivamente uno schema tale che \(\operatorname{Cl}(B_{s\concat a}) \subseteq B_{s}\) e ciascun \(B_{s}\neq \emptyset\): pertanto è indotta una funzione continua totale (per il Lemma 1.3.6)
\begin{equation*}
  	F:\omega^{\omega}\to X
\end{equation*}
\item Resta da mostrare che \(F^{-1}(A) = C\). Questo per definizione garantisce che \(A\) sia un \(\bm{\Pi}_{1}^{0}\)-hard.

Per ogni \(\beta \in C\),
\begin{equation*}
  	F(\beta) \in \bigcap_{n \in \omega} B_{\beta\upharpoonright n}
\end{equation*}
dove \(\beta\upharpoonright n \in T\). Quindi \(B_{\beta\upharpoonright n} = U_{n}\). Quindi
\begin{equation*}
  	F(\beta) \in \bigcap_{n \in \omega} U_{n} = \set{a_{0}} \subseteq A.
\end{equation*}

Viceversa, se \(\beta \notin C\), allora esiste \(n_{0} \in \omega\) tale che \(\beta\upharpoonright n_{0} \notin T\) e pertanto
\begin{equation*}
  	F(\beta) \in \bigcap_{n \in\omega} B_{\beta\upharpoonright n} \subseteq B_{\beta\upharpoonright n_{0}}
\end{equation*}
e per costruzione, siccome \(B_{\beta\upharpoonright n_{0}} = \set{v_{m}}\) per qualche \(m \in \omega\), e \(v_{m}\notin A\) per costruzione, allora
\begin{equation*}
  	A\cap B_{\beta\upharpoonright n_{0}} = \emptyset
\end{equation*}
e pertanto \(F(\beta)\notin A\).
\end{itemize}
\subsubsection{Caratterizzazione dei chiusi ma non aperti}
\label{sec:orgfe296a3}

\begin{itemize}
\item Se \(A\) è chiuso ma non aperto, allora \(A\) è \(\bm{\Pi}_{1}^{0}\)-hard e inoltre \(A \in \bm{\Pi}_{1}^{0}\). Per definizione, quindi \(A\) è un \(\bm{\Pi}_{1}^{0}\)-completo.

Viceversa, se \(A\) è un chiuso \(\bm{\Pi}_{1}^{0}\)-hard, si supponga per assurdo che sia aperto. Allora, per ogni \(B \in \bm{\Pi}_{1}^{0}(\omega^{\omega})\) esiste una funzione continua \(F:\omega^{\omega}\to X\) tale che \(F^{-1}(A) = B\), ovvero \(B \in \bm{\Sigma}_{1}^{0}\). Si avrebbe quindi che ogni chiuso di \(\omega^{\omega}\) sia un clopen. Come argomentato nell'\href{../../../../../../org/roam/20250505103058-caratterizzazione_dei_punti_non_isolati_di_uno_spazio_polacco.org}{esercizio precedente}, questo genera un assurdo.
\item L'insieme \(A\) è aperto ma non chiuso se e solo se \(X\setminus A\) è chiuso ma non aperto, se e solo se \(X\setminus A\) è \(\bm{\Pi}_{1}^{0}\)-completo per il punto precedente.

Per il Lemma 2.1.23, \(X\setminus A\) è \(\bm{\Pi}_{1}^{0}\)-completo se e solo se \(A\) è \(\check{\bm{\Pi}}_{1}^{0}\)-completo, ma (per l'Example 2.1.10)
\begin{equation*}
  	\check{\bm{\Pi}}_{1}^{0}=\bm{\Sigma}_{1}^{0}
\end{equation*}
e pertanto \(A\) è aperto ma non chiuso se e solo se \(A\) è \(\bm{\Sigma}_{1}^{0}\)-completo.\qed
\end{itemize}
\section{Esercizio 3}
\label{sec:org056271b}

Prove that the sets
\begin{align*}
C_0 &= c_0 \cap [0,1]^\omega = \left\{(x_n)_{n \in \omega} \in [0,1]^\omega \,\middle|\, x_n \to 0 \right\}\\
C &= \left\{(x_n)_{n \in \omega} \in [0,1]^\omega \,\middle|\, (x_n)_{n \in \omega} \text{ converges} \right\}
\end{align*}
are both \(\bm{\Pi}^0_3\)-complete.

\emph{Hint.} For the hardness part, compare these sets with the \(\bm{\Pi}^0_3\)-complete set \(C_3\) from Exercise 2.1.27 in the notes.
\subsection{Soluzione}
\label{sec:orgf63820f}

\subsubsection{\(C_{0}\) e \(C\) sono degli insiemi \(\mathbf{\Pi}_{0}^{3}\).}
\label{sec:org02e4c78}

\begin{enumerate}
\item Insieme \(C_{0}\).
\label{sec:org92a6d31}

Si ha che \((x_{j})_{j \in \omega} \in C_{0}\) se e solo se \((x_{j})_{j \in \omega} \in [0,1]^{\omega}\) e:
\begin{equation*}
	\forall\, \varepsilon \in \Q^{+}\ \exists\,N \in \N \ \forall\, n > N\ \left( |x_{n}|\le\varepsilon\right)
\end{equation*}
ovvero, se \(U_{n, \varepsilon} \coloneqq \set{(x_{j})_{j \in \omega} \in [0,1]^{\omega}: |x_{n}|\le\varepsilon}\), allora
\begin{equation*}
	C_{0} = \bigcap_{\varepsilon \in \Q^{+}} \bigcup_{N \in \N} \bigcap_{n>N} U_{n,\varepsilon}.
\end{equation*}

Quindi, dette \(\pi_{m} : [0,1]^{\omega}\to [0,1]\) le \(m\)-esime proiezioni (continue per definizione di topologia prodotto):
\begin{equation*}
	U_{n,\varepsilon}= \pi_{n}^{-1}\left([-\varepsilon,\varepsilon]\right)
\end{equation*}
e pertanto \(U_{n,\varepsilon}\) è chiuso. Per il Lemma 2.1.5:
\begin{align*}
	\bigcap_{n > N} U_{n,\varepsilon} &\in \bm{\Pi}_{1}^{0}\\
	\bigcup_{N \in \N}\bigcap_{n >N} U_{n,\varepsilon} &\in \bm{\Sigma}_{2}^{0}\\
	C_{0} = \bigcap_{\varepsilon \in \Q^{+}} \bigcup_{N \in \N} \bigcap_{n>N} U_{n,\varepsilon} &\in \bm{\Pi}_{3}^{0}.
\end{align*}
e si ottiene che \(C_{0} \in \bm{\Pi}_{3}^{0}\left([0,1]^{\omega}\right)\).
\item Insieme \(C\).
\label{sec:orgbb6026c}

Si ha che \((x_{j})_{j \in\omega} \in C\) se e solo se \((x_{j})_{j \in\omega} \in [0,1]^{\omega}\) e
\begin{equation*}
\forall\, \varepsilon \in \Q^{+}\ \exists\, N \in \N \ \forall\,n,m> N\ (|x_{n}-x_{m}|\le\varepsilon)
\end{equation*}
ovvero, se \(V_{m,n}^{\varepsilon} \coloneqq \set{(x_{j})_{j \in \omega} \in [0,1]^{\omega}: |x_{n}-x_{m}|\le\varepsilon}\), allora
\begin{equation*}
C = \bigcap_{\varepsilon \in \Q^{+}}\bigcup_{N \in \N}\bigcap_{n,m>N} V_{n,m}^{\varepsilon}.
\end{equation*}

Poiché la funzione \((\pi_{n}-\pi_{m}):[0,1]^{\omega}\to \R\) è continua, allora
\begin{equation*}
V_{n,m}^{\varepsilon} \coloneqq (\pi_{n}-\pi_{m})^{-1}\left([-\varepsilon,\varepsilon]\right)
\end{equation*}
e quindi \(V_{n,m}^{\varepsilon}\) è chiuso. Per il Lemma 2.1.5:
\begin{align*}
\bigcap_{n,m > N} V_{n,m}^{\varepsilon} &\in \bm{\Pi}_{1}^{0}\\
\bigcup_{N \in \N}\bigcap_{n,m > N} V_{n,m}^{\varepsilon} &\in \bm{\Sigma}_{2}^{0}\\
C = \bigcap_{\varepsilon \in \Q^{+}}\bigcup_{N \in \N}\bigcap_{n,m>N} V_{n,m}^{\varepsilon} &\in \bm{\Pi}_{3}^{0}
\end{align*}
e si ottiene che \(C \in \bm{\Pi}_{3}^{0}\left([0,1]^{\omega}\right)\).
\end{enumerate}
\subsubsection{Hardness}
\label{sec:org6873fb5}

È noto (Esercizio 2.1.27) che l'insieme \(C_{3} \coloneqq \set{x \in \omega^{\omega}\mid \lim_{n\to\infty}x(n) = \infty}\) sia \(\bm{\Pi}_{3}^{0}\)-hard. Pertanto si cercano delle funzioni continue
\begin{equation*}
\begin{tikzcd}[ampersand replacement=\&,cramped]
	{\omega^\omega} \& {[0,1]^\omega} \& {[0,1]^\omega}
	\arrow["F", from=1-1, to=1-2]
	\arrow["G", from=1-2, to=1-3]
\end{tikzcd}
\end{equation*}
tali che
\begin{equation*}
F^{-1}(C_{0}) = C_{3},\qquad G^{-1}(C) = C_{0}.
\end{equation*}
Questo, per mezzo del Lemma 2.1.23, garantisce che \(C_{0},C\) siano insiemi \(\bm{\Pi}_{3}^{0}\)-hard (e quindi, per il punto precedente, completi).

Le due funzioni si definiscono come segue:
\begin{align*}
&\begin{aligned}
F: \omega^{\omega} &\longrightarrow[0,1]^{\omega}\\
(x_{j})_{j \in \omega} &\longmapsto \left(\phi(x_{j})\right)_{j \in \omega}
\end{aligned} & &\text{dove} & &\begin{aligned}
\phi: \N &\longrightarrow [0,1]\\
m &\longmapsto \begin{cases}
1/m & m\neq 0\\
1 & m=0.
\end{cases}
\end{aligned}\\[1.5em]
&\begin{aligned}
G: [0,1]^{\omega} &\longrightarrow [0,1]^{\omega}\\
(x_{j})_{j \in \omega} &\longmapsto (y_{j})_{j \in \omega}
\end{aligned} & &\text{dove} &
&y_{j} \coloneqq \begin{cases}
0 & j\text{ dispari}\\
x_{j/2} & j\text{ pari}.
\end{cases}
\end{align*}
\begin{enumerate}
\item \underline{\(F\) è continua.}
\label{sec:org215824a}

La funzione \(F\) è continua poiché lo è su ciascuna componente (in quanto \(\N\) ha la topologia discreta).
\item \underline{\(G\) è continua.}
\label{sec:orgf3e2a55}

La funzione \(F\) è continua poiché lo è su ciascuna componente:
\begin{itemize}
\item la componente \(j\)-esima di \(G\), con \(j\) dispari, è data dalla funzione costante nulla, continua;
\item la componente \(j\)-esima di \(G\), con \(j\) pari, è data dalla funzione proiezione \(\pi_{j/2}: [0,1]^{\omega}\to [0,1]\), continua per definizione di topologia prodotto.
\end{itemize}
\item \underline{\(F^{-1}(C_{0})=C_{3}\).}
\label{sec:orgf6404b8}

Si dimostra che \(\alpha \in C_{3}\) sse \(F(\alpha) \in C_{0}\).

\begin{itemize}
\item Se \(\alpha = (x_{j})_{j \in \omega}\in C_{3}\) allora esiste \(N \in \N\) tale che, per ogni \(j>N\) si ha\(x_{j}\neq 1\).

Pertanto, per ogni \(j>N\), \(\phi(x_{j}) = 1/x_{j}\) e, siccome \(x_{j}\to \infty\), \(\phi(x_{j})\to 0\). Quindi \(F(\alpha) \in C_{0}\).
\item Viceversa, sia \(\alpha = (x_{j})_{j \in \omega} \notin C_{3}\). Si supponga per assurdo che \((y_{j})_{j \in \omega} = F(\alpha) \in C_{0}\).

Allora, definitivamente, \(y_{j} = 1/x_{j}\) (e in particolare \(x_{j}\neq 0\neq y_{j}\)), poiché altrimenti non si avrebbe convergenza a \(0\). In particolare, \(x_{j} = 1/y_{j}\), definitivamente:
\begin{equation*}
  \lim_{j\to \infty} x_{j} = \lim_{j\to\infty}\frac{1}{y_{j}} = \infty
\end{equation*}
poiché \(y_{j}\to 0\). Quindi \((x_{j})_{j \in \omega} \in C_{3}\). Assurdo.

Si ottiene perciò che \(F(\alpha) \notin C_{0}\).
\end{itemize}
\item \underline{\(G^{-1}(C)=C_{0}\).}
\label{sec:orgd18b9e9}

Si dimostra che \(\alpha \in C_{0}\) sse \(G(\alpha) \in C\).

\begin{itemize}
\item Se \(\alpha = (x_{j})_{j \in \omega} \in C_{0}\) allora la successione \(\beta= (y_{j})_{j \in \omega} \coloneqq G(\alpha)\) converge a \(0\), e pertanto converge: \(G(\alpha) \in C\).
\item Viceversa, se \(\alpha = (x_{j})_{j \in \omega}\notin C_{0}\) , allora la successione \(\beta= (y_{j})_{j \in \omega} \coloneqq G(\alpha)\) non converge, in quanto presenta due sottosuccessioni (\((y_{2j+1})_{j \in \omega}\) e \((y_{2j})_{j \in \omega}\)) con caratteri diversi: \(G(\alpha)\notin C\).\qed
\end{itemize}
\end{enumerate}
\section{Esercizio 4}
\label{sec:org4a4a996}

Prove that for any \(0 < p < \infty\) the set
\[
\ell^p \cap [0,1]^\omega =
\set{(x_n)_{n \in \omega} \in [0,1]^\omega
\mid
\|x\|_p = \left( \sum_{n=0}^\infty |x_n|^p \right)^{1/p} < \infty}
\]
is \(\bm{\Sigma}^0_2\)-complete.

\emph{Hint.} Recall that a series of positive terms converges if and only if the sequence of partial sums is bounded from above. For the hardness part, compare this set with the \(\bm{\Sigma}^0_2\)-complete set \(Q_2\) from the notes.
\subsection{Soluzione}
\label{sec:orga542720}

\subsubsection{Insieme \(\mathbf{\Sigma}^{0}_{2}\)}
\label{sec:orgb728007}

Sia \(x=(x_{j})_{j \in \omega} \in [0,1]^{\omega}\).

Si ha che \((x_{j})_{j \in \omega} \in \ell^{p}\cap[0,1]^{\omega}\) se e solo se
\begin{equation*}
\norma{x}_{p} = \left(\sum_{n=0}^{\infty}|x_{n}|^{p}\right)^{1/p}<\infty
\end{equation*}
se e solo se
\begin{equation*}
(\norma{x}_{p})^{p} = \sum_{n=0}^{\infty}|x_{n}|^{p} < \infty
\end{equation*}
se e solo se, sfruttando l'hint,
\begin{equation*}
\exists\, L \in \Q^{+}\ \forall\, N \in \N \ \left(\sum_{n=0}^{N} |x_{n}|^{p}\right) \le L
\end{equation*}

Sia dunque
\begin{align*}
G_{N}^{p}: [0,1]^{\omega} &\longrightarrow \R\\
(x_{j})_{j \in \omega} &\longmapsto \sum_{n=0}^{N} |x_{n}|^{p}
\end{align*}
Questa è una mappa continua, poiché composizione di mappe continue (proiezioni, continue per la definizione di topologia prodotto, e somma finita ed elevamento a potenza) e pertanto il seguente è un insieme chiuso:
\begin{equation*}
V_{L}^{N} \coloneqq \set{(x_{j})_{j \in \omega} \in [0,1]^{\omega}\mid \left(\sum_{n=0}^{N} |x_{n}|^{p}\right) \le L} = (G_{N}^{p})^{-1}\left([0,L]\right).
\end{equation*}

In definitiva
\begin{align*}
V_{L}^{N} &\in \bm{\Pi}_{1}^{0}\\
\bigcap_{N \in \N} V_{L}^{N} &\in \bm{\Pi}_{1}^{0}\\
\ell^{p}\cap[0,1]^{\omega} = \bigcup_{L \in \Q}\bigcap_{N \in \N} V_{L}^{N} &\in \bm{\Sigma}_{2}^{0}.
\end{align*}
\subsubsection{Insieme \(\mathbf{\Sigma}^{0}_{2}\)-hard}
\label{sec:orgd214343}

È noto che l'insieme
\begin{equation*}
Q_{2}\coloneqq \set{x \in 2^{\omega}\mid
\exists\, n \in \N\ \forall\, k \ge n\ \left(x(k) = 0\right)
}
\end{equation*}
sia \(\bm{\Sigma}_{2}^{0}\)-hard.

Si vuole quindi trovare una funzione continua
\begin{equation*}
F: 2^{\omega} \longrightarrow [0,1]^{\omega}
\end{equation*}
tale che \(F^{-1}\left(\ell^{p}\cap[0,1]^{\omega}\right)=Q_{2}\). Questo, per il Lemma 2.1.23, garantisce che \(\ell^{p}\cap[0,1]^{\omega}\) sia \(\bm{\Sigma}_{2}^{0}\)-hard, e quindi \(\bm{\Sigma}_{2}^{0}\)-completo.

\begin{itemize}
\item Considerando che \(2=\set{0,1} \subseteq [0,1]\), si può definire \(F\) come l'inclusione, ovvero
\begin{align*}
  F: 2^{\omega} &\longrightarrow [0,1]^{\omega}\\
  (x_{j})_{j \in \omega} &\longmapsto (x_{j})_{j \in \omega}
\end{align*}
\item Questa è una funzione continua, poiché è continua su ciascuna componente (infatti \(\set{0,1}\) ha la topologia di sottospazio rispetto a \([0,1]\), e per definizione quindi l'inclusione è continua).
\item Inoltre, \(F^{-1}\left(\ell^{p}\cap[0,1]^{\omega}\right) = Q_{2}\). In particolare, si dimostra che \(\alpha \in Q_{2}\) sse \(F(\alpha) \in \ell^{p}\cap[0,1]^{\omega}\)
\begin{itemize}
\item Sia \(\alpha = (x_{j})_{j \in \omega} \in Q_{2}\). Allora esiste \(N \in \N\) tale che \(x_{j}=0\) per ogni \(j>N\), e pertanto
\begin{equation*}
	\left(\sum_{j=0}^{\infty}|x_{j}|^{p}\right)^{1/p} = \left(\sum_{j=0}^{N} |x_{j}|^{p}\right)^{1/p}<\infty
\end{equation*}
Pertanto \(F(\alpha) \in \ell^{p}\cap[0,1]^{\omega}\).
\item Sia \(\alpha = (x_{j})_{j \in \omega} \notin Q_{2}\). Allora per ogni \(n \in \N\) esiste \(k_{n}\ge n\) tale che \(x_{k_{n}} = 1\). Pertanto, per ogni \(n \in \N\), esiste un numero infinito di indici \(j\) tali che \(x_{j}=1\), e dunque \(\lim_{j\to\infty} x_{j}\neq 0\) e dunque la serie
\begin{equation*}
	\sum_{j=0}^{\infty} |x_{j}|^{p}
\end{equation*}
diverge. Pertanto \(F(\alpha)\notin \ell^{p}\cap[0,1]^{\omega}\).\qed
\end{itemize}
\end{itemize}
\section{Esercizio 5}
\label{sec:orgdc6fc4f}

Show that the collection of all sequences \((x_n)_{n \in \omega} \in [0,1]^\omega\) having an irrational accumulation point is analytic.
\subsection{Soluzione}
\label{sec:org8569a5e}

Sia \(A_{[0,1]\setminus\Q}\) l'insieme di tutti gli \((x_{j})_{j \in \omega} \in [0,1]^{\omega}\) con un punto di accumulazione irrazionale.

Si ricorda che \(p \in [0,1]\) è un punto di accumulazione per \((x_{j})_{j \in \omega}\), per definizione, se:
\begin{equation*}
\forall\, \varepsilon > 0 \ \forall\, N \in \N\ \exists\, n > N\ \left(x_{n} \in (p-\varepsilon,p+\varepsilon)\right).
\end{equation*}

In particolare, \(p \in [0,1]\) è un punto di accumulazione per \((x_{j})_{j \in \omega}\) se e solo se:
\begin{equation*}
\forall\, \varepsilon \in \Q^{+} \ \forall\, N \in \N\ \exists\, n > N\ \left(x_{n} \in (p-\varepsilon,p +\varepsilon)\right).
\end{equation*}

Per il Remark 3.1.10, quindi, siccome \([0,1]\setminus \Q\) è uno spazio polacco, \(A_{[0,1]\setminus \Q}\) è un insieme analitico, in quanto definito dalla seguente formula:
\begin{equation*}
\exists\, p \in [0,1]\setminus \Q\ \forall\, \varepsilon \in \Q^{+} \ \forall\, N \in \N\ \exists\, n > N\ \left(x_{n} \in (p-\varepsilon,p +\varepsilon)\right)
\end{equation*}
composta unicamente (tranne che per il primo esistenziale), da quantificazioni numerabili, e da una formula atomica: \(x_{n} \in (p-\varepsilon,p+\varepsilon)\), che definisce un boreliano di \(([0,1]\setminus \Q)\times [0,1]^{\omega}\), in quanto, data la funzione continua
\begin{align*}
F_{n}: ([0,1]\setminus \Q)\times [0,1]^{\omega}  &\longrightarrow \R\\
\left(p,(x_{j})_{j \in \omega}\right) &\longmapsto x_{n}-p
\end{align*}
si ha che
\begin{equation*}
\set{\left(p,(x_{j})_{j \in \omega}\right) \in ([0,1]\setminus \Q)\times [0,1]^{\omega}\mid x_{n} \in (p-\varepsilon,p+\varepsilon)} = F_{n}^{-1}\left[(-\varepsilon,\varepsilon)\right]
\end{equation*}
è un aperto.\qed
\end{document}
