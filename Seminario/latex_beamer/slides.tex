% Created 2025-06-03 Tue 12:33
% Intended LaTeX compiler: pdflatex
\documentclass[babel]{beamer}

\institute[]{Università degli Studi di Torino}
\newcommand{\use}[2][]{\usepackage[#1]{#2}}
% PACCHETTI FONDAMENTLAI
\use[utf8]{inputenc}
\use[T1]{fontenc}
\use{graphicx}
\use{longtable}
\use{wrapfig}
\use{rotating}
\use[normalem]{ulem}
\use{amsmath}
\use{amsthm}
\use{amssymb}
\use{capt-of}
\use[italian]{babel}
\use[babel]{csquotes}
\use[style=numeric, hyperref]{biblatex}
\use{microtype}
\use{lmodern}
\use{subfig} % sottofigure
\use{multicol} % due colonne
\use{lipsum} % lorem ipsum
\use{color} % colori in latex
\use{parskip} % rimuove l'indentazione dei nuovi paragrafi %% Add parbox=false to all new tcolorbox
\use{centernot}
\use[outline]{contour}\contourlength{3pt}
\use{fancyhdr}
\use{layout}
\use[most]{tcolorbox} % Riquadri colorati
\use{ifthen} % IFTHEN
\use{geometry}

% pacchetti matematica
\use{yhmath}
\use{dsfont}
\use{mathrsfs}
\use{cancel} % semplificare
\use{polynom} %divisione tra polinomi
\use{forest} % grafi ad albero
\use{booktabs} % tabelle
\use{commath} %simboli e differenziali
\use{bm} %bold
\use[fulladjust]{marginnote} %to use marginnote for date notes
\use{arrayjobx}%array
\use[intlimits]{empheq} % Riquadri colorati attorno alle equazioni
\use{mathtools}
\use{circuitikz} % Disegnare i circuiti
%%%%%%%%%%%%%


%%%% QUIVER
\newcommand{\duepunti}{\,\mathchar\numexpr"6000+`:\relax\,}
% A TikZ style for curved arrows of a fixed height, due to AndréC.
\tikzset{curve/.style={settings={#1},to path={(\tikztostart)
    .. controls ($(\tikztostart)!\pv{pos}!(\tikztotarget)!\pv{height}!270:(\tikztotarget)$)
    and ($(\tikztostart)!1-\pv{pos}!(\tikztotarget)!\pv{height}!270:(\tikztotarget)$)
    .. (\tikztotarget)\tikztonodes}},
    settings/.code={\tikzset{quiver/.cd,#1}
        \def\pv##1{\pgfkeysvalueof{/tikz/quiver/##1}}},
    quiver/.cd,pos/.initial=0.35,height/.initial=0}

% TikZ arrowhead/tail styles.
\tikzset{tail reversed/.code={\pgfsetarrowsstart{tikzcd to}}}
\tikzset{2tail/.code={\pgfsetarrowsstart{Implies[reversed]}}}
\tikzset{2tail reversed/.code={\pgfsetarrowsstart{Implies}}}
% TikZ arrow styles.
\tikzset{no body/.style={/tikz/dash pattern=on 0 off 1mm}}
%%%%%%%%%%


%% DEFINIZIONI COMANDI MATEMATICI
\let\sin\relax %TOGLIE LA DEFINIZIONE SU "\sin"

% cambia la definizione di empty set
% ---
\let\oldemptyset\emptyset
% ---
% \let\emptyset\varnothing
% ---
% \let\emptyset\relax
% \newcommand{\emptyset}{\text{\textnormal{\O}}}
% ---

\DeclareMathOperator{\bounded}{bd}
\DeclareMathOperator{\sin}{sen}
\DeclareMathOperator{\epi}{Epi}
\DeclareMathOperator{\cl}{cl}
\DeclareMathOperator{\graph}{graph}
\DeclareMathOperator{\arcsec}{arcsec}
\DeclareMathOperator{\arccot}{arccot}
\DeclareMathOperator{\arccsc}{arccsc}
\DeclareMathOperator{\spettro}{Spettro}
\DeclareMathOperator{\nulls}{nullspace}
\DeclareMathOperator{\dom}{dom}
\DeclareMathOperator{\ar}{ar}
\DeclareMathOperator{\const}{Const}
\DeclareMathOperator{\fun}{Fun}
\DeclareMathOperator{\rel}{Rel}
\DeclareMathOperator{\altezza}{ht}
\let\det\relax %TOGLIE LA DEFINIZIONE SU "\det"
\DeclareMathOperator{\det}{det}
\DeclareMathOperator{\End}{End}
\DeclareMathOperator{\gl}{GL}
\DeclareMathOperator{\Id}{Id}
\DeclareMathOperator{\id}{Id}
\DeclareMathOperator{\I}{\mathds{1}}
\DeclareMathOperator{\II}{II}
\DeclareMathOperator{\rank}{rank}
\DeclareMathOperator{\tr}{tr}
\DeclareMathOperator{\tc}{t.c.}
\DeclareMathOperator{\T}{T}
\DeclareMathOperator{\var}{Var}
\DeclareMathOperator{\cov}{Cov}
\DeclareMathOperator{\st}{st}
\DeclareMathOperator{\mon}{Mon}
\newcommand{\card}[1]{\left\vert #1 \right\vert}
\newcommand{\trasposta}[1]{\prescript{\text{T}}{}{#1}}
\newcommand{\1}{\mathds{1}}
\newcommand{\R}{\mathds{R}}
\newcommand{\diesis}{\#}
\newcommand{\bemolle}{\flat}
\newcommand{\starR}{\nonstandard{\R}}
\newcommand{\borel}{\mathscr{B}}
\newcommand{\lebesgue}[1]{\mathscr{L}\left(#1\right)}
\newcommand{\media}{\mathds{E}}
\newcommand{\K}{\mathds{K}}
\newcommand{\A}{\mathds{A}}
\newcommand{\Q}{\mathds{Q}}
\newcommand{\N}{\mathds{N}}
\newcommand{\C}{\mathds{C}}
\newcommand{\Z}{\mathds{Z}}
\newcommand{\qo}{\hspace{1em}\text{q.o.}\,}
\renewcommand{\tilde}[1]{\widetilde{#1}}
\renewcommand{\parallel}{\mathrel{/\mkern-5mu/}}
\newcommand{\parti}[2][]{\wp_{#1}(#2)}
\newcommand{\diff}[1]{\operatorname{d}_{#1}}
\let\oldvec\vec
\renewcommand{\vec}[1]{\overrightarrow{\vphantom{i}#1}}
\newcommand{\floor}[1]{\left\lfloor #1 \right\rfloor}
\newcommand{\cat}[1]{\mathbf{#1}}
\newcommand{\dfreccia}[1]{\xrightarrow{\ #1 \ }}
\newcommand{\sfreccia}[1]{\xleftarrow{\ #1 \ }}
\newcommand{\formalsum}[2]{{\sum_{#1}^{#2}}{\vphantom{\sum}}'}
\newcommand{\minim}[2]{\mu_{#1}\, \left(#2\right)}
\newcommand{\concat}{\null^{\frown}} % concatenazione di stringe

%% Definizione di \dotminus

\makeatletter
\newcommand{\dotminus}{\mathbin{\text{\@dotminus}}}

\newcommand{\@dotminus}{%
  \ooalign{\hidewidth\raise1ex\hbox{.}\hidewidth\cr$\m@th-$\cr}%
}
\makeatother

%tramite i prossimi due comandi posso decidere come scrivere i logaritmi naturali in tutti i documenti: ho infatti eliminato qualsiasi differenza tra "ln" e "log": se si vuole qualcosa di diverso bisogna inserire manualmente il tutto
\let\ln\relax
\DeclareMathOperator{\ln}{ln}
\let\log\relax
\DeclareMathOperator{\log}{log}
%%%%%%

%% NUOVI COMANDI
\newcommand{\straniero}[1]{\textit{#1}} %parole straniere
\newcommand{\titolo}[1]{\textsc{#1}} %titoli
\newcommand{\qedd}{\tag*{$\blacksquare$}} %qed per ambienti matemastici
\renewcommand{\qedsymbol}{$\blacksquare$} %modifica colore qed
\newcommand{\ooverline}[1]{\overline{\overline{#1}}}
\newcommand{\circoletto}[1]{\left(#1\right)^{\text{o}}}
%
\newcommand{\qmatrice}[1]{\begin{pmatrix}
#1_{11} & \cdots & #1_{1n}\\
\vdots & \ddots & \vdots \\
#1_{m1} & \cdots & #1_{mn}
\end{pmatrix}}
%
\newcommand{\parentesi}[2]{%
\underset{#1}{\underbrace{#2}}%
}
%
\newcommand{\norma}[1]{% Norma
\left\lVert#1\right\rVert%
}
\newcommand{\scalare}[2]{% Scalare
\left\langle #1, #2\right\rangle
}
%%%%%

%%%% Change footnote appearance
%%%%

\makeatletter
% ---- marker nel TESTO: (nota 1)
\renewcommand\@makefnmark{%
  \hbox{\normalfont\footnotesize(nota~\@thefnmark)}%
}

% ---- layout e marker diverso in PIÉ DI PAGINA: (1)
\renewcommand\@makefntext[1]{%
  \parindent 1em
  \noindent
  % qui non chiamo \@makefnmark, ma uso direttamente (\@thefnmark)
  \hb@xt@1.8em{\hss\normalfont\footnotesize(\@thefnmark)} %
  #1%
}
\makeatother
%%%%
%%%%

%% RESTRIZIONI
\newcommand{\referenze}[2]{
	\phantomsection{}#2\textsuperscript{\textcolor{blue}{\textbf{#1}}}
}

\let\restriction\relax

\def\restriction#1#2{\mathchoice
              {\setbox1\hbox{${\displaystyle #1}_{\scriptstyle #2}$}
              \restrictionaux{#1}{#2}}
              {\setbox1\hbox{${\textstyle #1}_{\scriptstyle #2}$}
              \restrictionaux{#1}{#2}}
              {\setbox1\hbox{${\scriptstyle #1}_{\scriptscriptstyle #2}$}
              \restrictionaux{#1}{#2}}
              {\setbox1\hbox{${\scriptscriptstyle #1}_{\scriptscriptstyle #2}$}
              \restrictionaux{#1}{#2}}}
\def\restrictionaux#1#2{{#1\,\smash{\vrule height .8\ht1 depth .85\dp1}}_{\,#2}}
%%%%%%%%%%%

%% SEZIONE GRAFICA
\use{tikz}
\usetikzlibrary{matrix, patterns, calc, decorations.pathreplacing, hobby, decorations.markings, decorations.pathmorphing, babel}
\use{tikz-3dplot}
\use{mathrsfs} %per geogebra
\use{tikz-cd}
\tikzset
{
  %surface/.style={fill=black!10, shading=ball,fill opacity=0.4},
  plane/.style={black,pattern=north east lines},
  curve/.style={black,line width=0.5mm},
  dritto/.style={decoration={markings,mark=at position 0.5 with {\arrow{Stealth}}}, postaction=decorate},
  rovescio/.style={decoration={markings,mark=at position 0.5 with {\arrow{Stealth[reversed]}}}, postaction=decorate}
}
\use{pgfplots} % stampare le funzioni
	\pgfplotsset{/pgf/number format/use comma,compat=1.15}
	%\pgfplotsset{compat=1.15} %per geogebra
	\usepgfplotslibrary{fillbetween, polar}
%%%%%%

%% CITAZIONI
\use{lineno}

% % Change standard block colors

\setbeamerfont{block title}{series=\bfseries} % Grassetto nei titoli

% \setbeamertemplate{blocks}[default]   %%  PRESENTAZIONE SQUADRATA
\setbeamertemplate{blocks}[rounded][shadow=true]

% % 1- Block title (background and text)

\setbeamercolor{block title}{use={palette primary,palette secondary}, bg=palette secondary.bg, fg=palette primary.fg}

% % 2- Block body (background)
\setbeamercolor{block body}{use={palette primary}, bg=palette primary.bg}

% % Change alert block colors

% % 1- Block title (background and text)
\setbeamertemplate{blocks}[rounded][shadow=true]
\setbeamercolor{block title alerted}{use={palette secondary}, bg=palette secondary.fg, fg=white}

% % 2- Block body (background)
\setbeamercolor{block body alerted}{use={palette primary}, bg=palette primary.bg}


% {\usebeamercolor[fg]{palette secondary} Prova} % colore giusto per lo sfondo
% {\usebeamercolor[bg]{palette secondary} Due} % colore giusto per i titoli

\newenvironment{definizioneb}{\begin{block}{Definizione}}{\end{block}}
\newenvironment{osservazioneb}{\begin{block}{Osservazione}}{\end{block}}
\newenvironment{teoremab}{\begin{alertblock}{Teorema}}{\end{alertblock}}
\newenvironment{corollariob}{\begin{alertblock}{Corollario}}{\end{alertblock}}

% \setbeamertemplate{itemize items}[square] %%  PRESENTAZIONE SQUADRATA
\setbeamertemplate{itemize items}[circle]
\setbeamercolor{itemize item}{use={palette primary}, fg=palette primary.fg}
\setbeamertemplate{enumerate items}[default]
\setbeamercolor{enumerate item}{use={palette primary}, fg=palette primary.fg}

\setbeamertemplate{navigation symbols}{}%remove navigation symbols

\newcommand{\citazione}[1]{%
  \begin{quotation}
  \begin{linenumbers}
  \modulolinenumbers[5]
  \begingroup
  \setlength{\parindent}{0cm}
  \noindent #1
  \endgroup
  \end{linenumbers}
  \end{quotation}\setcounter{linenumber}{1}
  }

\bibliography{bibliography.bib}

\hypersetup{%
	pdfauthor={Davide Peccioli},
	pdfsubject={},
	allcolors=black,
	citecolor=black,
	colorlinks=true,
	bookmarksopen=true}

\hypersetup{pdfpagemode=FullScreen}
\renewcommand{\href}[2]{#2}
\usetheme{CambridgeUS}
\usecolortheme{spruce}
\usefonttheme{professionalfonts}
\author{Davide Peccioli}
\date{4 giugno 2025}
\title{Giochi di Banach-Mazur}
\begin{document}

\maketitle
\section{Teoria dei Giochi}
\label{sec:org0098109}

\begin{frame}[label={sec:orgc28c64b}]{Gioco Logico}
\begin{block}{Definizione 1.1}
Un \uline{gioco logico} è una quadrupla \(\mathcal{G} \coloneqq (\Omega, f, W_{\text{I}}, W_{\text{II}})\) dove:
\begin{itemize}
\item \(\Omega\) è un \href{../../../../../../../org/roam/20250130104331-insieme_mk.org}{insieme}, chiamato il \uline{dominio del gioco};
\item \(f:\Omega^{<\omega}\to \set{\text{I},\text{II}}\) è una \href{../../../../../../../org/roam/20250202170607-classe_relazione_binaria.org}{funzione}, chiamata \uline{funzione di turno} o \uline{funzione del giocatore};
\item \(W_{\text{I}},W_{\text{II}} \subseteq \Omega^{<\omega}\cup \Omega^{\omega}\) sono tali che
\begin{enumerate}
\item \(W_{\text{I}}\cap W_{\text{II}} = \emptyset\);
\item per ogni \(\bm{a} \in W_{\bullet}\) e per ogni \(\bm{b} \in\Omega^{<\omega}\cup \Omega^{\omega}\):
\end{enumerate}
\begin{equation*}
  \bm{a} \subseteq \bm{b}\quad\implies\quad \bm{b} \in W_{\bullet}
\end{equation*}
\end{itemize}

Gli elementi di \(\Omega^{<\omega}\) sono chiamati \uline{posizioni del gioco} \(\mathcal{G}\), mentre un elemento di \(\Omega^{\omega}\) è detto \uline{giocata} di \(\mathcal{G}\).
\end{block}
\end{frame}
\begin{frame}[label={sec:orga493e04}]{DATOGLIERE}
I giocatori I e II giocano scegliendo a turno elementi di \(\Omega\). La funzione di turno \(f\) associa a ciascuna posizione uno dei due giocatori: se
\begin{equation*}
f(a_{0},a_{1},\dots,a_{n}) = \text{I}
\end{equation*}
allora l'elemento \(a_{n+1}\) sarà scelto dal giocatore I.

Si dirà che il giocatore I \uline{vince la giocata \(\bm{a}\)} se \(\bm{a} \in W_{\text{I}}\); si dirà che il giocatore II \uline{vince la giocata \(\bm{b}\)} se \(\bm{b} \in W_{\text{II}}\).
\begin{block}{Definizione 1.2}
Un gioco è detto \uline{totale} se \(\Omega^{\omega} \subseteq W_{\text{I}}\cup W_{\text{II}}\).
\end{block}
\end{frame}
\begin{frame}[label={sec:org153fabe}]{Strategia per un gioco logico}
Una strategia per un giocatore è un insieme di regole che descrivono esattamente come un giocatore debba giocare, in base a tutte le mosse precedenti.

Una strategia è detta \uline{vincente} per un giocatore se questo vince ogni giocata in cui ne fa uso, a prescindere dalle mosse dell'altro giocatore.
\begin{block}{Definizione 1.4}
Un gioco si dice \uline{determinato} se esiste una strategia vincente per I o per II.
\end{block}
\end{frame}
\begin{frame}[label={sec:org9e4f6d2}]{Giochi logici equivalenti}
\begin{block}{Definizione 1.5}
Due \href{../../../../../../../org/roam/20250513155732-logic_game.org}{giochi logici} \(\mathcal{G}\) e \(\mathcal{G'}\) con giocatori I e II sono detti \uline{equivalenti} se sono soddisfate entrambe le seguenti ipotsi:
\begin{enumerate}
\item esiste una \href{../../../../../../../org/roam/20250513155732-logic_game.org}{strategia vincente} per I in \(\mathcal{G}\) sse esiste una \href{../../../../../../../org/roam/20250513155732-logic_game.org}{strategia vincente} per I in \(\mathcal{G}'\)
\item esiste una \href{../../../../../../../org/roam/20250513155732-logic_game.org}{strategia vincente} per II in \(\mathcal{G}\) sse esiste una \href{../../../../../../../org/roam/20250513155732-logic_game.org}{strategia vincente} per II in \(\mathcal{G}'\)
\end{enumerate}
\end{block}
\end{frame}
\begin{frame}[label={sec:org4cc428c}]{Giochi di Gale-Stewart}
\begin{block}{Definizione 1.6}
Sia \(A\) un \href{../../../../../../../org/roam/20250130104331-insieme_mk.org}{insieme} non vuoto, e sia \(C \subseteq A^{\omega}\). Si definisce il \uline{gioco di Gale-Stewart} associato ad \(C\) come il \href{../../../../../../../org/roam/20250513155732-logic_game.org}{gioco logico} seguente:
\begin{equation*}
G(A,C) = G(A) \coloneqq (A, \psi, C, A^{\omega}\setminus C)
\end{equation*}
dove la \href{../../../../../../../org/roam/20250202170607-classe_relazione_binaria.org}{funzione} \(\psi: A^{<\omega}\to \set{\text{I},\text{II}}\) è così definita
\begin{equation*}
\psi(s) \coloneqq \begin{cases}
\text{I} & \operatorname{lh}(s)\text{ è pari}\\
\text{II} & \operatorname{lh}(s)\text{ è dispari}
\end{cases}
\end{equation*}
\end{block}
\end{frame}
\begin{frame}[label={sec:orgbd541f3}]{DATOGLIERE}
Pertanto il gioco può essere codificato come segue:
\begin{equation*}
\begin{tikzcd}[ampersand replacement=\&,cramped,sep=tiny]
	{\text{I}} \& {a_0} \&\& {a_2} \&\& {a_4} \&\& \dots \\
	{\text{II}} \&\& {a_1} \&\& {a_3} \&\& \dots
\end{tikzcd}
\end{equation*}
e il giocatore I vince se e solo se \((a_{n})_{n \in \omega} \in C\).
\end{frame}
\begin{frame}[label={sec:org46289cb}]{Strategia per un gioco di Gale-Stewart}
Si specializza la definizione di \href{../../../../../../../org/roam/20250513155732-logic_game.org}{strategia per un gioco logico} al caso particolare di un gioco di Gale-Stewart.

Una strategia per un gioco \(G(A,C)\) è un \href{../../../../../../../org/roam/20250514142154-albero_teoria_descrittiva_degli_insiemi.org}{albero} \(\sigma \subseteq A^{<\omega}\) tale che:
\begin{enumerate}
\item \(\sigma\) sia \href{../../../../../../../org/roam/20250514142208-albero_potato.org}{potato} e non vuoto;

\item se \(\langle a_{0},\dots,a_{2j}\rangle \in \sigma\) allora per ogni \(a_{2j+1} \in A\): \(\langle a_{0},\dots,a_{2j+1}\rangle \in \sigma\);

\item se \(\langle a_{0},\dots,a_{2j-1}\rangle \in \sigma\) allora esiste un unico \(a_{2j} \in A\) tale che \(\langle a_{0},\dots,a_{2j}\rangle \in \sigma\).
\end{enumerate}

Una strategia è detta \uline{vincente} se il suo \href{../../../../../../../org/roam/20250514142251-corpo_di_un_albero.org}{corpo} \([\sigma] \in A\).
\end{frame}
\begin{frame}[label={sec:orgaf782d8}]{Gioco di Gale-Stewart con posizioni ammissibili}
Spesso è comodo considerare giochi in cui I e II non possano giocare ogni elemento di \(A\), ma debbano seguire delle \uline{regole}. Quindi, è necessario dare un alberto potato non vuoto \(T \subseteq A^{<\omega}\), che determina le \href{../../../../../../../org/roam/20250514142938-posizioni_ammissibili_in_un_gioco_logico.org}{\uline{posizioni ammissibili}}.

In questa situazione I e II si alternano giocando \((a_{i})_{i \in \omega}\) in maniera tale che, ad ogni passo \(n \in \omega\)
\begin{equation*}
\langle a_{0},\dots,a_{n}\rangle \in T
\end{equation*}

Si scriverà, in questo caso, \(G(T, C)\).
\end{frame}
\begin{frame}[label={sec:org2ef3855}]{Teorema di Gale-Stewart}
Sia \(A\) uno \href{../../../../../../../org/roam/20250103145124-topologia.org}{spazio topologico} \href{../../../../../../../org/roam/20250317165247-topologia_discreta.org}{discreto} e sia \(A^{\omega}\) dotato della \href{../../../../../../../org/roam/20250109154723-topologia_prodotto.org}{topologia prodotto}.
\begin{alertblock}{Teorema di Gale-Stewart 1.7}
Sia \(T\) un \href{../../../../../../../org/roam/20250514142154-albero_teoria_descrittiva_degli_insiemi.org}{albero} \href{../../../../../../../org/roam/20250514142208-albero_potato.org}{potato} non vuoto su \(A\). Se \(C \subseteq [T]\) è \href{../../../../../../../org/roam/20250103145124-topologia.org}{aperto} o \href{../../../../../../../org/roam/20250103145124-topologia.org}{chiuso} in \([T]\), allora \href{../../../../../../../org/roam/20250513171520-giochi_di_gale_stewart.org}{il gioco} \(G(T,C)\) è \href{../../../../../../../org/roam/20250513155732-logic_game.org}{determinato}.
\end{alertblock}
\end{frame}
\section{Insiemi Analitici e BP}
\label{sec:org4888705}

\begin{frame}[label={sec:org1e115ec}]{Gioco di Choquet}
\begin{block}{Definizione 2.1}
Sia \((X,\tau)\) uno \href{../../../../../../../org/roam/20250103145124-topologia.org}{spazio topologico} non vuoto. Il gioco di Choquet \(G_{X}\) è un \href{../../../../../../../org/roam/20250513155732-logic_game.org}{gioco} \href{../../../../../../../org/roam/20250513171520-giochi_di_gale_stewart.org}{di Gale-Stewart} totale codificato come segue: i giocatori I e II si alternano scegliendo sottoinsiemi aperti non vuoti di \(X\):
\begin{equation*}
\begin{tikzcd}[ampersand replacement=\&,cramped,sep=tiny]
	{\text{I}} \& {U_0} \&\& {U_1} \&\& {U_2} \&\& \cdots \\
	{\text{II}} \&\& {V_0} \&\& {V_1} \&\& \cdots
\end{tikzcd}
\end{equation*}
tali che \(U_{0} \supseteq V_{0}\supseteq U_{1}\supseteq V_{1}\supseteq \dots\)

Il giocatore II vince se
\begin{equation*}
\bigcap_{n \in \omega} V_{n} = \bigcap_{n \in \omega} U_{n} \neq \emptyset.
\end{equation*}
\end{block}
\end{frame}
\begin{frame}[label={sec:org9d69a79}]{DATOGLIERE}
\begin{alertblock}{Teorema 2.2}
Uno \href{../../../../../../../org/roam/20250103145124-topologia.org}{spazio topologico} \(X\) è uno \href{../../../../../../../org/roam/20250514154101-spazio_topologico_di_baire.org}{spazio topologico di Baire} se e solo se il giocatore I \uline{non ha una \href{../../../../../../../org/roam/20250513171520-giochi_di_gale_stewart.org}{strategia} \href{../../../../../../../org/roam/20250513171520-giochi_di_gale_stewart.org}{vincente}} nel \href{../../../../../../../org/roam/20250514174255-gioco_di_choquet.org}{gioco di Choquet} \(G_{X}\).
\end{alertblock}
\begin{block}{Definizione 2.3}
Uno spazio topologico \(X\) è detto \uline{spazio di Choquet} se il giocatore II ha una strategia vincente in \(G_{X}\).
\end{block}
\begin{block}{DATOGLIERE}
In particolare, ogni spazio Polacco è uno spazio di Choquet.
\end{block}
\end{frame}
\begin{frame}[label={sec:org838fb78}]{Gioco di Banach-Mazur}
Sia \(X\) uno \href{../../../../../../../org/roam/20250103145124-topologia.org}{spazio topologico} non vuoto, e sia \(A \subseteq X\).
\begin{block}{Definizione 2.5}
Il \uline{gioco di Banach-Mazur} (o anche **-gioco) di \(A\), denotato con \(G^{ * *}(A)\) oppure con \(G^{ * *}(A,X)\) è un \href{../../../../../../../org/roam/20250513155732-logic_game.org}{gioco} \href{../../../../../../../org/roam/20250513171520-giochi_di_gale_stewart.org}{di Gale-Stewart} codificato come segue: i giocatori I e II si alternano scegliendo sottoinsiemi aperti non vuoti di \(X\)
\begin{equation*}
\begin{tikzcd}[ampersand replacement=\&,cramped,sep=tiny]
	{\text{I}} \& {U_0} \&\& {U_1} \&\& {U_2} \&\& \cdots \\
	{\text{II}} \&\& {V_0} \&\& {V_1} \&\& \cdots
\end{tikzcd}
\end{equation*}
\href{../../../../../../../org/roam/20250513171520-giochi_di_gale_stewart.org}{tali che} \(U_{0}\supseteq V_{0}\supseteq U_{1}\supseteq V_{1}\supseteq \dots\)

Il giocatore II vince se
\begin{equation*}
\bigcap_{n \in \omega} U_{n} = \bigcap_{n \in \omega} V_{n} \subseteq A.
\end{equation*}
\end{block}
\end{frame}
\begin{frame}[label={sec:org70e3b80}]{DATOGLIERE}
\begin{alertblock}{Teorema 2.6}
Sia \(X\) uno \href{../../../../../../../org/roam/20250103145124-topologia.org}{spazio topologico} \href{../../../../../../../org/roam/20250131161811-insieme_vuoto_mk.org}{non vuoto}, e sia \(A \subseteq X\) un \href{../../../../../../../org/roam/20250131155822-operazioni_insiemistiche_tra_classi_mk.org}{sottoinsieme} qualsiasi. Allora \(A\) è \href{../../../../../../../org/roam/20250419122752-insieme_magro.org}{comagro} se e solo se il giocatore II ha una \href{../../../../../../../org/roam/20250513171520-giochi_di_gale_stewart.org}{strategia vincente} nel \href{../../../../../../../org/roam/20250513111844-gioco_di_banach_mazur.org}{gioco di Banach-Mazur} \(G^{**}(A)\).
\end{alertblock}
\begin{alertblock}{Teorema 2.7}
Se \(X\) è uno \href{../../../../../../../org/roam/20250103145124-topologia.org}{spazio topologico} \href{../../../../../../../org/roam/20250514174255-gioco_di_choquet.org}{di Choquet} non \href{../../../../../../../org/roam/20250131161811-insieme_vuoto_mk.org}{vuoto} ed esiste una \href{../../../../../../../org/roam/20250301193511-spazio_metrico.org}{distanza} \(d\) su \(X\) le cui \href{../../../../../../../org/roam/20250301193511-spazio_metrico.org}{palle aperte} sono aperti di \(X\), allora:

\(A\) è \href{../../../../../../../org/roam/20250419122752-insieme_magro.org}{magro} in un \href{../../../../../../../org/roam/20250103145124-topologia.org}{aperto} non vuoto se e solo se il giocatore I ha una \href{../../../../../../../org/roam/20250513171520-giochi_di_gale_stewart.org}{strategia vincente} nel \href{../../../../../../../org/roam/20250513111844-gioco_di_banach_mazur.org}{gioco di Banach-Mazur} \(G^{**}(A)\).
\end{alertblock}
\end{frame}
\begin{frame}[label={sec:orgc8e3bc0}]{Dimostrazione Teorema 2.7 (\(\Rightarrow\))}
Se \(A\) è magro in \(Y \subseteq X\), sia per ogni \(n \in \omega\): \(W_{n} \subseteq Y\) aperto denso di \(Y\), con
\begin{equation*}
\bigcap_{n \in\omega} W_{n} \subseteq Y \setminus A.
\end{equation*}

\href{../../../../../../../org/roam/20250514174255-gioco_di_choquet.org}{Poiché} \(Y\) è uno \href{../../../../../../../org/roam/20250514174255-gioco_di_choquet.org}{spazio di Choquet}, allora nel \href{../../../../../../../org/roam/20250513171520-giochi_di_gale_stewart.org}{gioco}:
\begin{equation*}
\begin{tikzcd}[ampersand replacement=\&,cramped, sep=tiny]
	{\text{I}} \&\& {B_1} \&\& {B_2} \&\& \dots \\
	{\text{II}} \& {A_0} \&\& {A_1} \&\& \dots
\end{tikzcd}
\end{equation*}
con gli aperti non vuoti \(Y\supseteq V_{0}\supseteq U_{1}\supseteq V_{1}\supseteq \dots\) in cui I vince sse \(\bigcap_{n \in \omega}{B_{n}} \neq \emptyset\), I ha una \href{../../../../../../../org/roam/20250513171520-giochi_di_gale_stewart.org}{strategia vincente}. Questo infatti è un gioco di Choquet a giocatori invertiti.

Sia quindi \(\sigma\) la strategia vincente di I in questo gioco di Choquet.\hfill \textit{(cont.)}
\end{frame}
\begin{frame}[label={sec:org50d7230}]{Dimostrazione Teorema 2.7 (\(\Rightarrow\)) (cont.)}
Nel gioco \(G^{**}(A)\), il giocatore I pone \(U_{0} \coloneqq Y\). Si costruisce per induzione la strategia vincente per I.

Al passo \(n+1\)-esimo, sia \((U_{0},V_{0},\dots, U_{n}, V_{n})\) la sequenza di insiemi giocati. Si pone, per ogni \(i\le n\): \(V_{i}'\coloneqq V_{i}\cap W_{i}\), e si sceglie \(U_{n+1}\) come l'unico sottoinsieme aperto non vuoto di \(V_{n}\) tale che
\begin{equation*}
(V_{0}', U_{1}, V_{1}', U_{2},\dots, V_{n}', U_{n+1}) \in\sigma.
\end{equation*}

Allora \(\bigcap_{n \in \omega} U_{n}\neq\emptyset\) e inoltre
\begin{equation*}
\bigcap_{n \in\omega} U_{n} = \bigcap_{n \in\omega} V_{n}' \subseteq \bigcap_{n \in \omega} W_{n} \subseteq Y\setminus A
\end{equation*}
e dunque \(\bigcap_{n \in\omega} U_{n} \not\subseteq A\).\qed
\end{frame}
\begin{frame}[label={sec:org7e23e96}]{Dimostrazione Teorema 2.7 (\(\Leftarrow\))}
Sia \(\sigma\) una strategia vincente per I in \(G^{**}(A)\), e sia \(U_{0}\) l'elemento di partenza per \(\sigma\).

Esiste allora una strategia \(\sigma'\) per I, vincente, e tale che l'insieme giocato al passo \(n\)-esimo \(U_{n}\) abbia diametro (in una metrica fissata):
\begin{equation*}
\operatorname{diam}(U_{n})<2^{-n}.
\end{equation*}

Allora \(\bigcap_{n \in \omega} U_{n} = \set{x}\), con \(x \in U_{0}\setminus A\).\hfill \textit{(cont.)}
\end{frame}
\begin{frame}[label={sec:orga29e24b}]{Dimostrazione Teorema 2.7 (\(\Leftarrow\)) (cont.)}
Sia quindi
\begin{equation*}
W\coloneqq \set{x \in U_{0}\mid
\exists\, (U_{i}, V_{i})_{i \in \omega} \in [\sigma']\ x \in \bigcap_{n \in \omega} U_{i}}
\end{equation*}
\begin{itemize}
\item \(W\) è denso in \(U_{0}\), poiché per ogni \(B \subseteq U_{0}\) esiste \(p = (U_{i}, V_{i})_{i \in \omega} \in [\sigma']\) tale che \(V_{0} = B\), e, siccome \(p \in [\sigma']\) allora
\begin{equation*}
  \bigcap_{n \in \omega} U_{i} = \set{x} \subseteq U_{1} \subseteq V_{0} = B
\end{equation*}
e dunque \(W\cap B\neq \emptyset\).
\item Inoltre \(W \subseteq U_{0}\setminus A\), per costruzione di \(\sigma'\).
\end{itemize}

Pertanto \(A\) è magro in \(U_{0}\).\qed
\end{frame}
\begin{frame}[label={sec:org6bb6658}]{DATOGLIERE}
\begin{alertblock}{Lemma 2.8}
Sia \(X\) uno \href{../../../../../../../org/roam/20250103145124-topologia.org}{spazio topologico} \href{../../../../../../../org/roam/20250514174255-gioco_di_choquet.org}{di Choquet} non \href{../../../../../../../org/roam/20250131161811-insieme_vuoto_mk.org}{vuoto} tale che esista una \href{../../../../../../../org/roam/20250301193511-spazio_metrico.org}{distanza} \(d\) su \(X\) le cui \href{../../../../../../../org/roam/20250301193511-spazio_metrico.org}{palle aperte} sono aperti di \(X\). Sia \(A \subseteq X\).

Se per ogni aperto \(U \subseteq X\) il \href{../../../../../../../org/roam/20250513111844-gioco_di_banach_mazur.org}{gioco} \(G^{**}\left((X\setminus A)\cup U\right)\) è \href{../../../../../../../org/roam/20250513155732-logic_game.org}{determinato} allora \(A \subseteq X\) ha \href{../../../../../../../org/roam/20250514154039-proprieta_di_baire.org}{BP}.
\end{alertblock}
\end{frame}
\begin{frame}[label={sec:org465e651}]{DATOGLIERE}
\begin{block}{Definizione 2.9}
Una \uline{base debole} per uno spazio topologico \((X,\tau)\) è una collezione di aperti \(\set{A_{\alpha}}_{\alpha \in \Omega} \subseteq \tau\) tali che, per ogni aperto non vuoto di \(X\), \(\emptyset\neq U \subseteq X\) esista \(\alpha_{0} \in \Omega\) tale che
\begin{equation*}
A_{\alpha_{0}} \subseteq U.
\end{equation*}
\end{block}
\end{frame}
\begin{frame}[label={sec:org0d0e532}]{Gioco di Banach-Mazur unfolded}
Sia \(X\) uno \href{../../../../../../../org/roam/20250301194013-spazio_polacco.org}{spazio polacco} non vuoto con una \href{../../../../../../../org/roam/20250301193511-spazio_metrico.org}{metrica} fissata e sia \(\mathcal{W}\) una \href{../../../../../../../org/roam/20250525113346-base_debole_di_uno_spazio_topologico.org}{base debole} \href{../../../../../../../org/roam/20250111143651-insieme_numerabile.org}{numerabile} di \(X\).
\begin{block}{Definizione 2.10}
Dato \(F \subseteq X\times \omega^{\omega}\), il \uline{gioco di Banach-Mazur unfolded} \(G^{**}_{\text{u}}(F)\) è il \href{../../../../../../../org/roam/20250513155732-logic_game.org}{gioco} \href{../../../../../../../org/roam/20250513171520-giochi_di_gale_stewart.org}{di Gale-Stewart} codificato come segue:
\begin{equation*}
\begin{tikzcd}[ampersand replacement=\&,cramped,sep=tiny]
	{\text{I}} \& {U_0} \&\& {U_1} \&\& \dots \\
	{\text{II}} \&\& {y_0,V_0} \&\& {y_1, V_{1}} \& \dots
\end{tikzcd}
\end{equation*}
tali che:
\begin{itemize}
\item per ogni \(i \in \omega\): \(U_{i}, V_{i} \in \mathcal{W}\), \(y_{n} \in \omega\);
\item \(\operatorname{diam}(U_{n}), \operatorname{diam}(V_{n}) < 2^{-n}\);
\item \(U_{0}\supseteq V_{0}\supseteq U_{1}\supseteq V_{1}\supseteq \dots\)\hfill \textit{(cont.)}
\end{itemize}
\end{block}
\end{frame}
\begin{frame}[label={sec:org27e3d89}]{DATOGLIERE}
\begin{block}{Definizione 2.10 (cont.)}
Posto
\begin{equation*}
\set{x}\coloneqq\bigcap_{i \in \omega} \operatorname{Cl}_{X}(U_{n}) = \bigcap_{i \in \omega} \operatorname{Cl}_{X}(V_{n})
\end{equation*}
e \(y\coloneqq (y_{i})_{i \in \omega} \in \omega^{\omega}\), il \uline{giocatore II vince} sse
\begin{equation*}
(x,y) \in F \subseteq X\times \omega^{\omega}.
\end{equation*}
\end{block}
\begin{alertblock}{Lemma 2.11}
Se \(F\) è aperto o chiuso di \(X\times\omega^{\omega}\), allora \(G^{**}_{\text{u}}(F)\) è determinato.
\end{alertblock}
\end{frame}
\begin{frame}[label={sec:orgfeb2e5d}]{DATOGLIERE}
\begin{alertblock}{Teorema 2.12}
Sia \(X\) uno spazio polacco con una metrica fissata e sia \(\mathcal{W}\) una base debole di X.

Dato \(F \subseteq X\times \omega^{\omega}\) si consideri il \href{../../../../../../../org/roam/20250513111844-gioco_di_banach_mazur.org}{**-gioco}: \(G^{**}_{\text{u}}(F)\). Indicato con \(A\coloneqq \pi_{X}(F)\):
\begin{enumerate}
\item se I ha una \href{../../../../../../../org/roam/20250513171520-giochi_di_gale_stewart.org}{strategia} \href{../../../../../../../org/roam/20250513171520-giochi_di_gale_stewart.org}{vincente} in \(G^{**}_{\text{u}}(F)\), allora \(A\) è magro in un aperto non vuoto di \(X\);
\item se II ha una \href{../../../../../../../org/roam/20250513171520-giochi_di_gale_stewart.org}{strategia} \href{../../../../../../../org/roam/20250513171520-giochi_di_gale_stewart.org}{vincente} in \(G^{**}_{\text{u}}(F)\) allora \(A\) è comagro.
\end{enumerate}
\end{alertblock}
\end{frame}
\begin{frame}[label={sec:orgcde9e28}]{Dimostrazione Teorema 2.12(1)}
Sia \(\sigma\) una \href{../../../../../../../org/roam/20250513171520-giochi_di_gale_stewart.org}{strategia} \href{../../../../../../../org/roam/20250513171520-giochi_di_gale_stewart.org}{vincente} per I, e sia \(U_{0}\) la prima mossa. Si mostra che \(A\) è \href{../../../../../../../org/roam/20250419122752-insieme_magro.org}{magro} in \(U_{0}\).

Per ogni \(a \in \omega\) e per ogni \(p \in \sigma\) della forma:
\begin{equation*}
 p=\langle
 	U_{0},({y}_{0}, V_{0}), \dots, U_{n-1}, ({y}_{n-1}, V_{n-1}), U_{n}
 \rangle
\end{equation*}
si definisce \(F_{p,a} \subseteq U_{0}\):
\begin{align*}
 F_{p,a} = \{&z \in U_{n}\mid \text{per ogni mossa legale }(a, V_{n})\\
 &\text{se } U_{n+1}\text{ è l'unico elemento di }\mathcal{W}\text{ tale che}\\
 &p \concat \langle(a,V_{n}), U_{n+1}\rangle \in \sigma \text{ allora } z \notin U_{n+1}\}
\end{align*}
L'insieme \(F_{p,a}\) è mai denso, poiché chiuso e con interno vuoto.\hfill \textit{(cont.)}
\end{frame}
\begin{frame}[label={sec:orgde780d6}]{Dimostrazione 2.12(1) (cont.)}
Sia ora \(x \in A\cap U_{0}\). Allora esiste \(y \in \omega^{\omega}\), \(y=(y_{i})_{i \in\omega}\) tale che \((x,y) \in F\).

Una posizione \(p \in \sigma\):
\begin{equation*}
     p=\langle
     	U_{0},({y}_{0}, V_{0}), \dots, U_{n-1}, ({y}_{n-1}, V_{n-1}), U_{n}
     \rangle
\end{equation*}
è \uline{buona} per \((x,{y})\) se \(x \in U_{n}\). Siccome \(\sigma\) è una strategia vincente per il giocatore I, allora esiste una posizione \(p_{(x,y)} \in \sigma\) buona per \((x,y)\) e massimale, ovvero ogni estensione di \(p_{(x,y)}\) \uline{non è buona}. Ma allora, se
\begin{equation*}
 p_{(x,y)} = \langle U_{0}, (y_{0},V_{0}),\dots, U_{n}\rangle
\end{equation*}
si ha che \(x \in F_{p_{(x,y)}, y_{n}}\).

Pertanto \(A\cap U_{0} \subseteq \bigcup_{p \in \sigma', a \in \omega} F_{p,a}\) è magro.\qed
\end{frame}
\begin{frame}[label={sec:org1ea7ae6}]{Teorema di Lusin-Sierpiński}
\begin{alertblock}{Teorema di Lusin-Sierpiński 2.13}
Sia \(X\) uno \href{../../../../../../../org/roam/20250301194013-spazio_polacco.org}{spazio polacco}. Allora ogni \href{../../../../../../../org/roam/20250525220742-insieme_analitico.org}{insieme analitico} di \(X\) ha la \href{../../../../../../../org/roam/20250514154039-proprieta_di_baire.org}{Baire Property}.
\end{alertblock}
\begin{proof}[Dimostrazione]
Siccome \(\mathrm{BP}(X)\) è una \href{../../../../../../../org/roam/20250526100313-sigma_algebra.org}{\(\sigma\)-algebra} allora è chiusa per complementi, e pertanto se ogni insieme coanalitico ha BP allora si è dimostrata la tesi.

Sia dunque \(C\) un insieme coanalitico e sia \(U \subseteq X\) un aperto. Posto \(A\coloneqq (X\setminus C)\cup U\), questo è un insieme analitico, e pertanto  esiste un chiuso \(F \subseteq X\times\omega^{\omega}\) tale che \(A=\pi_{X}(F)\).

Per il \href{../../../../../../../org/roam/20250514144736-teorema_di_gale_stewart.org}{Teorema di Gale-Stewart} 1.7 (e per il Lemma 2.11), allora, il \href{../../../../../../../org/roam/20250513111844-gioco_di_banach_mazur.org}{**-gioco \(G^{ * *}_{\text{u}}(F)\)} è \href{../../../../../../../org/roam/20250513155732-logic_game.org}{determinato}, ed in particolare vale una tra le condizioni 1. e 2. del Teorema 2.12.

Per i \href{../../../../../../../org/roam/20250514174717-teorema_di_caratterizzazione_dei_comagri_tramite_il_gioco_di_banach_mazur.org}{Teoremi} 2.26 e 2.27, allora, il \href{../../../../../../../org/roam/20250513111844-gioco_di_banach_mazur.org}{gioco \(G^{**}(A) = G^{ * *}\left((X\setminus C) \cup U\right)\)} è determinato: per il Lemma 2.8, quindi \(C\) ha la BP.
\end{proof}
\end{frame}
\section{\null}
\label{sec:orgb9588db}

\begin{frame}[label={sec:orgdd9b26a}]{Bibliografia minimale}
\nocite{*}
\printbibliography
\end{frame}
\end{document}
