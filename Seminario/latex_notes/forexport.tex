% Created 2025-05-29 Thu 23:14
% Intended LaTeX compiler: pdflatex
\documentclass{article}
\newcommand{\use}[2][]{\usepackage[#1]{#2}}
% PACCHETTI FONDAMENTLAI
\use[utf8]{inputenc}
\use[T1]{fontenc}
\use{graphicx}
\use{longtable}
\use{wrapfig}
\use{rotating}
\use[normalem]{ulem}
\use{amsmath}
\use{amsthm}
\use{amssymb}
\use{capt-of}
\use[italian]{babel}
\use[babel]{csquotes}
\use[style=numeric, hyperref]{biblatex}
\use{microtype}
\use{lmodern}
\use{subfig} % sottofigure
\use{multicol} % due colonne
\use{lipsum} % lorem ipsum
\use{color} % colori in latex
\use{parskip} % rimuove l'indentazione dei nuovi paragrafi %% Add parbox=false to all new tcolorbox
\use{centernot}
\use[outline]{contour}\contourlength{3pt}
\use{fancyhdr}
\use{layout}
\use[most]{tcolorbox} % Riquadri colorati
\use{ifthen} % IFTHEN
\use{geometry}

% pacchetti matematica
\use{yhmath}
\use{dsfont}
\use{mathrsfs}
\use{cancel} % semplificare
\use{polynom} %divisione tra polinomi
\use{forest} % grafi ad albero
\use{booktabs} % tabelle
\use{commath} %simboli e differenziali
\use{bm} %bold
\use[fulladjust]{marginnote} %to use marginnote for date notes
\use{arrayjobx}%array
\use[intlimits]{empheq} % Riquadri colorati attorno alle equazioni
\use{mathtools}
\use{circuitikz} % Disegnare i circuiti
%%%%%%%%%%%%%


%%%% QUIVER
\newcommand{\duepunti}{\,\mathchar\numexpr"6000+`:\relax\,}
% A TikZ style for curved arrows of a fixed height, due to AndréC.
\tikzset{curve/.style={settings={#1},to path={(\tikztostart)
    .. controls ($(\tikztostart)!\pv{pos}!(\tikztotarget)!\pv{height}!270:(\tikztotarget)$)
    and ($(\tikztostart)!1-\pv{pos}!(\tikztotarget)!\pv{height}!270:(\tikztotarget)$)
    .. (\tikztotarget)\tikztonodes}},
    settings/.code={\tikzset{quiver/.cd,#1}
        \def\pv##1{\pgfkeysvalueof{/tikz/quiver/##1}}},
    quiver/.cd,pos/.initial=0.35,height/.initial=0}

% TikZ arrowhead/tail styles.
\tikzset{tail reversed/.code={\pgfsetarrowsstart{tikzcd to}}}
\tikzset{2tail/.code={\pgfsetarrowsstart{Implies[reversed]}}}
\tikzset{2tail reversed/.code={\pgfsetarrowsstart{Implies}}}
% TikZ arrow styles.
\tikzset{no body/.style={/tikz/dash pattern=on 0 off 1mm}}
%%%%%%%%%%


%% DEFINIZIONI COMANDI MATEMATICI
\let\sin\relax %TOGLIE LA DEFINIZIONE SU "\sin"

% cambia la definizione di empty set
% ---
\let\oldemptyset\emptyset
% ---
% \let\emptyset\varnothing
% ---
% \let\emptyset\relax
% \newcommand{\emptyset}{\text{\textnormal{\O}}}
% ---

\DeclareMathOperator{\bounded}{bd}
\DeclareMathOperator{\sin}{sen}
\DeclareMathOperator{\epi}{Epi}
\DeclareMathOperator{\cl}{cl}
\DeclareMathOperator{\graph}{graph}
\DeclareMathOperator{\arcsec}{arcsec}
\DeclareMathOperator{\arccot}{arccot}
\DeclareMathOperator{\arccsc}{arccsc}
\DeclareMathOperator{\spettro}{Spettro}
\DeclareMathOperator{\nulls}{nullspace}
\DeclareMathOperator{\dom}{dom}
\DeclareMathOperator{\ar}{ar}
\DeclareMathOperator{\const}{Const}
\DeclareMathOperator{\fun}{Fun}
\DeclareMathOperator{\rel}{Rel}
\DeclareMathOperator{\altezza}{ht}
\let\det\relax %TOGLIE LA DEFINIZIONE SU "\det"
\DeclareMathOperator{\det}{det}
\DeclareMathOperator{\End}{End}
\DeclareMathOperator{\gl}{GL}
\DeclareMathOperator{\Id}{Id}
\DeclareMathOperator{\id}{Id}
\DeclareMathOperator{\I}{\mathds{1}}
\DeclareMathOperator{\II}{II}
\DeclareMathOperator{\rank}{rank}
\DeclareMathOperator{\tr}{tr}
\DeclareMathOperator{\tc}{t.c.}
\DeclareMathOperator{\T}{T}
\DeclareMathOperator{\var}{Var}
\DeclareMathOperator{\cov}{Cov}
\DeclareMathOperator{\st}{st}
\DeclareMathOperator{\mon}{Mon}
\newcommand{\card}[1]{\left\vert #1 \right\vert}
\newcommand{\trasposta}[1]{\prescript{\text{T}}{}{#1}}
\newcommand{\1}{\mathds{1}}
\newcommand{\R}{\mathds{R}}
\newcommand{\diesis}{\#}
\newcommand{\bemolle}{\flat}
\newcommand{\starR}{\nonstandard{\R}}
\newcommand{\borel}{\mathscr{B}}
\newcommand{\lebesgue}[1]{\mathscr{L}\left(#1\right)}
\newcommand{\media}{\mathds{E}}
\newcommand{\K}{\mathds{K}}
\newcommand{\A}{\mathds{A}}
\newcommand{\Q}{\mathds{Q}}
\newcommand{\N}{\mathds{N}}
\newcommand{\C}{\mathds{C}}
\newcommand{\Z}{\mathds{Z}}
\newcommand{\qo}{\hspace{1em}\text{q.o.}\,}
\renewcommand{\tilde}[1]{\widetilde{#1}}
\renewcommand{\parallel}{\mathrel{/\mkern-5mu/}}
\newcommand{\parti}[2][]{\wp_{#1}(#2)}
\newcommand{\diff}[1]{\operatorname{d}_{#1}}
\let\oldvec\vec
\renewcommand{\vec}[1]{\overrightarrow{\vphantom{i}#1}}
\newcommand{\floor}[1]{\left\lfloor #1 \right\rfloor}
\newcommand{\cat}[1]{\mathbf{#1}}
\newcommand{\dfreccia}[1]{\xrightarrow{\ #1 \ }}
\newcommand{\sfreccia}[1]{\xleftarrow{\ #1 \ }}
\newcommand{\formalsum}[2]{{\sum_{#1}^{#2}}{\vphantom{\sum}}'}
\newcommand{\minim}[2]{\mu_{#1}\, \left(#2\right)}
\newcommand{\concat}{\null^{\frown}} % concatenazione di stringe

%% Definizione di \dotminus

\makeatletter
\newcommand{\dotminus}{\mathbin{\text{\@dotminus}}}

\newcommand{\@dotminus}{%
  \ooalign{\hidewidth\raise1ex\hbox{.}\hidewidth\cr$\m@th-$\cr}%
}
\makeatother

%tramite i prossimi due comandi posso decidere come scrivere i logaritmi naturali in tutti i documenti: ho infatti eliminato qualsiasi differenza tra "ln" e "log": se si vuole qualcosa di diverso bisogna inserire manualmente il tutto
\let\ln\relax
\DeclareMathOperator{\ln}{ln}
\let\log\relax
\DeclareMathOperator{\log}{log}
%%%%%%

%% NUOVI COMANDI
\newcommand{\straniero}[1]{\textit{#1}} %parole straniere
\newcommand{\titolo}[1]{\textsc{#1}} %titoli
\newcommand{\qedd}{\tag*{$\blacksquare$}} %qed per ambienti matemastici
\renewcommand{\qedsymbol}{$\blacksquare$} %modifica colore qed
\newcommand{\ooverline}[1]{\overline{\overline{#1}}}
\newcommand{\circoletto}[1]{\left(#1\right)^{\text{o}}}
%
\newcommand{\qmatrice}[1]{\begin{pmatrix}
#1_{11} & \cdots & #1_{1n}\\
\vdots & \ddots & \vdots \\
#1_{m1} & \cdots & #1_{mn}
\end{pmatrix}}
%
\newcommand{\parentesi}[2]{%
\underset{#1}{\underbrace{#2}}%
}
%
\newcommand{\norma}[1]{% Norma
\left\lVert#1\right\rVert%
}
\newcommand{\scalare}[2]{% Scalare
\left\langle #1, #2\right\rangle
}
%%%%%

%%%% Change footnote appearance
%%%%

\makeatletter
% ---- marker nel TESTO: (nota 1)
\renewcommand\@makefnmark{%
  \hbox{\normalfont\footnotesize(nota~\@thefnmark)}%
}

% ---- layout e marker diverso in PIÉ DI PAGINA: (1)
\renewcommand\@makefntext[1]{%
  \parindent 1em
  \noindent
  % qui non chiamo \@makefnmark, ma uso direttamente (\@thefnmark)
  \hb@xt@1.8em{\hss\normalfont\footnotesize(\@thefnmark)} %
  #1%
}
\makeatother
%%%%
%%%%

%% RESTRIZIONI
\newcommand{\referenze}[2]{
	\phantomsection{}#2\textsuperscript{\textcolor{blue}{\textbf{#1}}}
}

\let\restriction\relax

\def\restriction#1#2{\mathchoice
              {\setbox1\hbox{${\displaystyle #1}_{\scriptstyle #2}$}
              \restrictionaux{#1}{#2}}
              {\setbox1\hbox{${\textstyle #1}_{\scriptstyle #2}$}
              \restrictionaux{#1}{#2}}
              {\setbox1\hbox{${\scriptstyle #1}_{\scriptscriptstyle #2}$}
              \restrictionaux{#1}{#2}}
              {\setbox1\hbox{${\scriptscriptstyle #1}_{\scriptscriptstyle #2}$}
              \restrictionaux{#1}{#2}}}
\def\restrictionaux#1#2{{#1\,\smash{\vrule height .8\ht1 depth .85\dp1}}_{\,#2}}
%%%%%%%%%%%

%% SEZIONE GRAFICA
\use{tikz}
\usetikzlibrary{matrix, patterns, calc, decorations.pathreplacing, hobby, decorations.markings, decorations.pathmorphing, babel}
\use{tikz-3dplot}
\use{mathrsfs} %per geogebra
\use{tikz-cd}
\tikzset
{
  %surface/.style={fill=black!10, shading=ball,fill opacity=0.4},
  plane/.style={black,pattern=north east lines},
  curve/.style={black,line width=0.5mm},
  dritto/.style={decoration={markings,mark=at position 0.5 with {\arrow{Stealth}}}, postaction=decorate},
  rovescio/.style={decoration={markings,mark=at position 0.5 with {\arrow{Stealth[reversed]}}}, postaction=decorate}
}
\use{pgfplots} % stampare le funzioni
	\pgfplotsset{/pgf/number format/use comma,compat=1.15}
	%\pgfplotsset{compat=1.15} %per geogebra
	\usepgfplotslibrary{fillbetween, polar}
%%%%%%

%% CITAZIONI
\use{lineno}

\newcommand{\citazione}[1]{%
  \begin{quotation}
  \begin{linenumbers}
  \modulolinenumbers[5]
  \begingroup
  \setlength{\parindent}{0cm}
  \noindent #1
  \endgroup
  \end{linenumbers}
  \end{quotation}\setcounter{linenumber}{1}
  }
%%%%%%
%%%%%%
%%%%%%
%%%%%%
%%%%%%
%%%%%% AMSTHM
%%%%%%
%%%%%%
%%%%%%


%%%%%% MISSING REFERENCES
%%%%%%
\newcommand{\missingreference}[#1]{\textcolor{red}{#1}}


\use{hyperref}
\hypersetup{%
	pdfauthor={Davide Peccioli},
	pdfsubject={},
	allcolors=black,
	citecolor=black,
	colorlinks=true,
	bookmarksopen=true}
\pagestyle{empty}
\author{Davide Peccioli}
\date{4 giugno 2025}
\title{Giochi di Banach-Mazur}
\begin{document}

\maketitle
\tableofcontents

\section{Giochi Logici}
\label{sec:org7949131}

:ID:       90b2021f-dbe8-499f-8c79-32379384fdb8
:ROAM\textsubscript{ALIASES}: ``Gioco Logico'' ``Giocata di un gioco logico'' ``Posizione di un gioco logico''
\subsection{Definizione}
\label{sec:org0010399}
Un \uline{gioco logico} è una quadrupla \(\mathcal{G} \coloneqq (\Omega, f, W_{\text{I}}, W_{\text{II}})\) dove:
\begin{itemize}
\item \(\Omega\) è un \href{../../../../../../../org/roam/20250130104331-insieme_mk.org}{insieme}, chiamato il \uline{dominio del gioco};
\item \(f:\Omega^{<\omega}\to \set{\text{I},\text{II}}\) è una \href{../../../../../../../org/roam/20250202170607-classe_relazione_binaria.org}{funzione}, chiamata \uline{funzione di turno} o \uline{funzione del giocatore};
\item \(W_{\text{I}},W_{\text{II}} \subseteq \Omega^{<\omega}\cup \Omega^{\omega}\) \footnote{\href{../../../../../../../org/roam/20250202192030-classe_delle_classi_funzioni.org}{Classe delle Classi-Funzioni} e \href{../../../../../../../org/roam/20250203161110-numeri_naturali_sono_ordinali.org}{Ordinale omega}
:ID:       28ca09ab-bb39-4802-842e-bebecf0d2a4f} sono tali che
\begin{enumerate}
\item \(W_{\text{I}}\cap W_{\text{II}} = \emptyset\);
\item per ogni \(\bm{a} \in W_{\bullet}\) e per ogni \(\bm{b} \in\Omega^{<\omega}\cup \Omega^{\omega}\):
\end{enumerate}
\begin{equation*}
  \bm{a} \subseteq \bm{b}\quad\implies\quad \bm{b} \in W_{\bullet}
\end{equation*}
\end{itemize}

Gli elementi di \(\Omega^{<\omega}\) sono chiamati \uline{posizioni del gioco} \(\mathcal{G}\), mentre un elemento di \(\Omega^{\omega}\) è detto \uline{giocata} di \(\mathcal{G}\).

I giocatori I e II giocano scegliendo a turno elementi di \(\Omega\). La funzione di turno \(f\) associa a ciascuna posizione uno dei due giocatori: se
\begin{equation*}
f(a_{0},a_{1},\dots,a_{n}) = \text{I}
\end{equation*}
allora l'elemento \(a_{n+1}\) sarà scelto dal giocatore I.

Si dirà che il giocatore I \uline{vince la giocata \(\bm{a}\)} se \(\bm{a} \in W_{\text{I}}\); si dirà che il giocatore II \uline{vince la giocata \(\bm{b}\)} se \(\bm{b} \in W_{\text{II}}\).
\subsubsection{Gioco Logico totale}
\label{sec:org8c60113}
Un gioco è detto \uline{totale} se \(\Omega^{\omega} \subseteq W_{\text{I}}\cup W_{\text{II}}\).
\subsubsection{Strategia per un gioco logico}
\label{sec:org7f68e17}
Si definiscano i seguenti insiemi:
\begin{align*}
\Omega^{<\omega}_{\text{I}} &\coloneqq \set{s \in \Omega^{<\omega}\mid f(s) = \text{I}}\\
\Omega^{<\omega}_{\text{II}} &\coloneqq \set{s \in \Omega^{<\omega}\mid f(s) = \text{II}}
\end{align*}

A strategy for a player is a set of rules that describe exactly how that player should choose, depending on how the two players have chosen at earlier moves.

Formalmente, una strategia per il giocatore \(j\) (con \(j=\text{I},\text{II}\)) è una funzione
\begin{equation*}
\varphi: \Omega_{j}^{<\omega} \to \Omega
\end{equation*}

A strategy for a player is said to be \uline{winning} if that player wins every play in which he or she uses the strategy, regardless of what the other player does.

Un gioco si dice \uline{determinato} se esiste una strategia vincente per I o per II.
\subsection{Bibliography}
\label{sec:org749d900}
\begin{itemize}
\item Hodges, Wilfrid and Jouko Väänänen, ``Logic and Games'', The Stanford Encyclopedia of Philosophy (Winter 2024 Edition), Edward N. Zalta \& Uri Nodelman (eds.), URL = \url{https://plato.stanford.edu/archives/win2024/entries/logic-games/}.
\end{itemize}
\subsection{Definzione}
\label{sec:orgb665389}
Due \href{../../../../../../../org/roam/20250513155732-logic_game.org}{giochi logici} \(\mathcal{G}\) e \(\mathcal{G'}\) con giocatori I e II sono detti \uline{equivalenti} se sono soddisfate entrambe le seguenti ipotsi:
\begin{enumerate}
\item esiste una \hyperref[sec:org7f68e17]{strategia vincente} per I in \(\mathcal{G}\) sse esiste una \hyperref[sec:org7f68e17]{strategia vincente} per I in \(\mathcal{G}'\)
\item esiste una \hyperref[sec:org7f68e17]{strategia vincente} per II in \(\mathcal{G}\) sse esiste una \hyperref[sec:org7f68e17]{strategia vincente} per II in \(\mathcal{G}'\)
\end{enumerate}
\subsection{Giochi di Gale-Stewart}
\label{sec:org382d9d1}

Sia \(A\) un \href{../../../../../../../org/roam/20250130104331-insieme_mk.org}{insieme} non vuoto, e sia \(C \subseteq A^{\omega}\). \footnote{\href{../../../../../../../org/roam/20250202192030-classe_delle_classi_funzioni.org}{Classe delle Classi-Funzioni} e \href{../../../../../../../org/roam/20250203161110-numeri_naturali_sono_ordinali.org}{Ordinale omega}}
\subsubsection{Giochi di Gale-Stewart}
\label{sec:org21a45cb}
Si definisce il \uline{gioco di Gale-Stewart} associato ad \(C\) come il \href{../../../../../../../org/roam/20250513155732-logic_game.org}{gioco logico} seguente:
\begin{equation*}
G(A,C) = G(A) \coloneqq (A, \psi, C, A^{\omega}\setminus C)
\end{equation*}
dove la \href{../../../../../../../org/roam/20250202170607-classe_relazione_binaria.org}{funzione} \(\psi: A^{<\omega}\to \set{\text{I},\text{II}}\) è così definita
\begin{equation*}
\psi(s) \coloneqq \begin{cases}
\text{I} & \operatorname{lh}(s)\text{ è pari}\\
\text{II} & \operatorname{lh}(s)\text{ è dispari}
\end{cases}
\end{equation*}
Pertanto il gioco può essere codificato come segue:
\begin{equation*}
\begin{tikzcd}[ampersand replacement=\&,cramped,sep=tiny]
	{\text{I}} \& {a_0} \&\& {a_2} \&\& {a_4} \&\& \dots \\
	{\text{II}} \&\& {a_1} \&\& {a_3} \&\& \dots
\end{tikzcd}
\end{equation*}
e il giocatore I vince se e solo se \((a_{n})_{n \in \omega} \in C\). \footnote{\href{../../../../../../../org/roam/20250115100904-successione.org}{Successione}}
\subsubsection{Strategia per un gioco di Gale-Stewart}
\label{sec:org1db90c5}
Si specializza la definizione di \hyperref[sec:org7f68e17]{strategia per un gioco logico} al caso particolare di un gioco di Gale-Stewart.

È possibile vedere una strategia \(\varphi\) per il giocatore I in tre modi diversi, del tutto equivalenti.
\begin{enumerate}
\item Una mappa \(\psi: A^{<\omega}\to A^{<\omega}\) tale che, per ogni \(s \in A^{<\omega}\) valga che \href{../../../../../../../org/roam/20250206170922-sequenze_e_stringhe.org}{lunghezza} sia
\begin{equation*}
 \operatorname{lh}\varphi(s) = \operatorname{lh}(s) + 1
\end{equation*}
Intuitivamente, questa funzione associa alla sequenza degli \((a_{2i+1})\) giocati dal giocatore II una sequenza degli \((a_{2i})\) per il giocatore I:
\begin{equation*}
 \varphi(\emptyset) = \langle a_{0}\rangle,\quad \varphi(\langle a_{1}\rangle) = \langle a_{0},a_{2} \rangle,\quad \varphi(\langle a_{1},a_{3}\rangle) = \langle a_{0},a_{2},a_{4}\rangle.
\end{equation*}
\item Una mappa \(\psi: A^{<\omega}\to A\).

Intuitivamente, questa funzione associa alla sequenza degli \((a_{2i+1})\) giocati dal giocatore II l'elemento \(a_{j} \in A\) che deve giocare il giocatore I:
\begin{equation*}
 \varphi(\emptyset) = a_{0},\quad \varphi(\langle a_{1}\rangle) = a_{2},\quad \varphi(\langle a_{1},a_{3}\rangle) = a_{4}.
\end{equation*}
\item Una \href{../../../../../../../org/roam/20250514142154-albero_teoria_descrittiva_degli_insiemi.org}{albero} \(\sigma \subseteq A^{<\omega}\) tale che:
\begin{enumerate}
\item \(\sigma\) sia \href{../../../../../../../org/roam/20250514142208-albero_potato.org}{potato} e non vuoto;

\item se \(\langle a_{0},\dots,a_{2j}\rangle \in \sigma\) allora per ogni \(a_{2j+1} \in A\): \(\langle a_{0},\dots,a_{2j+1}\rangle \in \sigma\);

\item se \(\langle a_{0},\dots,a_{2j-1}\rangle \in \sigma\) allora esiste un unico \(a_{2j} \in A\) tale che \(\langle a_{0},\dots,a_{2j}\rangle \in \sigma\).
\end{enumerate}
\end{enumerate}

Una strategia è detta \uline{vincente} se il suo \href{../../../../../../../org/roam/20250514142251-corpo_di_un_albero.org}{corpo} \([\sigma] \in A\).
\subsubsection{Gioco di Gale-Stewart con posizioni ammissibili}
\label{sec:org32c74c6}
Spesso è comodo considerare giochi in cui I e II non possano giocare ogni elemento di \(A\), ma debbano seguire delle \uline{regole}. Quindi, è necessario dare un alberto potato non vuoto \(T \subseteq A^{<\omega}\), che determina le \href{../../../../../../../org/roam/20250514142938-posizioni_ammissibili_in_un_gioco_logico.org}{\uline{posizioni ammissibili}}.

In questa situazione I e II si alternano giocando \(\langle a_{0},\dots,a_{n},\dots,\rangle\) in maniera tale che, ad ogni passo \(n \in \omega\)
\begin{equation*}
\langle a_{0},\dots,a_{n}\rangle \in T
\end{equation*}

Si scriverà, in questo caso, \(G(T, C)\).

Si noti che questo non modifica il formalismo, in quanto è sufficiente cambiare gli insiemi di vittoria \(C\) e \(A^{\omega}\setminus C\) in maniera da far perdere automaticamente il giocatore che effettua una mossa illegale.

Inoltre, il gioco definito sopra \(G(T,C)\) è \href{../../../../../../../org/roam/20250514143441-giochi_logici_equivalenti.org}{equivalente} al gioco \(G(A, C')\), dove
\begin{equation*}
C' \coloneqq \set{x \in A^{\omega}\mid \left[\exists\, n (x\upharpoonright n \notin T) \,\land\, \text{il minore }n\text{ tale che }x\upharpoonright n \notin T\text{ è pari}\right] \,\lor\, (x \in [T] \,\land\, x \in C)}.
\end{equation*}
\subsection{Teorema di Gale-Stewart}
\label{sec:orgf56b48c}

:ID:       3c4f7c3f-7f63-4d59-9240-0c60d79e42ad
Sia \(A\) uno \href{../../../../../../../org/roam/20250103145124-topologia.org}{spazio topologico} \href{../../../../../../../org/roam/20250317165247-topologia_discreta.org}{discreto} e sia \(A^{\omega}\) dotato della \href{../../../../../../../org/roam/20250109154723-topologia_prodotto.org}{topologia prodotto}.
\subsubsection{Teorema}
\label{sec:orgc60abbb}
Sia \(T\) un \href{../../../../../../../org/roam/20250514142154-albero_teoria_descrittiva_degli_insiemi.org}{albero} \href{../../../../../../../org/roam/20250514142208-albero_potato.org}{potato} non vuoto su \(A\). Se \(C \subseteq [T]\) è \href{../../../../../../../org/roam/20250103145124-topologia.org}{aperto} o \href{../../../../../../../org/roam/20250103145124-topologia.org}{chiuso} in \([T]\) \footnote{Vedi \href{../../../../../../../org/roam/20250514142251-corpo_di_un_albero.org}{Corpo di un albero} e \href{../../../../../../../org/roam/20250103163814-sottospazio_topologico.org}{Sottospazio topologico}}, allora \hyperref[sec:org32c74c6]{il gioco} \(G(T,C)\) è \hyperref[sec:org7f68e17]{determinato}.
\section{Ripasso TDI}
\label{sec:orgffd436a}

:ID:       41953408-de97-4240-bed0-37f9de8706c4
:ROAM\textsubscript{ALIASES}: ``Insieme non magro'' ``Insieme comagro''
Sia \(X\) uno \href{../../../../../../../org/roam/20250103145124-topologia.org}{spazio topologico}
\subsection{Definizione}
\label{sec:org6117d9d}

\begin{itemize}
\item Un \href{../../../../../../../org/roam/20250131155822-operazioni_insiemistiche_tra_classi_mk.org}{sottoinsieme} \(A \subseteq X\) è \uline{magro} se può essere scritto come \href{../../../../../../../org/roam/20250131183016-classe_unione.org}{unione} \href{../../../../../../../org/roam/20250111143651-insieme_numerabile.org}{numerabile} di insiemi \href{../../../../../../../org/roam/20250417180515-insieme_mai_denso.org}{mai densi}.
\item Un sottoinsieme \(A \subseteq X\) è \uline{comagro} se il suo \href{../../../../../../../org/roam/20250317100425-complementare_di_un_insieme.org}{complementare} è magro.

Equivalentemente, \(A\) è \uline{comagro} se e solo se contiene l'\href{../../../../../../../org/roam/20250131183141-classe_intersezione.org}{intersezione} di una quantità numerabile di aperti \href{../../../../../../../org/roam/20250301193045-sottoinsieme_denso.org}{densi}.
\end{itemize}
\subsection{Definizione}
\label{sec:org5fa9f27}

Sia \(U \subseteq X\) un aperto.
\begin{itemize}
\item Un sottoinsieme \(A \subseteq X\) si dice \uline{magro in \(U\)} se \(A\cap U\) è magro;
\item Un sottoinsieme \(A \subseteq X\) è \uline{comagro in \(U\)} se \(U\setminus A\) è magro, ovvero se esiste una sequenza di aperti \href{../../../../../../../org/roam/20250301193045-sottoinsieme_denso.org}{densi} in \(U\) la cui intersezione è contenuta in \(A\).
\end{itemize}

Si noti che queste definizioni sono equivalenti a richiedere che \(U\cap A\) sia, rispettivamente, magro e comagro nella topologia di sottospazio di \(U\).
:ID:       6232eb32-a530-4c1b-aefc-403acfd7c057
\subsection{Definizione}
\label{sec:org463bb52}

Uno \href{../../../../../../../org/roam/20250103145124-topologia.org}{spazio topologico} \(X\) si dice \uline{Baire} se soddisfa le seguenti condizioni equivalenti:
\begin{enumerate}
\item ogni \href{../../../../../../../org/roam/20250131155822-operazioni_insiemistiche_tra_classi_mk.org}{sottoinsieme} di \(X\) \href{../../../../../../../org/roam/20250103145124-topologia.org}{aperto} non \href{../../../../../../../org/roam/20250131161811-insieme_vuoto_mk.org}{vuoto} è \href{../../../../../../../org/roam/20250419122752-insieme_magro.org}{non-magro};
\item ogni \href{../../../../../../../org/roam/20250131155822-operazioni_insiemistiche_tra_classi_mk.org}{sottoinsieme} \href{../../../../../../../org/roam/20250419122752-insieme_magro.org}{comagro} di \(X\) è \href{../../../../../../../org/roam/20250301193045-sottoinsieme_denso.org}{denso};
\item l'\href{../../../../../../../org/roam/20250131183141-classe_intersezione.org}{intersezione} di una quantità \href{../../../../../../../org/roam/20250111143651-insieme_numerabile.org}{numerabile} di sottoinsiemi densi e aperti di \(X\) è denso.
\end{enumerate}

In particolare, se \(X\) è non vuoto e Baire, allora l'intersezione di ogni coppia di \href{../../../../../../../org/roam/20250304152026-sottoinsiemi_gdelta_e_fsigma.org}{\(\bm{G}_{\delta}\)} densi di \(X\) è densa, e pertanto non vuota.
\subsubsection{Sottospazi aperti di uno spazio di Baire è uno spazio di Baire}
\label{sec:orgce12667}
Se \(X\) è uno spazio topologico di Baire e \(A \subseteq X\) è aperto, allora \(A\) è uno \href{../../../../../../../org/roam/20250103163814-sottospazio_topologico.org}{spazio topologico} di Baire.
:ID:       a18bc557-1ac7-4d19-8cd8-ad3923e1a68f
:ROAM\textsubscript{ALIASES}: BP ``Baire Property''
Sia \(X\) uno \href{../../../../../../../org/roam/20250103145124-topologia.org}{spazio topologico}.
\subsection{Insiemi uguali modulo un magro}
\label{sec:orgeba7346}
Dati due \href{../../../../../../../org/roam/20250131155822-operazioni_insiemistiche_tra_classi_mk.org}{sottoinsiemi} \(A, B \subseteq X\) si dirà che
\begin{equation*}
A \mathrel{=^{*}} B
\end{equation*}
se la \href{../../../../../../../org/roam/20250514162818-differenza_simmetrica_tra_due_insiemi.org}{differenza simmetrica} \(A\mathrel{\triangle} B \coloneqq (A\setminus B)\cup (B\setminus A)\) è un \href{../../../../../../../org/roam/20250419122752-insieme_magro.org}{insieme magro}. \footnote{Vedi ``\href{../../../../../../../org/roam/20250131155822-operazioni_insiemistiche_tra_classi_mk.org}{Sottrazione di classi MK}'' e ``\href{../../../../../../../org/roam/20250131155822-operazioni_insiemistiche_tra_classi_mk.org}{Unione di classi MK}''
:ID:       648c8718-f653-4932-9c00-900d90560720
:ROAM\textsubscript{ALIASES}: ``Insieme coanalitico''}
\subsection{Definizione}
\label{sec:org95bdf49}

Un \href{../../../../../../../org/roam/20250131155822-operazioni_insiemistiche_tra_classi_mk.org}{sottoinsieme} \(A \subseteq X\) ha la \uline{proprietà di Baire} (\emph{Baire Property} o BP) se esiste un \uline{\href{../../../../../../../org/roam/20250103145124-topologia.org}{aperto}} \(U \subseteq X\) tale che
\begin{equation*}
A\mathrel{=^{*}}B.
\end{equation*}

Si definisce \(\operatorname{BP}(X)\) come la collezione di tutti i sottoinsiemi di \(X\) con BP.
\subsection{Definizione}
\label{sec:orgcda100c}

Sia \(X\) uno spazio topologico metrizzabile e separabile.
\begin{itemize}
\item \(A \subseteq X\) è detto \uline{analitico} se esiste uno spazio Polacco \(Y\) e una funzione continua
\begin{equation*}
f:Y\to X
\end{equation*}
tale che \(f(Y) = A\).
\item \(C \subseteq X\) è detto \uline{coanalitico} se \(X\setminus C\) è analitico.
\item \(B \subseteq X\) è detto \uline{bianalitico} se \(B\) e \(X\setminus B\) sono analitici.
\end{itemize}

La collezione dei sottoinsiemi di \(X\) analitici, coanalitici e bianalitici è indicata, rispettivamente, da
\begin{equation*}
\bm{\Sigma}_{1}^{1}(X),\quad \bm{\Pi}_{1}^{1}(X),\quad \bm{\Delta}_{1}^{1}(X)
\end{equation*}
\subsection{Proposizione}
\label{sec:org1eb824f}

Sia \(X\) uno spazio Polacco, e sia \(\emptyset \neq A \subseteq X\). Sono fatti equivalenti:
\begin{enumerate}
\item \(A\) è analitico;
\item \(A\) è immagine continua di \(\omega^{\omega}\);
\item \(A= \pi_{X}(C)\) per qualche \(C \in \bm{\Pi}_{1}^{0}\left(X\times \omega^{\omega}\right)\), \(C\neq \emptyset\), dove \(\pi_{X}\) è la proiezione su \(X\)
\item \(A = \pi_{X}(C)\) per qualche spazio Polacco \(Y\) e per qualche \(C \in \bm{{\operatorname{Bor}}}(X\times Y)\), \(C\neq \emptyset\);
\item \(A=f(C)\) per qualche spazio Polacco \(Y\), per qualche \(C \in \bm{{\operatorname{Bor}}}(X\times Y)\) e per qualche funzione Boreliana \(f:Y\to X\).
\end{enumerate}
\section{Tutti analytic BP}
\label{sec:orgb092fd1}

\subsection{Gioco di Banach-Mazur}
\label{sec:orge8e0b9e}
Sia \(X\) uno \href{../../../../../../../org/roam/20250103145124-topologia.org}{spazio topologico} non vuoto, e sia \(A \subseteq X\).

Il \uline{gioco di Banach-Mazur} di \(A\), denotato con \(G^{**}(A)\) oppure con \(G^{**}(A,X)\) è un \href{../../../../../../../org/roam/20250513155732-logic_game.org}{gioco logico} \hyperref[sec:org21a45cb]{di Gale-Stewart} codificato come segue: i giocatori I e II si alternano scegliendo sottoinsiemi aperti non vuoti di \(X\)
\begin{equation*}
\begin{tikzcd}[ampersand replacement=\&,cramped,sep=tiny]
	{\text{I}} \& {U_0} \&\& {U_1} \&\& {U_2} \&\& \cdots \\
	{\text{II}} \&\& {V_0} \&\& {V_1} \&\& \cdots
\end{tikzcd}
\end{equation*}
\hyperref[sec:org32c74c6]{tali che} \(U_{0}\supseteq V_{0}\supseteq U_{1}\supseteq V_{1}\supseteq \dots\)

Il giocatore II vince se
\begin{equation*}
\bigcap_{n \in \omega} U_{n} = \bigcap_{n \in \omega} V_{n} \subseteq A.
\end{equation*}
\subsection{Gioco di Choquet}
\label{sec:org9c9de7e}
Sia \((X,\tau)\) uno \href{../../../../../../../org/roam/20250103145124-topologia.org}{spazio topologico} non vuoto. Il gioco di Choquet \(G_{X}\) è un \href{../../../../../../../org/roam/20250513155732-logic_game.org}{gioco logico} \hyperref[sec:org21a45cb]{di Gale-Stewart} codificato come segue: i giocatori I e II si alternano scegliendo sottoinsiemi aperti non vuoti di \(X\):
\begin{equation*}
\begin{tikzcd}[ampersand replacement=\&,cramped,sep=tiny]
	{\text{I}} \& {U_0} \&\& {U_1} \&\& {U_2} \&\& \cdots \\
	{\text{II}} \&\& {V_0} \&\& {V_1} \&\& \cdots
\end{tikzcd}
\end{equation*}
tali che \(U_{0} \supseteq V_{0}\supseteq U_{1}\supseteq V_{1}\supseteq \dots\)

Il giocatore II vince se
\begin{equation*}
\bigcap_{n \in \omega} V_{n} = \bigcap_{n \in \omega} U_{n} \neq \emptyset.
\end{equation*}
e, poiché il gioco è \hyperref[sec:org8c60113]{totale}, il giocatore II vince se
\begin{equation*}
\bigcap_{n \in \omega} V_{n} = \bigcap_{n \in \omega} U_{n} = \emptyset.
\end{equation*}
\subsection{Caratterizzazione degli spazi di Baire tramite il gioco di Choquet}
\label{sec:orga746ed3}
\subsubsection{Teorema}
\label{sec:org3cbe1dc}

Uno \href{../../../../../../../org/roam/20250103145124-topologia.org}{spazio topologico} \(X\) è uno \href{../../../../../../../org/roam/20250514154101-spazio_topologico_di_baire.org}{spazio topologico di Baire} se e solo se il giocatore I \uline{non ha una \hyperref[sec:org1db90c5]{strategia} \hyperref[sec:org1db90c5]{vincente}} nel \hyperref[sec:org9c9de7e]{gioco di Choquet} \(G_{X}\).
\subsubsection{Spazio di Choquet}
\label{sec:org12a4e1c}
\begin{enumerate}
\item Definizione
\label{sec:orga058a3c}

Uno spazio topologico \(X\) è detto \uline{spazio di Choquet} se il giocatore II ha una strategia vincente in \(G_{X}\).
\item Osservazione
\label{sec:org31657a2}

Se il giocatore I non ha una strategia vincente, \textbf{non è detto} che il giocatore II ne abbia una.

Viceversa, però, se II ha una strategia vincente, allora necessariamente I non ne ha una. Quindi ogni spazio di Choquet è uno spazio topologico di Baire.
\item Aperti non vuoti di spazi di Choquet sono Choquet
\label{sec:orge945bdd}
I \href{../../../../../../../org/roam/20250103163814-sottospazio_topologico.org}{sottospazi} \href{../../../../../../../org/roam/20250103145124-topologia.org}{aperti} non vuoti di uno spazio di Choquet sono spazi di Choquet.
\item Prodotto di spazi di Choquet è Choquet
\label{sec:orge4a4a9a}
Il \href{../../../../../../../org/roam/20250109154723-topologia_prodotto.org}{prodotto} finito di Spazi di Choquet sono spazi di Choquet.
:ID:       8d136b11-9afd-48e7-aea3-dcc74393aff8
\end{enumerate}
\subsection{Teorema I}
\label{sec:org5e232d7}

Sia \(X\) uno \href{../../../../../../../org/roam/20250103145124-topologia.org}{spazio topologico} \href{../../../../../../../org/roam/20250131161811-insieme_vuoto_mk.org}{non vuoto}, e sia \(A \subseteq X\) un \href{../../../../../../../org/roam/20250131155822-operazioni_insiemistiche_tra_classi_mk.org}{sottoinsieme} qualsiasi. Allora \(A\) è \href{../../../../../../../org/roam/20250419122752-insieme_magro.org}{comagro} se e solo se il giocatore II ha una \hyperref[sec:org1db90c5]{strategia vincente} nel \hyperref[sec:orge8e0b9e]{gioco di Banach-Mazur} \(G^{**}(A)\).
\subsubsection{Dimostrazione}
\label{sec:org3046cd1}

(\(\Rightarrow\)): Se \(A\) è comagro, allora esistono \((W_{n})_{n \in\omega}\) aperti densi di \(X\) tali che
\begin{equation*}
\bigcap_{n \in\omega} W_{n} \supseteq A.
\end{equation*}

Il giocatore II gioca \(V_{n} \coloneqq W_{n}\cap U_{n}\); questo è aperto, e inoltre è non vuoto poiché \(W_{n}\) è denso in \(X\).

(\(\Leftarrow\)): Sia \(\sigma\) una strategia vincente di II. Si costruisce \(\sigma' \subseteq \sigma\) albero potato e non vuoto per induzione sulla lunghezza delle stringhe.
\begin{itemize}
\item \(\emptyset \in \sigma'\).
\item Sia \(s=\langle U_{0},V_{0},\dots,U_{n}\rangle\). Allora esiste un unico \(V_{n} \subseteq U_{n}\) tale che \(s\concat V_{n} \in \sigma\). Si pone \(s\concat V_{n} \in \sigma'\).
\item Sia \(s = \langle U_{0},V_{0},\dots, U_{n}, V_{n}\rangle \in \sigma'\). Per ogni sottoinsieme aperto \(U \subseteq V_{n}\) si definisce \(U^{*}\) l'unico sottoinsieme di \(U\) tale che
\begin{equation*}
  s\concat \langle U,U^{*}\rangle \in \sigma
\end{equation*}

È possibile, tramite un'applicazione del Lemma di Zorn, garantire l'esistenza di una collezione massimale \(\mathcal{U}_{s}\) di aperti non vuoti \(U \subseteq V_{n}\) tale che la collezione \(\mathcal{V}_{s} \coloneqq \set{U^{*}\mid U \in \mathcal{U}_{s}}\) sia composta da insiemi a due a due disgiunti.

Infatti, data una catena di collezioni di aperti che soddisfino la proprietà richiesta \((\mathcal{U}_{\alpha})_{\alpha}\) ordinata dall'inclusione, allora
\begin{equation*}
\mathcal{U}^{\star}\coloneqq \bigcup_{\alpha} \mathcal{U}_{\alpha}
\end{equation*}
è un maggiorante della catena, in quanto detto
\begin{equation*}
\mathcal{V}^{\star} \coloneqq \set{U^{*}\mid U \in \mathcal{U}^{\star}}
\end{equation*}
dati \(V,V' \in \mathcal{V}^{\star}\) allora esiste \(\mathcal{U}_{\alpha_{0}}\) ed esistono \(U_{0},U_{1} \in \mathcal{U}_{\alpha_{0}}\) tali che
\begin{equation*}
U_{0}^{*}=V,\quad U_{1}^{*} = V'
\end{equation*}
e pertanto \(V\cap V' =\emptyset\).

Dunque, per ogni \(U \in \mathcal{U}_{s}\), \(s\concat U \in \sigma'\).

Inoltre \(\bigcup \mathcal{V}_{s}\) è denso in \(V_{n}\). Infatti, se per assurdo esistesse \(B \subseteq V_{n}\) aperto tale che \(B\cap \bigcup \mathcal{V}_{s} = \emptyset\), allora \(\mathcal{U}_{s}\cup \set{B}\) viola la massimalità di \(\mathcal{U}_{s}\).
\end{itemize}

Sia ora, per ogni \(n \in\omega\):
\begin{equation*}
W_{n+1} \coloneqq \bigcup_{\substack{s \in \sigma'\\
\operatorname{lh}(s) = 2n}} \bigcup \mathcal{V}_{s} = \bigcup_{\langle U_{0},V_{0},\dots, U_{n+1}, V_{n+1}\rangle \in \sigma'} V_{n+1}
\end{equation*}

Per ogni \(n \in \omega\), \(W_{n+1} \subseteq X\) è denso.
\begin{itemize}
\item \(W_{1}\) è denso, poiché \(\mathcal{U}_{\emptyset}\) è una collezione di aperti di \(X\) tali che \(\mathcal{V}_{\emptyset}\) sia composta da insiemi a due a due disgiunti, e pertanto, se vi fosse \(B \subseteq X\) aperto tale che \(B\cap W_{1} = \emptyset\), allora \(\mathcal{U}_{\emptyset}\cup\set{B}\) viola la massimalità di \(\mathcal{U}_{\emptyset}\).
\item Se \(W_{n+1}\) è denso, allora lo è anche \(W_{n+2}\). Sia \(B \subseteq X\) aperto.

Siccome \(W_{n+1}\) è denso allora \(W_{n+1}\cap B\neq \emptyset\), ed esiste \(\tilde{s} = \langle U_{0},V_{0},\dots,U_{n}, V_{n}\rangle \in \sigma'\) tale che \(B\cap\bigcup \mathcal{V}_{\tilde{s}} \neq \emptyset\).

Quindi esistono \(V_{n}\supseteq U \supseteq V\) tali che \(\tilde{s}\concat \langle U, V\rangle \in \sigma'\), con \(V\cap B\neq \emptyset\). Infatti, se così non fosse, allora \(\mathcal{U}_{\tilde{s}}\cup\set{V_{n}\cap B}\) contraddice la massimalità di \(\mathcal{U}_{\tilde{s}}\).

Poiché \(\bigcup \mathcal{V}_{s\concat \langle U, V\rangle}\) è denso in \(V\), allora \(\bigcup \mathcal{V}_{s\concat \langle U, V\rangle} \cap B \neq \emptyset\), ed inoltre
\begin{equation*}
  \bigcup \mathcal{V}_{\tilde{s}\concat \langle U, V\rangle} \subseteq W_{n+2}
\end{equation*}
e pertanto \(W_{n+2}\cap B\neq \emptyset\).
\end{itemize}

Per finire, si dimostra che \(\bigcap_{n \in \omega} W_{n+1} \subseteq A\). Sia \(x \in \bigcap_{n \in \omega} W_{n+1}\).

Allora esiste \((U_{i}, V_{i})_{i \in \omega} \in [\sigma']\) tale che \(x \in V_{n}\) per ogni \(n\). Questa si costruisce per induzione.
\begin{itemize}
\item Poiché \(x \in W_{1}\), allora esiste \(\langle U_{0},V_{0},U_{1},V_{1}\rangle \in \sigma'\) tale che \(x \in V_{1}\).
\item Sia ora \(p=\langle U_{0},V_{0},\dots,U_{n}, V_{n}\rangle \in\sigma'\) tale che \(x \in V_{n}\).

Siccome \(x \in W_{n+1}\) allora esiste \(p' \in\sigma'\),
\begin{equation*}
  p'\coloneqq \langle U_{0}',V_{0}',\dots,U_{n+1}',V_{n+1}'\rangle
\end{equation*}
tale che \(x \in V_{n+1}\). Necessariamente \(p'\) estende \(p\).

Infatti, si supponga per assurdo che \(p \neq \langle U_{0}',V_{0}',\dots,U_{n}',V_{n}'\rangle\), e sia \(j\le n\) il primo indice tale che
\begin{equation*}
  \langle U_{j}, V_{j}\rangle \neq \langle U_{j}', V_{j}'\rangle.
\end{equation*}
Necessariamente allora \(U_{j}\neq U_{j}'\), poiché \(V_{j}\) e \(V_{j}'\) sono univocamente determinati dall'insieme precedente. In particolare, però:
\begin{equation*}
  U_{j}, U_{j}' \in \mathcal{U}_{\langle U_{0},V_{0},\dots,U_{j-1},V_{j-1}\rangle} = \mathcal{U}_{\langle U_{0}',V_{0}',\dots,U_{j-1}',V_{j-1}'\rangle}
\end{equation*}
e pertanto, per definizione, \(V_{j}\cap V_{j'} = \emptyset\). Assurdo, poiché \(x \in V_{j}\cap V_{j}'\).
\end{itemize}

Dunque \(\langle U_{0},V_{0},\dots,U_{n}, V_{n}, U_{n+1}', V_{n+1}'\rangle\) estende la sequenza iniziale.

In particolare, quindi \(x \in \bigcap_{n \in \omega} V_{n}\).

Poiché \(\sigma\) è una strategia vincente per il giocatore II, allora per ogni \((U_{i}, V_{i})_{i \in \omega} \in [\sigma'] \subseteq [\sigma]\),
\begin{equation*}
\bigcap_{i \in \omega} U_{i} = \bigcap_{i \in \omega} V_{i}\subseteq A
\end{equation*}
e dunque \(x \in A\).\qed
\subsection{Teorema II}
\label{sec:orged1da91}
Se \(X\) è uno \href{../../../../../../../org/roam/20250103145124-topologia.org}{spazio topologico} \hyperref[sec:org12a4e1c]{di Choquet} non \href{../../../../../../../org/roam/20250131161811-insieme_vuoto_mk.org}{vuoto} ed esiste una \href{../../../../../../../org/roam/20250301193511-spazio_metrico.org}{distanza} \(d\) su \(X\) le cui \href{../../../../../../../org/roam/20250301193511-spazio_metrico.org}{palle aperte} sono aperti di \(X\), allora:

\(A\) è \href{../../../../../../../org/roam/20250419122752-insieme_magro.org}{magro} in un \href{../../../../../../../org/roam/20250103145124-topologia.org}{aperto} non vuoto se e solo se il giocatore I ha una \hyperref[sec:org1db90c5]{strategia vincente} nel \hyperref[sec:orge8e0b9e]{gioco di Banach-Mazur} \(G^{**}(A)\).
\subsubsection{Dimostrazione}
\label{sec:org20477db}

(\(\Rightarrow\)): Se \(A\) è magro in \(Y \subseteq X\), sia per ogni \(n \in \omega\): \(W_{n} \subseteq Y\) aperti densi di \(Y\), con
\begin{equation*}
\bigcap_{n \in\omega} W_{n} \subseteq Y \setminus A.
\end{equation*}

\hyperref[sec:orge945bdd]{Poiché} \(Y\) è uno \hyperref[sec:org12a4e1c]{spazio di Choquet}, allora nel \hyperref[sec:org21a45cb]{gioco}:
\begin{equation*}
\begin{tikzcd}[ampersand replacement=\&,cramped, sep=tiny]
	{\text{I}} \&\& {B_1} \&\& {B_2} \&\& \dots \\
	{\text{II}} \& {A_0} \&\& {A_1} \&\& \dots
\end{tikzcd}
\end{equation*}
con gli aperti non vuoti \(Y\supseteq V_{0}\supseteq U_{1}\supseteq V_{1}\supseteq \dots\) in cui I vince sse \(\bigcap_{n \in \omega}{B_{n}} \neq \emptyset\), I ha una \hyperref[sec:org1db90c5]{strategia vincente}. Questo infatti è un gioco di Choquet a giocatori invertiti.

Sia quindi \(\sigma\) la strategia vincente di I in questo gioco di Choquet.

Nel gioco \(G^{**}(A)\), il giocatore I pone \(U_{0} \coloneqq Y\). Si costruisce per induzione la strategia vincente per I.

Al passo \(n+1\)-esimo, sia \((U_{0},V_{0},\dots, U_{n}, V_{n})\) la sequenza di insiemi giocati. Si pone, per ogni \(i\le n\): \(V_{i}'\coloneqq V_{i}\cap W_{i}\), e si sceglie \(U_{n+1}\) come l'unico sottoinsieme aperto non vuoto di \(V_{n}\) tale che
\begin{equation*}
(V_{0}', U_{1}, V_{1}', U_{2},\dots, V_{n}', U_{n+1}) \in\sigma.
\end{equation*}

Allora \(\bigcap_{n \in \omega} U_{n}\neq\emptyset\) e inoltre
\begin{equation*}
\bigcap_{n \in\omega} U_{n} = \bigcap_{n \in\omega} V_{n}' \subseteq \bigcap_{n \in \omega} W_{n} \subseteq Y\setminus A
\end{equation*}
e dunque \(\bigcap_{n \in\omega} U_{n} \not\subseteq A\).

(\(\Leftarrow\)): Sia \(\sigma\) una strategia vincente per I in \(G^{**}(A)\), e sia \(U_{0}\) l'elemento di partenza per \(\sigma\).

Si costruisce una strategia \(\sigma'\) per I, vincente, e tale che l'insieme giocato al passo \(n\)-esimo \(U_{n}\) abbia diametro (rispetto alla metrica \(d\)):
\begin{equation*}
\operatorname{diam}(U_{n})<2^{-n}.
\end{equation*}

Al passo \(n+1\), sia \((U_{0},V_{0},\dots,U_{n}, V_{n})\) la sequenza di insiemi giocati, e sia \(v_{n} \in V_{n}\). Si definisce
\begin{equation*}
V_{n}'\coloneqq V_{n}\cap B_{d}(v_{n}, 2^{-n-1}), \qquad \operatorname{diam}(V_{n}) \le 2^{-n}
\end{equation*}
che è un aperto non vuoto. Si pone infine \(U_{n+1}\) come l'unico sottoinsieme aperto di \(V_{n}'\) tale che
\begin{equation*}
(U_{0},V_{0},\dots,U_{n}, V_{n}', U_{n+1}) \in \sigma.
\end{equation*}

Questo \(U_{n+1}\) è la risposta secondo la strategia \(\sigma'\), in quanto \(\operatorname{diam}(U_{n})\le \operatorname{diam}(V_{n}')\le 2^{-n}\).

Siccome \(\sigma'\) è una strategia vincente per I, allora
\begin{equation*}
\emptyset\neq\bigcap_{n \in \omega} U_{n}
\end{equation*}
e inoltre
\begin{equation*}
\operatorname{diam}\left(\bigcap_{n \in \omega} U_{n}\right) = 0
\end{equation*}
Segue che \(\bigcap_{n \in \omega} U_{n} = \set{x}\), con \(x \in U_{0}\setminus A\).

Si definisce l'aperto
\begin{equation*}
W_{n} \coloneqq \bigcup_{\langle U_{0},V_{0},\dots,U_{n}\rangle \in \sigma'} U_{n}.
\end{equation*}
Questo è denso in \(U_{0}\): se \(B \subseteq U_{0}\) è aperto, allora sicuramente \(\langle U_{0},B\rangle \in \sigma'\). Si costruisce \(\langle U_{0},B,U_{1},V_{1},\dots,U_{n}\rangle \in \sigma'\), e in particolare \(U_{n} \subseteq B\) e pertanto \(W_{n}\cap B\supseteq U_{n}\neq\emptyset\).

Inoltre \(\bigcap_{n \in \omega} W_{n} \subseteq U_{0}\setminus A\), dunque la tesi.\qed
:ID:       f72c58ad-fd18-440f-8ce2-30054f4996f7
\subsection{Lemma}
\label{sec:org29f61ee}

Sia \(X\) uno \href{../../../../../../../org/roam/20250103145124-topologia.org}{spazio topologico} \hyperref[sec:org12a4e1c]{di Choquet} non \href{../../../../../../../org/roam/20250131161811-insieme_vuoto_mk.org}{vuoto} tale che esista una \href{../../../../../../../org/roam/20250301193511-spazio_metrico.org}{distanza} \(d\) su \(X\) le cui \href{../../../../../../../org/roam/20250301193511-spazio_metrico.org}{palle aperte} sono aperti di \(X\). Sia \(A \subseteq X\).

Se per ogni aperto \(U \subseteq X\) il \hyperref[sec:orge8e0b9e]{gioco} \(G^{**}\left((X\setminus A)\cup U\right)\) è \hyperref[sec:org7f68e17]{determinato} allora \(A \subseteq X\) ha \href{../../../../../../../org/roam/20250514154039-proprieta_di_baire.org}{BP}.
\subsubsection{Dimostrazione}
\label{sec:orga50e260}

Sia \(A \subseteq X\). Si definisce l'aperto
\begin{equation*}
U(A) \coloneqq \bigcup \set{
U\subseteq X\text{ aperto}\mid U\setminus A\text{ è magro}
}.
\end{equation*}
Allora \(U(A)\setminus A\) è magro e inoltre, se \(A\) ha la BP, allora \(A\mathrel{=^{*}} U(A)\). Questo segue direttamente dal Teorema 8.29 del Kechris.

In particolare quindi il gioco è determinato  per
\begin{equation*}
G^{**}\left((X\setminus A)\cup U(A)\right).
\end{equation*}

Necessariamente è il giocatore II a vincere questo gioco. Infatti, si supponga per assurdo che I abbia una strategia vincente. Allora, per il \href{../../../../../../../org/roam/20250514174717-teorema_di_caratterizzazione_dei_comagri_tramite_il_gioco_di_banach_mazur.org}{Teorema II} \((X\setminus A)\cup U(A)\) è magro in un aperto non vuoto \(B\). In particolare, quindi \(U(A)\) è magro in \(B\), ovvero \(U(A)\cap B\) è magro in \(B\).
\begin{itemize}
\item Se \(U(A)\cap B\neq\emptyset\), siccome \(B \subseteq X\) è un aperto di uno spazio di Baire, allora è uno spazio di Baire; inoltre \(U(A)\cap B\) è un aperto non vuoto di \(B\), quindi è \uline{non magro}. Assurdo.
\item Se invece \(U(A)\cap B = \emptyset\), si consideri il seguente insieme, magro per definizione:
\begin{equation*}
  \left((X\setminus A)\cup U(A)\right) \cap B = \left((X\setminus A)\cap B\right) \cup \left(U(A)\cap B\right) = (X\setminus A)\cap B = B\setminus A
\end{equation*}
Allora, per definizione di \(U(A)\), \(B \subseteq U(A)\). Assurdo.
\end{itemize}

Pertanto, per il \href{../../../../../../../org/roam/20250514174717-teorema_di_caratterizzazione_dei_comagri_tramite_il_gioco_di_banach_mazur.org}{Teorema I}, \((X\setminus A)\cup U(A)\) è comagro. Ma
\begin{equation*}
(X\setminus A)\cup U(A) = X \setminus\left(A\setminus U(A)\right)
\end{equation*}
e pertanto \(A\setminus U(A)\) è magro. Per il risultato precedente \(U(A)\setminus A\) è magro, e dunque
\begin{equation*}
A\mathrel{\triangle}U(A)
\end{equation*}
è magro, ovvero \(A\) ha la BP.\qed
\subsection{Gioco di Banach-Mazur Unfolded}
\label{sec:org8e1057c}
Sia \(X\) uno \href{../../../../../../../org/roam/20250301194013-spazio_polacco.org}{spazio polacco} non vuoto con una \href{../../../../../../../org/roam/20250301193511-spazio_metrico.org}{metrica} fissata e sia \(\mathcal{W}\) una \href{../../../../../../../org/roam/20250525113346-base_debole_di_uno_spazio_topologico.org}{base debole} \href{../../../../../../../org/roam/20250111143651-insieme_numerabile.org}{numerabile} di \(X\).

Dato \(F \subseteq X\times \omega^{\omega}\), il \uline{gioco di Banach-Mazur unfolded} \(G^{**}_{\text{u}}(F)\) è il \href{../../../../../../../org/roam/20250513155732-logic_game.org}{gioco logico} \hyperref[sec:org21a45cb]{di Gale-Stewart} codificato come segue:
\begin{equation*}
\begin{tikzcd}[ampersand replacement=\&,cramped,sep=tiny]
	{\text{I}} \& {U_0} \&\& {U_1} \&\& \dots \\
	{\text{II}} \&\& {y_0,V_0} \&\& {y_1, V_{1}} \& \dots
\end{tikzcd}
\end{equation*}
tali che:
\begin{itemize}
\item per ogni \(i \in \omega\): \(U_{i}, V_{i} \in \mathcal{W}\), \(y_{n} \in \omega\);
\item \(\operatorname{diam}(U_{n}), \operatorname{diam}(V_{n}) < 2^{-n}\);
\item \(U_{0}\supseteq V_{0}\supseteq U_{1}\supseteq V_{1}\supseteq \dots\)
\end{itemize}

posto
\begin{equation*}
\set{x}\coloneqq\bigcap_{i \in \omega} \operatorname{Cl}_{X}(U_{n}) = \bigcap_{i \in \omega} \operatorname{Cl}_{X}(V_{n})
\end{equation*}
e \(y\coloneqq (y_{i})_{i \in \omega} \in \omega^{\omega}\), il \uline{giocatore II vince} sse
\begin{equation*}
(x,y) \in F \subseteq X\times \omega^{\omega}.
\end{equation*}
\subsubsection{Applicazione del Teorema di Gale-Stewart}
\label{sec:org682507b}

Se \(F\) è aperto o chiuso di \(X\times\omega^{\omega}\), allora \(G^{**}_{\text{u}}(F)\) è determinato.
\begin{enumerate}
\item Dimostrazione
\label{sec:org871b9fe}

\begin{itemize}
\item Si costruisce un \(\mathcal{A}\)-schema su \(X\). Per ogni \(\langle (A_{0},a_{0}),\dots,(A_{k},a_{k})\rangle \in \mathcal{A}^{<\omega}\), si definisce
\begin{equation*}
  B_{\langle (A_{0},a_{0}),\dots,(A_{k},a_{k})\rangle} \coloneqq \begin{cases}
  	\bigcap_{i\le k} \operatorname{Cl}_{X}(A_{i}) & \forall\,i\le k: \ \operatorname{diam}(A_{i})<2^{-i}\\
  	& A_{0} \supseteq A_{1}\supseteq \dots \supseteq A_{k}\\[1em]
  	\emptyset &\text{altrimenti}
      \end{cases}
\end{equation*}

Ovviamente, per ogni \(s \in \mathcal{A}^{<\omega}\) e per ogni \(a \in \mathcal{A}\):
\begin{equation*}
  	B_{s\concat a} \subseteq B_{s}
\end{equation*}
ed inoltre per ogni \(x \in \mathcal{A}^{\omega}\): \(\operatorname{diam}(B_{x\upharpoonright n})\to 0\).

Inoltre ciascun \(B_{s}\) è chiuso (poiché intersezione finita di chiusi oppure il vuoto) e pertanto, per il Lemma 1.3.6, questo schema induce una funzione continua
\begin{equation*}
  	f:[T]\to X,\quad T\coloneqq \set{s \in \mathcal{A}^{<\omega}\mid B_{s}\neq \emptyset}.
\end{equation*}

\item Si osserva che \(T=\set{\langle (A_{i},a_{i})\rangle_{i\le k}: A_{0}\supseteq A_{1}\supseteq \dots A_{k} \,\land\, \forall\, i\le k:\ \operatorname{diam}A_{i}<2^{-i}}\), ovvero è esattamente l'albero delle posizioni ammissibili di \(G^{**}_{\text{u}}(F)\), ed inoltre \(T\) è un albero potato non vuoto.

\item La funzione
\begin{align*}
g: \mathcal{A}^{\omega} &\longrightarrow \omega^{\omega}\\
\left((A_{i},a_{i})\right)_{i \in\omega}&\longmapsto (a_{2i+1})_{i \in\omega}
\end{align*}
è continua.

\item Si ottiene quindi una funzione continua
\begin{align*}
\psi: [T] &\longrightarrow X\times\omega^{\omega}\\
s &\longmapsto \left(f(s),g(s)\right)
\end{align*}
\end{itemize}

Sia ora dunque \(F \subseteq X\times\omega^{\omega}\) aperto o chiuso, sia \(F'\coloneqq [T]\setminus\psi^{-1}(F)\) aperto o chiuso, e si consideri il \hyperref[sec:org32c74c6]{Gioco di Gale-Stewart con posizioni ammissibili} \(G(T,F')\). Per il \href{../../../../../../../org/roam/20250514144736-teorema_di_gale_stewart.org}{Teorema di Gale-Stewart}, questo è determinato, poiché \(F'\) è aperto o chiuso; ovvero esattamente uno tra i giocatori I e II ha una strategia vincente.

\uline{Caso 1}. Sia \(\sigma'\) una strategia vincente per il giocatore I nel gioco \(G(T,F')\), \(\sigma' \subseteq T \subseteq \mathcal{A}^{<\omega}\) con \([\sigma'] \subseteq F'\), ovvero \([\sigma'] \cap \psi^{-1}(F)=\emptyset\).

Si costruisce una strategia \(\sigma\) per il giocatore I nel gioco \(G^{**}_{\text{u}}(F)\):
\begin{equation*}
\sigma \coloneqq \set{
\begin{gathered}
\langle A_{0}, (A_{1},a_{1}),\dots,(A_{2k-1}, a_{2k-1}), A_{2k}\rangle,\\
\langle A_{0}, (A_{1},a_{1}),\dots,(A_{2k+1}, a_{2k+1})\rangle
\end{gathered}
\mid \langle (A_{0},a_{0}), (A_{1},a_{1}),\dots,(A_{k}, a_{k})\rangle \in \sigma'
}.
\end{equation*}

Sia quindi \(\left( U_{i}, (V_{i},y_{i}) \right)_{i \in \omega}\) una giocata per I seguendo la strategia \(\sigma\), e siano
\begin{equation*}
\set{x} \coloneqq \bigcap_{i \in \omega} U_{i} = \bigcap_{i \in \omega} V_{i},\quad y\coloneqq(y_{i})_{i \in\omega}.
\end{equation*}
Allora esiste \(s \in [\sigma']\) tale che \((x,y) = \psi(s)\) per costruzione. Siccome \([\sigma'] \cap \psi^{-1}(F) = \emptyset\) segue che \((x,y)\notin F\).

\uline{Caso 2}. Sia \(\sigma'\) una strategia vincente per il giocatore II nel gioco \(G(T,F')\), \(\sigma' \subseteq T \subseteq \mathcal{A}^{<\omega}\) con \([\sigma'] \cap  F' = \emptyset\), ovvero \([\sigma'] \subseteq \psi^{-1}(F)\)

Si costruisce una strategia \(\sigma\) per il giocatore II nel gioco \(G^{**}_{\text{u}}(F)\):
\begin{equation*}
\sigma \coloneqq \set{
\begin{gathered}
\langle A_{0}, (A_{1},a_{1}),\dots,(A_{2k-1}, a_{2k-1}), A_{2k}\rangle,\\
\langle A_{0}, (A_{1},a_{1}),\dots,(A_{2k+1}, a_{2k+1})\rangle
\end{gathered}
\mid \langle (A_{0},a_{0}), (A_{1},a_{1}),\dots,(A_{k}, a_{k})\rangle \in \sigma'
}.
\end{equation*}

Sia quindi \(\left( U_{i}, (V_{i},y_{i}) \right)_{i \in \omega}\) una giocata per I seguendo la strategia \(\sigma\), e siano
\begin{equation*}
\set{x} \coloneqq \bigcap_{i \in \omega} U_{i} = \bigcap_{i \in \omega} V_{i},\quad y\coloneqq(y_{i})_{i \in\omega}.
\end{equation*}
Allora esiste \(s \in [\sigma'] \subseteq \psi^{-1}(F)\) tale che \((x,y) = \psi(s)\) per costruzione, e pertanto \((x,y) \in F\).\qed
:ID:       8cf7248e-0ea4-4a86-b862-49881cabb122
\end{enumerate}
\subsection{Teorema}
\label{sec:org21015ec}

Sia \(X\) uno spazio polacco con una metrica fissata e sia \(\mathcal{W}\) una base debole di X.

Dato \(F \subseteq X\times \omega^{\omega}\) si consideri il \hyperref[sec:org8e1057c]{**-gioco}: \(G^{**}_{\text{u}}(F)\). Indicato con \(A\coloneqq \pi_{X}(F)\):
\begin{enumerate}
\item se I ha una \hyperref[sec:org1db90c5]{strategia} \hyperref[sec:org1db90c5]{vincente} in \(G^{**}_{\text{u}}(F)\), allora \(A\) è magro in un aperto non vuoto di \(X\times\omega^{\omega}\);
\item se II ha una \hyperref[sec:org1db90c5]{strategia} \hyperref[sec:org1db90c5]{vincente} in \(G^{**}_{\text{u}}(F)\) allora \(A\) è comagro.
\end{enumerate}
\subsubsection{Dimostrazione}
\label{sec:org842a52a}

\begin{enumerate}
\item Sia \(\sigma\) una \hyperref[sec:org1db90c5]{strategia} \hyperref[sec:org1db90c5]{vincente} per I, e sia \(U_{0}\) la prima mossa. Si mostra che \(A\) è \href{../../../../../../../org/roam/20250419122752-insieme_magro.org}{magro} in \(U_{0}\).

Per ogni \(a \in \omega\) e per ogni \(p \in \sigma\) della forma:
\begin{equation*}
 p=\langle
 	U_{0},({y}_{0}, V_{0}), \dots, U_{n-1}, ({y}_{n-1}, V_{n-1}), U_{n}
 \rangle
\end{equation*}
si definisce \(F_{p,a} \subseteq U_{0}\):
\begin{align*}
 F_{p,a} = \{&z \in U_{n}\mid \text{per ogni mossa legale }(a, V_{n})\\
 &\text{se } U_{n+1}\text{ è l'unico elemento di }\mathcal{W}\text{ tale che}\\
 &p \concat \langle(a,V_{n}), U_{n+1}\rangle \in \sigma \text{ allora } z \notin U_{n+1}\}
\end{align*}
\begin{itemize}
\item L'insieme \(F_{p,a}\) è mai denso, poiché chiuso e con interno vuoto \footnote{Esempio 1.5.2 (2) di LMR ``Notes on descriptive set theory''
:ID:       51b0a529-1947-4b4b-a37b-4a4e4f9a0e60}. Infatti, se per assurdo \(\operatorname{Int}(F_{p,a}) \neq \emptyset\), allora esiste \(W \in \mathcal{W}\) tale che
\begin{equation*}
   W \subseteq \operatorname{Int}(F_{p,a}), \quad \operatorname{diam}(W)<2^{-n}
\end{equation*}
pertanto se II gioca \(V_{n} \coloneqq W\) allora I dovrà giocare \(U_{n+1} \subseteq V_{n} \subseteq F_{p,a}\). Ma per definizione \(U_{n+1}\cap F_{p,a} = \emptyset\). Assurdo.

Inoltre, se \(\eta \in U_{n}\setminus F_{p,a}\), allora esiste una sequenza
\begin{equation*}
   p\concat\langle (a,V_{n}), U_{n+1}\rangle \in \sigma
\end{equation*}
con \(\eta \in U_{n+1}\); siccome \(U_{n+1} \cap F_{p,a} = \emptyset\) segue
\begin{equation*}
   \eta \in U_{n+1} \subseteq U_{n}\setminus F_{p,a} \subseteq X\setminus F_{p,a}
\end{equation*}
ovvero \(F_{p,a}\) chiuso.

\item Siccome \(\sigma\) e \(\omega\) sono insiemi numerabili allora
\begin{equation*}
   \bigcup_{p \in \sigma', a \in \omega} F_{p,a}
\end{equation*}
è un insieme magro, dove \(\sigma' \subseteq \sigma\) è l'insieme delle sequenze di lunghezze dispari.
\end{itemize}

Sia ora \(x \in A\cap U_{0}\). Allora esiste \(y \in \omega^{\omega}\), \(y=(y_{i})_{i \in\omega}\) tale che \((x,y) \in F\).

Una posizione \(p \in \sigma'\):
\begin{equation*}
     p=\langle
     	U_{0},({y}_{0}, V_{0}), \dots, U_{n-1}, ({y}_{n-1}, V_{n-1}), U_{n}
     \rangle
\end{equation*}
è \uline{buona} per \((x,{y})\) se \(x \in U_{n}\). Siccome \(\sigma\) è una strategia vincente per il giocatore I, allora esiste una posizione \(p_{(x,y)} \in \sigma\) buona per \((x,y)\) e massimale, ovvero ogni estensione di \(p_{(x,y)}\) \uline{non è buona}. Ma allora, se
\begin{equation*}
 p_{(x,y)} = \langle U_{0}, (y_{0},V_{0}),\dots, U_{n}\rangle
\end{equation*}
si ha che \(x \in F_{p_{(x,y)}, y_{n}}\).

Pertanto \(A\cap U_{0} \subseteq \bigcup_{p \in \sigma', a \in \omega} F_{p,a}\) è magro.

\item Se II ha una strategia vincente per \(G^{**}_{\text{u}}(F)\), allora ha una strategia vincente \hyperref[sec:orge8e0b9e]{in \(G^{**}(A)\)}. Per il \href{../../../../../../../org/roam/20250514174717-teorema_di_caratterizzazione_dei_comagri_tramite_il_gioco_di_banach_mazur.org}{teorema precedente}, \(A\) è comagro.\qed
\end{enumerate}
\subsection{Teorema (Lusin-Sierpiński)}
\label{sec:org65a9bfc}

Sia \(X\) uno \href{../../../../../../../org/roam/20250301194013-spazio_polacco.org}{spazio polacco}. Allora ogni \href{../../../../../../../org/roam/20250525220742-insieme_analitico.org}{insieme analitico} di \(X\) ha la \href{../../../../../../../org/roam/20250514154039-proprieta_di_baire.org}{Baire Property}.
\subsubsection{Dimostrazione}
\label{sec:orge50bb88}

Siccome \(\mathrm{BP}(X)\) è una \href{../../../../../../../org/roam/20250526100313-sigma_algebra.org}{\(\sigma\)-algebra} \footnote{Vedi Prop. 1.5.9 di LMR ``Notes on Descriptive Set Theory''.} allora è chiusa per complementi, e pertanto se ogni insieme coanalitico ha BP allora si è dimostrata la tesi.

Sia dunque \(C\) un insieme coanalitico e sia \(U \subseteq X\) un aperto. Posto \(A\coloneqq (X\setminus C)\cup U\), questo è un insieme analitico, e pertanto  esiste un chiuso \(F \subseteq X\times\omega^{\omega}\) tale che \(A=\pi_{X}(F)\).

Per il \href{../../../../../../../org/roam/20250514144736-teorema_di_gale_stewart.org}{Teorema di Gale-Stewart}, allora, il \hyperref[sec:org8e1057c]{**-gioco \(G^{ * *}_{\text{u}}(F)\)} è \hyperref[sec:org7f68e17]{determinato}, ed in particolare vale una tra le condizioni a. e b. del \href{../../../../../../../org/roam/20250514175252-magrezza_dentro_ad_un_polacco_tramite_gioco_di_banach_mazur.org}{teorema precedente}.

Per i \href{../../../../../../../org/roam/20250514174717-teorema_di_caratterizzazione_dei_comagri_tramite_il_gioco_di_banach_mazur.org}{teoremi I e II}, allora, il \hyperref[sec:orge8e0b9e]{gioco \(G^{**}(A) = G^{ * *}\left((X\setminus C) \cup U\right)\)} è determinato: per il \href{../../../../../../../org/roam/20250526100910-caratterizzazione_bp_tramite_gioco_di_banach_mazur.org}{lemma precedente}, quindi \(C\) ha la BP. \qed
\end{document}
