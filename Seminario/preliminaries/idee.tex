% Created 2025-04-30 Wed 12:38
% Intended LaTeX compiler: pdflatex
\documentclass[11pt]{article}

\usepackage[utf8]{inputenc}
\usepackage[T1]{fontenc}
\usepackage{fixltx2e}
\usepackage{graphicx}
\usepackage{longtable}
\usepackage{float}
\usepackage{wrapfig}
\usepackage{rotating}
\usepackage[normalem]{ulem}
\usepackage{amsmath}
\usepackage{textcomp}
\usepackage{marvosym}
\usepackage{wasysym}
\usepackage{amssymb}
\usepackage{hyperref}
\author{ChatGPT}
\date{30 aprile 2025}
\title{Overview degli argomenti del seminario}
\begin{document}

\maketitle
\section{Polish groups}
\label{polish-groups}
\subsection{Overview}
\label{overview}
A \textbf{Polish group} is a topological group whose topology is separable and
completely metrizable; equivalently, it is a group equipped with a
Polish space topology. These groups arise naturally in analysis, algebra
and logic (for example the real numbers under addition, the unitary
group of a Hilbert space, or the infinite permutation group
\(S_\infty\)) and are fundamental in descriptive set theory. They
provide the setting for rich interactions between group actions and
definable sets: for instance, any Borel action of a Polish group on a
Polish space yields Borel or analytic orbit equivalence relations of
central interest in classification problems.
\subsection{Detailed Summary}
\label{detailed-summary}
\begin{itemize}
\item \textbf{Definition:} A \emph{topological group} \(G\) is \textbf{Polish} if its topology
is induced by a complete metric and \(G\) is separable (equivalently,
second-countable). In particular, every Polish group is a \(G_\delta\)
subgroup of some completely metrizable group. Classic examples include
\((\mathbb{R}^n,+)\), any separable Banach space (as an additive
group), the infinite symmetric group \(S_\infty\) (with the pointwise
convergence topology), and all compact metrizable groups (which are
complete).

\item \textbf{Basic properties:} Polish groups have cardinality continuum unless
trivial. They admit a compatible complete invariant metric. Notably,
any Polish group can be embedded as a closed subgroup of the unitary
group of a separable Hilbert space (by Uspenskiĭ's theorem), or of
\(S_\infty\), making \(S_\infty\) a universal Polish group.
Topologically, Polish groups are in particular \textbf{Baire spaces} (the
interior of any countable union of nowhere dense sets is empty), which
follows from the fact that they are complete metrizable.

\item \textbf{Descriptive set aspects:} Any Polish group \(G\) has an underlying
standard Borel structure, so one can consider Borel or analytic
subsets and Borel measurable homomorphisms of \(G\). Actions of Polish
groups on Polish (or standard Borel) spaces produce equivalence
relations whose complexity is studied via DST. A key result is
Effros's theorem: for a continuous action of a Polish group on a
Polish space, the orbit of any point is an analytic set, and the orbit
map \(x\mapsto G\cdot x\) is a Borel function into the Effros Borel
space of closed sets. In fact, “the condition of being Polish is
essential in classification problems of mathematics and logic” --
Polish groups naturally unify many examples and appear ubiquitously in
DST.

\item \textbf{Theorems (Kechris Ch.9):} Kechris develops the theory of Polish
groups, including the fact that any Polish group is isomorphic (as a
Borel group) to a closed subgroup of \(S_\infty\). One also has the
\textbf{Dichotomy Theorem}: for any Borel (or analytic) homomorphism of a
Polish group, the kernel and image have regular
descriptive-set-theoretic properties. More generally, results in this
chapter (9.1--9.3 of Kechris) illustrate that problems about group
actions (orbit equivalence, ergodicity, invariant measures) for Polish
groups can often be reduced to questions about well-understood model
cases. Although we do not reproduce Kechris's proofs here, the core
idea is that Polishness brings powerful Baire-category and metrization
tools (like the Open Mapping and Closed Graph Theorems) into play for
group-theoretic contexts.

\item \textbf{Relevance:} Polish groups form the stage for many structural theorems
in DST, such as the theory of Borel reducibility of equivalence
relations (orbit equivalence of Polish-group actions) and the study of
invariant sets. They connect descriptive set theory to areas like
group representation, ergodic theory, and logic. The interplay between
the algebraic structure of \(G\) and its topology (and Borel
\(\sigma\)-algebra) is key to understanding classification up to isomorphism or
orbit equivalence in a definable way.
\end{itemize}
\section{Standard Borel spaces}
\label{standard-borel-spaces}
\subsection{Overview}
\label{overview-1}
A \textbf{standard Borel space} is a measurable space isomorphic to the Borel
\(\sigma\)-algebra of some Polish space
(\href{https://en.wikipedia.org/wiki/Standard\_Borel\_space\#:\~:text=A\%20measurable\%20space\%20Image\%3A\%20,algebra.\%5B\%201}{Standard
Borel space - Wikipedia}). Equivalently, it is a set \(X\) with a
\(\sigma\)-algebra \(\Sigma\) such that there is a topology on \(X\) making it
Polish and \(\Sigma\) is exactly the collection of Borel sets for that
topology
(\href{https://en.wikipedia.org/wiki/Standard\_Borel\_space\#:\~:text=A\%20measurable\%20space\%20Image\%3A\%20,algebra.\%5B\%201}{Standard
Borel space - Wikipedia}). Standard Borel spaces capture the most
general setting for “nice” measurability theory: up to isomorphism,
there is essentially only one uncountable such space (of cardinality
continuum) and the discrete countable cases
(\href{https://en.wikipedia.org/wiki/Standard\_Borel\_space\#:\~:text=one\%20of\%20\%281\%29\%20Image\%3A\%20,is\%20reminiscent\%20of\%20Maharam\%27s\%20theorem}{Standard
Borel space - Wikipedia}).
\subsection{Detailed Summary}
\label{detailed-summary-1}
\begin{itemize}
\item \textbf{Definition:} Formally, \((X,\Sigma)\) is \textbf{standard Borel} if there
exists a Polish topology on \(X\) whose Borel \(\sigma\)-algebra is \(\Sigma\)
(\href{https://en.wikipedia.org/wiki/Standard\_Borel\_space\#:\~:text=A\%20measurable\%20space\%20Image\%3A\%20,algebra.\%5B\%201}{Standard
Borel space - Wikipedia}). In particular, \(\Sigma\) must be
countably generated and the space must admit a complete separable
metric compatible with \(\Sigma\). This notion is independent of the
specific topology chosen (up to isomorphism).

\item \textbf{Kuratowski's theorem:} A fundamental classification states that if
\(X\) is an uncountable Polish space, then its Borel space is
isomorphic (as measurable spaces) to the real line \(\mathbb{R}\), or
to \(\mathbb{Z}\), or is finite
(\href{https://en.wikipedia.org/wiki/Standard\_Borel\_space\#:\~:text=Theorem.\%20Let\%20Image\%3A\%20,is\%20reminiscent\%20of\%20Maharam\%27s\%20theorem}{Standard
Borel space - Wikipedia}). In consequence, \emph{up to Borel isomorphism
every uncountable standard Borel space has the cardinality of the
continuum}
(\href{https://en.wikipedia.org/wiki/Standard\_Borel\_space\#:\~:text=one\%20of\%20\%281\%29\%20Image\%3A\%20,is\%20reminiscent\%20of\%20Maharam\%27s\%20theorem}{Standard
Borel space - Wikipedia}). Thus any two uncountable standard Borel
spaces are isomorphic as measurable spaces, which justifies treating
“the” standard Borel structure on a continuum as unique.

\item \textbf{Properties:} Standard Borel spaces enjoy many convenient properties
not shared by arbitrary measurable spaces. For instance, any bijection
between standard Borel spaces that is measurable in one direction is
automatically an isomorphism (its inverse is measurable)
(\href{https://en.wikipedia.org/wiki/Standard\_Borel\_space\#:\~:text=,are\%20standard\%20Borel\%20spaces\%20and}{Standard
Borel space - Wikipedia}). Equivalently, by Suslin's theorem, a set
that is both analytic and co-analytic in a standard Borel space must
be Borel, so measurability behaves well under complementation. Also,
countable products or coproducts of standard Borel spaces remain
standard Borel
(\href{https://en.wikipedia.org/wiki/Standard\_Borel\_space\#:\~:text=,it\%20into\%20a\%20\%2059}{Standard
Borel space - Wikipedia}).

\item \textbf{Theorems (Kechris 12.A--12.C):} Kechris proves (Theorem 12.13) that
any Borel set in a Polish space \(X\) is the preimage of a Borel set
in \(\mathbb{R}\) under some Borel bijection from \(X\) onto
\(\mathbb{R}\) (when \(X\) is uncountable). Equivalently, one can show
any standard Borel space of size continuum is Borel-isomorphic to
\((\mathbb{R},\mathcal{B})\). He also establishes the “transfer”
theorem: most nice properties (analyticity, etc.) are preserved when
mapping one standard Borel onto another.

\item \textbf{Relevance:} Standard Borel spaces are the natural universe for
measurable dynamics, ergodic theory, and any area using Borel sets
abstractly. In DST, one often reduces problems to standard Borel
spaces since then one can choose convenient Polish topologies. They
justify statements like “we may assume \(X\) is a complete separable
metric space” when dealing with Borel sets abstractly. This notion
underpins uniformization theorems and classification of Borel
equivalence relations, among other results.
\end{itemize}
\section{(Strong) Choquet games and spaces}
\label{strong-choquet-games-and-spaces}
\subsection{Overview}
\label{overview-2}
\textbf{Choquet games} are two-player topological games that characterize
Baire-category properties and completeness of metrics. In the \emph{Choquet
game} \(G(X)\) on a nonempty space \(X\), players I and II alternate
picking nonempty open sets
\(U_0\supseteq V_0\supseteq U_1\supseteq V_1\supseteq\cdots\), and
Player II wins if the intersection \(\bigcap_n U_n\) is nonempty
(\href{https://en.wikipedia.org/wiki/Choquet\_game\#:\~:text=,wins\%2C\%20otherwise\%20Player\%20II\%20wins}{Choquet
game - Wikipedia}). A space is \textbf{Choquet} if Player II has a winning
strategy in \(G(X)\) (equivalently, Player I has \emph{no} winning strategy)
(\href{https://en.wikipedia.org/wiki/Choquet\_game\#:\~:text=if\%20Player\%20I\%20has\%20no,that\%20are}{Choquet
game - Wikipedia}). There is a stronger version, the \emph{strong Choquet
game}, where Player I names a point and a neighborhood each move. In
fact, a metrizable space is \textbf{strong Choquet} if and only if it is
completely metrizable (Polish)
(\href{https://en.wikipedia.org/wiki/Choquet\_game\#:\~:text=All\%20nonempty\%20complete\%20metric\%20spaces,displaystyle}{Choquet
game - Wikipedia}). Choquet games thus provide a game-theoretic
characterization of key topological features: Choquet spaces are Baire
spaces, and \textbf{Polish spaces are exactly the strong Choquet spaces}
(\href{https://en.wikipedia.org/wiki/Choquet\_game\#:\~:text=All\%20nonempty\%20complete\%20metric\%20spaces,displaystyle}{Choquet
game - Wikipedia}).
\subsection{Detailed Summary}
\label{detailed-summary-2}
\begin{itemize}
\item \textbf{Definitions:} In the \emph{Choquet game} on \(X\) (written \(G(X)\)),
Player I first chooses any nonempty open set \(U_0\subseteq X\), then
Player II chooses a nonempty open \(V_0\subseteq U_0\), then I chooses
\(U_1\subseteq V_0\), and so on, always shrinking (possibly with
\(U_n\supseteq V_n\supseteq U_{n+1}\)). If the intersection
\(\bigcap_{n}U_n\) is empty, Player I wins; otherwise (the
intersection is nonempty) Player II wins
(\href{https://en.wikipedia.org/wiki/Choquet\_game\#:\~:text=,wins\%2C\%20otherwise\%20Player\%20II\%20wins}{Choquet
game - Wikipedia}). A space \(X\) is called a \textbf{Choquet space} if
Player II has a (winning) strategy ensuring nonemptiness of the
intersection. Equivalently, by Oxtoby's theorem, \(X\) is Choquet iff
Player I has no winning strategy, which is in turn equivalent to \(X\)
being a Baire space (every countable union of nowhere dense sets has
empty interior)
(\href{https://en.wikipedia.org/wiki/Choquet\_game\#:\~:text=if\%20Player\%20I\%20has\%20no,that\%20are}{Choquet
game - Wikipedia}).

\item \textbf{Strong Choquet game:} The strong version \(G^s(X)\) modifies the play
so that at stage \(n\) Player I first chooses a point \(x_n\in X\) and
an open neighborhood \(U_n\) of \(x_n\), then Player II chooses a
nonempty open \(V_n\subseteq U_n\) containing \(x_n\), with
\(V_n\subseteq U_n\). Player II wins if \(\{x_n\}\) converges to some
point (equivalently, the neighborhoods shrink to a point). One shows
\textbf{every nonempty complete metric space (and every compact Hausdorff
\(T_2\) space)} is strong Choquet
(\href{https://en.wikipedia.org/wiki/Choquet\_game\#:\~:text=All\%20nonempty\%20complete\%20metric\%20spaces,displaystyle}{Choquet
game - Wikipedia}). Conversely, a \emph{separable metrizable}
(i.e. second-countable) space is Polish (complete metric) if and only
if it is strong Choquet
(\href{https://en.wikipedia.org/wiki/Choquet\_game\#:\~:text=All\%20nonempty\%20complete\%20metric\%20spaces,displaystyle}{Choquet
game - Wikipedia}). Thus strong Choquet is exactly the game-theoretic
analog of complete metrizability.

\item \textbf{Properties:} Every strong Choquet space is Choquet; but not every
Choquet space is strong Choquet. Choquet spaces are always Baire: in
fact \(X\) is Baire iff Player I has no winning strategy in \(G(X)\)
(\href{https://en.wikipedia.org/wiki/Choquet\_game\#:\~:text=if\%20Player\%20I\%20has\%20no,that\%20are}{Choquet
game - Wikipedia}). Many classical spaces are Choquet: e.g. any
complete metric or compact metric space (strong Choquet), and any
\(G_\delta\) subspace of a complete metric space. The Choquet property
is hereditary for \(G_\delta\) subsets. In non-metrizable settings,
Choquet conditions are more subtle (studied by Choquet himself and
later Becker--Kechris), but in DST one mostly focuses on metrizable
cases.

\item \textbf{Relevance:} Choquet games connect descriptive set theory to topology:
many DST arguments use these games to establish the Baire property or
perfect set property for definable sets. For example, one can prove a
set is comeager by showing Player I has a winning strategy in a
Banach--Mazur game (a variant of Choquet game). In classification,
Choquet games characterize when a definable set is “large” in the
sense of category. In functional analysis, the Choquet game relates to
the existence of generic points in Banach spaces. Kechris (Chapter
8.C--E) uses these games to prove equivalences like: \emph{“A separable
metrizable space is Polish if and only if it is strong Choquet”}
(\href{https://en.wikipedia.org/wiki/Choquet\_game\#:\~:text=All\%20nonempty\%20complete\%20metric\%20spaces,displaystyle}{Choquet
game - Wikipedia}), and \emph{“every Choquet space is Baire”}
(\href{https://en.wikipedia.org/wiki/Choquet\_game\#:\~:text=if\%20Player\%20I\%20has\%20no,that\%20are}{Choquet
game - Wikipedia}).
\end{itemize}
\section{The Banach--Mazur game}
\label{the-banachmazur-game}
\subsection{Overview}
\label{overview-3}
The \textbf{Banach--Mazur game} is another two-player game on a topological
space \(X\), closely related to Choquet games, used to characterize the
Baire property. In one common version, players alternate choosing nested
nonempty open sets
\(U_0\supseteq V_0\supseteq U_1\supseteq V_1\supseteq\cdots\) (just like
the strong Choquet game, but here Player I wins if the intersection
contains a point of a predetermined target set \(A\subseteq X\), and
Player II wins otherwise). A fundamental theorem states that Player I
has a winning strategy in the Banach--Mazur game with target \(A\) if
and only if \(A\) is comeager (dense \(G_\delta\)) in some nonempty open
set, while Player II has a winning strategy if and only if \(A\) is
meager. Thus determinacy of this game is equivalent to the Baire
property: in ZF+AD one deduces all sets have the Baire property, and in
ZFC one uses it to show all \emph{Borel} sets do.
\subsection{Detailed Summary}
\label{detailed-summary-3}
\begin{itemize}
\item \textbf{Definition:} The classical \emph{Banach--Mazur (B--M) game} \(G^{**}(A)\)
on a Polish (or any topological) space \(X\) with target
\(A\subseteq X\) proceeds as follows: Player I chooses any nonempty
open set \(U_0\), Player II chooses a nonempty open
\(V_0\subseteq U_0\), then I picks \(U_1\subseteq V_0\), II picks
\(V_1\subseteq U_1\), etc., continuing indefinitely. After play
\((U_0,V_0,U_1,V_1,\dots)\), the players look at \(\bigcap_n U_n\). If
this intersection meets \(A\), then Player I wins; otherwise Player II
wins. (Equivalently, II wins if
\(\bigcap_n U_n\subseteq X\setminus A\).)

\item \textbf{Main result:} It is well-known (and used to prove the Baire Category
Theorem) that
\[\text{I has a winning strategy in }G^{**}(A)\iff A\text{ is comeager in some nonempty open set,}\]
\[\text{II has a winning strategy in }G^{**}(A)\iff A\text{ is meager in }X.\]
In particular, Player I can force landing in \(A\) exactly when \(A\)
contains a dense \(G_\delta\) in some open set; otherwise \(A\) is
“small” and Player II can avoid it. The cited result shows the strong
link between this game and the notion of meager/comeager. In effect,
\textbf{the Banach--Mazur game determines Baire-category} of sets.

\item \textbf{Consequences:} From this it follows that \textbf{all Borel sets have the
Baire property}: since Borel determinacy holds for this game in ZF,
every Borel \(A\) yields one of the two cases, meaning \(A\) differs
from a \(G_\delta\) set by a meager set. More generally, one derives
that analytic sets have the Baire property under AD. The game also
illustrates the “topological determinacy” phenomenon: for many
definable classes (Borel, analytic), these games are determined,
linking set-theoretic axioms with regularity properties.

\item \textbf{Relation to Choquet:} The Banach--Mazur game can be seen as a variant
of the Choquet game where Player I tries to steer the play into a
given set \(A\). Actually, if \(A=X\) this game is essentially the
strong Choquet game. Kechris (Chapter 8.H and 21.C) uses it to prove
that any Borel (indeed any analytic) set has the perfect set or Baire
properties, and that determinacy of these games for \emph{arbitrary} sets
would imply strong axioms like AD. The core idea in proofs is to
construct winning strategies by carefully choosing shrinking open
sets; one often sketches it as in: e.g. to show I wins when \(A\) is
comeager, I always plays a basic open in a fixed dense
\(G_\delta\subseteq A\).
\end{itemize}
\section{Games for the Perfect Set Property (PSP)}
\label{games-for-the-perfect-set-property-psp}
\subsection{Overview}
\label{overview-4}
A set \(A\subseteq X\) in a Polish space is said to have the \textbf{Perfect
Set Property (PSP)} if either \(A\) is countable or else \(A\) contains
a nonempty perfect subset (hence of cardinality continuum). The PSP can
be characterized by a two-player game, often called the “\(\ast\)-game”.
In this game \(G^*(A)\) (played on a perfect Polish space \(X\)), the
players alternately choose two disjoint open sets, and Player II selects
one of them; intuitively II is trying to stay out of \(A\) while I tries
to force the play into \(A\). The outcome of a play is a single point
\(x\in X\), and Player I wins if \(x\in A\). A central theorem is:
\textbf{Player I has a winning strategy in \(G^*(A)\) if and only if \(A\)
contains a perfect (Cantor) set, and Player II has a winning strategy if
and only if \(A\) is countable}. Thus determinacy of this game exactly
captures the PSP: one of the players wins, meaning any set is either
countable or has a perfect subset.
\subsection{Detailed Summary}
\label{detailed-summary-4}
\begin{itemize}
\item \textbf{Definition of \(G^*(A)\):} Fix a nonempty perfect Polish space \(X\)
with a compatible complete metric and a basis \(\{V_n\}\) of nonempty
open sets. The \emph{\(\ast\)-game} \(G^*(A)\) for \(A\subseteq X\) is
defined by transfinite “cut-and-choose” moves: First I plays two
disjoint nonempty basic open sets \(U_0^0,U_1^0\) (of diameter
\(<1\)). Then II picks one of them (say \(U^1_0\)) and I responds with
two disjoint open sets \(U_0^1,U_1^1\) of diameter \(<1/2\) contained
in the chosen set. Then II picks one of \(U_0^1,U_1^1\), and I plays
two smaller opens of diameter \(<1/4\) inside that, etc. Because \(X\)
is perfect, this process can continue indefinitely. At the end there
is a unique point \(x\in X\) in the nested intersection. Player \textbf{I}
wins if \(x\in A\), and Player \textbf{II} wins otherwise.

\item \textbf{Main theorem:} It can be shown (cf. Kechris 21.A--21.B) that

\begin{itemize}
\item \emph{I has a winning strategy in \(G^*(A)\)} if and only if \emph{\(A\)
contains a perfect Cantor-like subset} (hence is uncountable with a
perfect part).
\item \emph{II has a winning strategy} if and only if \emph{\(A\) is countable}. In
other words, \(A\) has PSP (uncountable \(\implies\) contains
perfect) precisely when \(G^*(A)\) is determined and one of these
conditions holds. This result is Theorem 8.2 in the lecture notes.
The proof uses classic strategies: if \(A\) has a perfect subset, I
can “force” the play to land inside that Cantor set; conversely if
\(A\) is countable, II can successively avoid enumerated points of
\(A\).
\end{itemize}

\item \textbf{Relevance:} This game-theoretic characterization implies immediately
that \emph{every Borel set has the PSP}: by Martin's theorem all Borel
games are determined, so for any Borel \(A\) exactly one of I or II
wins \(G^*(A)\), yielding one of the two outcomes. More conceptually,
it shows PSP is a “second-order Borel property” and can be derived
from determinacy. In DST, the PSP and related games connect to large
cardinals (AD implies all sets of reals have PSP) and to classical
results like the perfect set theorem for analytic sets. In practice,
one often uses this game as a tool to prove specific sets are
uncountable by describing a winning strategy.
\end{itemize}
\section{Structural properties of the Borel hierarchy}
\label{structural-properties-of-the-borel-hierarchy}
\subsection{Overview}
\label{overview-5}
The \textbf{Borel hierarchy} on a Polish space \(X\) consists of the
pointclasses \(\Sigma^0_\alpha\), \(\Pi^0_\alpha\), and
\(\Delta^0_\alpha\) indexed by countable ordinals \(\alpha\). Two key
structural features are: (1) \emph{Strictness}: for any nonempty Polish \(X\)
and any countable \(\alpha\), the inclusions
\[\Delta^0_\alpha\;\subsetneq\;\Sigma^0_\alpha\;\subsetneq\;\Delta^0_{\alpha+1}\]
are all proper
(\href{https://math.stackexchange.com/questions/509326/borel-hierarchy-doesnt-collapse-before-omega-1\#:\~:text=,0\_\%7B\%5Cxi\%2B1}{descriptive
set theory - Borel hierarchy doesn't ``collapse'' before \(\omega_1\) -
Mathematics Stack Exchange}). In particular, no new Borel sets appear
before reaching height \(\omega_1\) -- the Borel hierarchy runs through
all countable ordinals. (2) \emph{Cardinality}: each nontrivial class has
continuum many sets. In fact, every uncountable standard Borel space has
cardinality \(2^{\aleph_0}\)
(\href{https://en.wikipedia.org/wiki/Standard\_Borel\_space\#:\~:text=one\%20of\%20\%281\%29\%20Image\%3A\%20,is\%20reminiscent\%20of\%20Maharam\%27s\%20theorem}{Standard
Borel space - Wikipedia}), so each level of the Borel hierarchy also
has size continuum in that case.
\subsection{Detailed Summary}
\label{detailed-summary-5}
\begin{itemize}
\item \textbf{Strictness of levels:} Kechris proves (Theorem 22.4) that for any
uncountable Polish space \(X\) and any countable ordinal \(\xi\), the
hierarchy does not collapse:
\[\Delta^0_\xi(X)\subsetneq\Sigma^0_\xi(X)\subsetneq\Delta^0_{\xi+1}(X)\,. \]
Equivalently, there are sets in \(\Sigma^0_\xi\setminus\Pi^0_\xi\) and
in \(\Pi^0_\xi\setminus\Sigma^0_\xi\) at every level
(\href{https://math.stackexchange.com/questions/509326/borel-hierarchy-doesnt-collapse-before-omega-1\#:\~:text=,0\_\%7B\%5Cxi\%2B1}{descriptive
set theory - Borel hierarchy doesn't ``collapse'' before \(\omega_1\) -
Mathematics Stack Exchange}). This means one cannot generate the full
Borel \(\sigma\)-algebra by fewer than \(\omega_1\) steps of alternate
countable unions and intersections of open sets. The cited
MathOverflow answer summarizes: \emph{“for any uncountable Polish space,
the Borel hierarchy is strict”}
(\href{https://math.stackexchange.com/questions/509326/borel-hierarchy-doesnt-collapse-before-omega-1\#:\~:text=,0\_\%7B\%5Cxi\%2B1}{descriptive
set theory - Borel hierarchy doesn't ``collapse'' before \(\omega_1\) -
Mathematics Stack Exchange}).

\item \textbf{Density of classes:} Another structural fact is that the Borel
\(\sigma\)-algebra is exhausted only at level \(\omega_1\):
\(\bigcup_{\alpha<\xi}\Sigma^0_\alpha\neq \mathcal{B}(X)\) for every
countable \(\xi\)
(\href{https://math.stackexchange.com/questions/509326/borel-hierarchy-doesnt-collapse-before-omega-1\#:\~:text=,0\_\%7B\%5Cxi\%2B1}{descriptive
set theory - Borel hierarchy doesn't ``collapse'' before \(\omega_1\) -
Mathematics Stack Exchange}). This implies e.g. that there are
arbitrarily high (countable) Borel ranks: for every countable
\(\alpha\) there exists a Borel set of exact Borel rank \(\alpha\).

\item \textbf{Cardinality:} Since the Borel \(\sigma\)-algebra is generated by a countable
basis, it has cardinality at most continuum. Conversely, for
uncountable \(X\) there are continuum many basic opens, so each
nontrivial class \(\Sigma^0_\alpha(X)\) has cardinality continuum. In
fact, by Kuratowski's theorem all uncountable Borel sets have
cardinality \(2^{\aleph_0}\)
(\href{https://en.wikipedia.org/wiki/Standard\_Borel\_space\#:\~:text=one\%20of\%20\%281\%29\%20Image\%3A\%20,is\%20reminiscent\%20of\%20Maharam\%27s\%20theorem}{Standard
Borel space - Wikipedia}).

\item \textbf{Separation and reduction:} An important property is that disjoint
Borel sets in \(\Sigma^0_\xi\) can often be separated by a
\(\Delta^0_\xi\) set (the Separation Theorem) and one can reduce
questions about a given Borel set to canonical examples (universal
\(\Sigma^0_\xi\) sets). While Kechris's section 22.C focuses on these
structural results (and items like closure under continuous images),
the key takeaway is that the Borel pointclasses are “as complicated as
possible” at each level: none of them coincides with another, and each
level is closed under the natural operations (countable unions for
\(\Sigma^0\), intersections for \(\Pi^0\)) but otherwise
distinguished.
\end{itemize}
\section{The difference hierarchy}
\label{the-difference-hierarchy}
\subsection{Overview}
\label{overview-6}
The \textbf{difference hierarchy} is a refinement of the Borel hierarchy that
decomposes \(\Delta^0_{\alpha+1}\) sets into iterated differences of
simpler sets. The classic Hausdorff--Kuratowski theorem says that any
\(\Delta^0_{\alpha+1}\) set can be written as a union of differences of
a decreasing sequence of \(\Pi^0_\alpha\) sets. In effect, one measures
the “complexity” of a \(\Delta^0_{\alpha+1}\) set by how many times one
needs to alternate set-differences at the \(\Pi^0_\alpha\) level. For
example, every \(F_\sigma\) (\(\Sigma^0_2\)) set is a difference of two
closed (\(\Pi^0_1\)) sets, etc.
\subsection{Detailed Summary}
\label{detailed-summary-6}
\begin{itemize}
\item \textbf{Definition:} Formally, for each countable ordinal \(\alpha\) one
defines the \emph{\(\alpha\)-th difference hierarchy} \(D(\Pi^0_\alpha)\)
consisting of sets that can be expressed as alternating differences of
\(\Pi^0_\alpha\) sets. For instance, \(D_2(\Pi^0_\alpha)\) are sets of
the form \(C_0\setminus C_1\) with \(C_i\in\Pi^0_\alpha\),
\(D_3(\Pi^0_\alpha)\) are finite unions of two differences of
\(\Pi^0_\alpha\) sets, etc., extending transfinitely.

\item \textbf{Hausdorff theorem:} The key result (Hausdorff) is that \emph{every}
\(\Delta^0_{\alpha+1}\) set arises in this way. Precisely:

\begin{quote}
\textbf{Theorem:} \(B\subseteq X\) is \(\Delta^0_{\alpha+1}\) if and only if
there is a countable decreasing sequence of \(\Pi^0_\alpha\) sets
\((C_\beta)_{\beta<\omega_1}\) such that
\[B \;=\; \bigcup_{\beta\text{ even}} (C_\beta \setminus C_{\beta+1}).\]
\end{quote}

In other words, \(B\) is a countable union of disjoint “blocks” where
we alternately subtract one \(\Pi^0_\alpha\) set from another. This
characterization shows that the difference hierarchy \emph{exhausts} the
\(\Delta\)-classes.

\item \textbf{Examples:} Concretely, any \(F_\sigma\) set (a \(\Sigma^0_2\) set)
can be written as \(C_0\setminus C_1\) for closed sets
\(C_1\subseteq C_0\). Similarly, any Boolean combination of
\(G_\delta\) sets (a \(\Delta^0_3\) set) is a finite union of
differences of two \(G_\delta\)'s, etc. The construction in Kechris
and related notes uses transfinite recursion and the completeness of
the metric to peel off “layers” of a Borel set.

\item \textbf{Relevance:} The difference hierarchy gives a finer measure of Borel
complexity than mere class rank. It is fundamental in proofs (via
transfinite induction) that analyze Borel sets: for example, in Wadge
theory one often needs to know how to decompose sets. In descriptive
set theory, it also appears in connection with hierarchies of
equivalence relations (like the \(\Delta\)-hierarchy of equivalence
relations) and in determinacy: each level of the difference hierarchy
corresponds to determinacy of a certain type of game. In summary, the
difference hierarchy theorem (Hausdorff's theorem) shows exactly how
\(\Delta^0_{\alpha+1}\) sets are built from \(\Pi^0_\alpha\) sets by
countably many differences.
\end{itemize}
\section{The Baire hierarchy}
\label{the-baire-hierarchy}
\subsection{Overview}
\label{overview-7}
The \textbf{Baire hierarchy} classifies real-valued functions on a Polish space
by successive pointwise limits of simpler functions. The \emph{Baire class 0}
functions are the continuous ones, and in general a function is of
\emph{Baire class \(\alpha\)} if it can be obtained as a pointwise limit of a
sequence of functions from lower classes
(\href{https://en.wikipedia.org/wiki/Baire\_function\#:\~:text=,Baire\%20class\%20less\%20than\%20\%CE\%B1}{Baire
function - Wikipedia}). Thus Baire class 1 consists of all pointwise
limits of continuous functions. Classical results (going back to
Lebesgue) show that this hierarchy is strict: for each countable
\(\alpha\) there are functions in class \(\alpha+1\) not in any lower
class, and moreover there exist functions (without AC) not in any Baire
class
(\href{https://en.wikipedia.org/wiki/Baire\_function\#:\~:text=Baire\%20class\%20of\%20a\%20countable,not\%20in\%20any\%20Baire\%20class}{Baire
function - Wikipedia}). In fact, Baire-measurable functions correspond
exactly to pointwise limits of continuous functions, and this hierarchy
parallels the Borel hierarchy of level sets.
\subsection{Detailed Summary}
\label{detailed-summary-7}
\begin{itemize}
\item \textbf{Definition:} A real-valued function \(f:X\to\mathbb{R}\) on a
topological space \(X\) is said to be \emph{Baire class 0} if it is
continuous. For a countable ordinal \(\alpha>0\), \(f\) is \emph{Baire
class \(\alpha\)} if there is a sequence \((f_n)\) of functions of
class \(<\alpha\) that converge pointwise to \(f\)
(\href{https://en.wikipedia.org/wiki/Baire\_function\#:\~:text=,Baire\%20class\%20less\%20than\%20\%CE\%B1}{Baire
function - Wikipedia}). Equivalently, \(f\) is Baire class \(\alpha\)
if \(f\) can be obtained by \(\alpha\) many iterated pointwise limits
starting from continuous functions.

\item \textbf{Characterizations:} In metric spaces one has classical
characterizations: for example \(f\) is Baire class 1 iff for every
open set \(U\subseteq\mathbb{R}\) the preimage \(f^{-1}(U)\) is an
\(F_\sigma\) set (a countable union of closed sets). Higher classes
correspond to more complicated preimages (e.g. \(f\) is Baire 2 if
\(f^{-1}(U)\) is a countable union of \(G_\delta\)'s, etc.). These
facts are discussed in §24.A--24.B of Kechris (with proofs).

\item \textbf{Strictness and non-Baire sets:} A theorem of Lebesgue (cited in
Kechris, or see Wikipedia) says that for each countable \(\alpha\),
there are functions of Baire class \(\alpha\) not in any lower class.
Moreover, there exist functions (on \([0,1]\) for example) that are
not of Baire class \(\alpha\) for any \(\alpha<\omega_1\) (so some
Borel functions are not Baire-measurable in the pointwise limit sense)
(\href{https://en.wikipedia.org/wiki/Baire\_function\#:\~:text=Baire\%20class\%20of\%20a\%20countable,not\%20in\%20any\%20Baire\%20class}{Baire
function - Wikipedia}). In fact, the Baire hierarchy \emph{does not
exhaust} all Borel functions: it was shown that under AD every set of
reals is Baire-measurable, but under ZFC there are pathological Borel
functions with no countable pointwise approximations.

\item \textbf{Relation to Borel:} Every Baire class function is Borel-measurable
(since continuous functions are Borel and pointwise limits of Borel
functions remain Borel). The converse fails in general, but one has
the \textbf{Lusin theorem}: every Borel function from \(\mathbb{R}\) to
\(\mathbb{R}\) can be made continuous on a large set, showing it is
\emph{almost} Baire of small class. Kechris's exposition (24.A--24.B)
includes proofs that the Baire classes are closed under natural
operations (sums, products, etc.) and that if \(f\) is Borel then
\(f\) belongs to some countable Baire class (though this requires
additional set theory).

\item \textbf{Summary:} The Baire hierarchy provides a fine gradation of
measurability for functions:

\begin{itemize}
\item Class 0 = continuous.
\item Class 1 = pointwise limits of continuous (characterized by
\(F_\sigma\) preimages).
\item In general, class \(\alpha\) = limits of lower classes
(\href{https://en.wikipedia.org/wiki/Baire\_function\#:\~:text=,Baire\%20class\%20less\%20than\%20\%CE\%B1}{Baire
function - Wikipedia}).
\item Each class properly extends the previous
(\href{https://en.wikipedia.org/wiki/Baire\_function\#:\~:text=Baire\%20class\%20of\%20a\%20countable,not\%20in\%20any\%20Baire\%20class}{Baire
function - Wikipedia}), and many classical pathological functions
live at high levels. This hierarchy is fundamental in real analysis
and DST, especially in effective descriptive set theory and the
study of Polish group representations.
\end{itemize}
\end{itemize}
\section{Uniformization theorems}
\label{uniformization-theorems}
\subsection{Overview}
\label{overview-8}
A \textbf{uniformization} of a relation \(P\subseteq X\times Y\) is a subset
\(P^*\subseteq P\) that is the graph of a (partial) function whose
domain is \(\mathrm{proj}_X(P)\), picking exactly one \(y\)-value for
each \(x\) in the projection
(\href{https://link.springer.com/content/pdf/10.1007/978-1-4612-4190-4\_18\#:\~:text=Given\%20two\%20sets\%20X\%2C\%20Y,a\%20uniformizing\%20function\%20for\%20P}{Uniformization
Theorems | SpringerLink}). Uniformization theorems give conditions
under which one can choose such definable selections. Kechris's Chapter
18 presents classic results: e.g. \textbf{Lusin's and Novikov's theorems} for
analytic and Borel relations, and the \textbf{Kondo--Novikov--Addison theorem}
in the projective hierarchy. Roughly speaking, if \(P\) is Borel (or
analytic) and its vertical sections \(P_x\) are “nice” (for instance,
all countable or all \(\sigma\)-compact), then there exists a \emph{Borel}
function \(f: \mathrm{proj}_X(P)\to Y\) with graph inside \(P\). For
example, \textbf{Novikov's theorem} states: if \(P\subseteq X\times Y\) is
Borel and each section \(P_x\) is at most countable, then \(P\) admits a
Borel uniformization. \textbf{Arsenin--Kunugui} extended this by allowing
sections that are \(\sigma\)-compact. These theorems are indispensable
in DST for constructing measurable selections and studying equivalence
relations.
\subsection{Detailed Summary}
\label{detailed-summary-8}
\begin{itemize}
\item \textbf{Uniformization (Definition):} For \(P\subseteq X\times Y\), a
\emph{uniformization} is any subset \(P^*\subseteq P\) such that each
\(x\in\mathrm{proj}_X(P)\) appears exactly once; equivalently, \(P^*\)
is the graph of a function \(f\) with
\(\mathrm{dom}(f)=\mathrm{proj}_X(P)\) and \((x,f(x))\in P\) for all
\(x\)
(\href{https://link.springer.com/content/pdf/10.1007/978-1-4612-4190-4\_18\#:\~:text=Given\%20two\%20sets\%20X\%2C\%20Y,a\%20uniformizing\%20function\%20for\%20P}{Uniformization
Theorems | SpringerLink}). In other words, \(f(x)\in P_x\) is a
“choice” of a \(y\)-coordinate for each \(x\). The question is: when
can \(f\) be chosen to be Borel (or analytic, etc.) if \(P\) itself is
Borel (or analytic)?

\item \textbf{Key results:} Kechris's sections 18.A--18.D include the following
prototypical theorems:

\begin{itemize}
\item \textbf{Novikov's Uniformization Theorem:} If \(P\subseteq X\times Y\) is
Borel (with \(X,Y\) Polish) and each section \(P_x\) is countable
(or more generally uniformly countable), then there is a Borel
uniformizing function \(f\) on \(\mathrm{proj}_X(P)\). Thus any
countable-to-one Borel relation can be resolved by a Borel selector.
\item \textbf{Arsenin--Kunugui Theorem:} If \(P\) is Borel and each section
\(P_x\) is \(\sigma\)-compact (in \(Y\)), then there is again a
Borel uniformization. This covers situations where each \(P_x\) is,
say, a countable union of compact sets.
\item \textbf{Lusin's Theorem:} For analytic \(P\) with projections covering a
Polish space, there exists an \emph{analytic} uniformization. In fact,
Lusin showed that any analytic relation can be uniformized by an
analytic function on a co-analytic domain. Novikov's theorem is
often proved first for \(F_\sigma\) relations by transfinite
induction and then extended to analytic.
\item \textbf{Kondo-Novikov-Addison Theorem:} In the projective hierarchy, every
\(\mathbf{\Pi}^1_{2n+1}\) set admits a \(\mathbf{\Sigma}^1_{2n+2}\)
uniformization, and similarly for higher levels (using scales). This
implies, for instance, that any co-analytic set
\(P\subseteq\mathbb{R}^2\) has a \(\Sigma^1_2\) (analytic) selector
on a co-analytic domain.
\end{itemize}

\item \textbf{Techniques and relevance:} The proofs combine topology (like
selection theorems for complete metric spaces) with effective
descriptive set theory (scales, pointclasses) and transfinite
recursion. The upshot is that many selection problems admit solutions
of the same or only slightly higher definability level. Uniformization
theorems are used throughout DST: for example, to reduce
classification problems to single-valued functions, to prove Silver's
dichotomy for equivalence relations, and to analyze the structure of
Borel equivalence classes. Kechris's exposition emphasizes theorems
18.10--18.18 (including those by Arsenin--Kunugui and Novikov) and
points out how they follow from or imply separation results.
Intuitively, they say “if the relation \(P\) is not too wild (e.g. has
small sections), one can choose a measurable section”.
\end{itemize}
\section{Partition theorems}
\label{partition-theorems}
\subsection{Overview}
\label{overview-9}
Partition theorems in descriptive set theory are analogues of classical
Ramsey-theoretic results, asserting that certain “nice” colorings (Borel
or analytic) of infinite structures admit large homogeneous subsets. Two
fundamental examples are \textbf{Silver's theorem (dichotomy)} for equivalence
relations and the \textbf{Galvin--Prikry theorem} for colorings of infinite
subsets of \(\mathbb{N}\). Silver's theorem says that any co-analytic
equivalence relation on a Polish space either has only countably many
classes, or else there is a perfect set of pairwise inequivalent points
(\href{https://en.wikipedia.org/wiki/Silver\%27s\_dichotomy\#:\~:text=A\%20relation\%20is\%20said\%20to,2}{Silver's
dichotomy - Wikipedia}). Galvin--Prikry proved that any Borel coloring
of the space \([\mathbb{N}]^\omega\) (all infinite subsets of
\(\mathbb{N}\)) admits an infinite monochromatic subset; Silver extended
this to analytic colorings
(\href{https://mathoverflow.net/questions/67483/is-there-ramsey-theorem-for-infinitary-tuples\#:\~:text=In\%20contrast\%2C\%20Galvin\%20and\%20Prikry,Bbb\%7BN}{co.combinatorics -
Is there Ramsey Theorem for infinitary tuples? - MathOverflow}). These
results (covered in Kechris 19.A--19.E) show that for definable
partitions, one always finds either a “small” homogeneous structure or a
large one of perfect size.
\subsection{Detailed Summary}
\label{detailed-summary-9}
\begin{itemize}
\item \textbf{Galvin--Prikry theorem:} Consider the space \([\mathbb{N}]^\omega\)
of infinite subsets of \(\mathbb{N}\) (with the topology inherited
from Cantor space). If this space is partitioned (colored) into
finitely many Borel pieces, then one of the pieces contains a
homeomorphic copy of \([\mathbb{N}]^\omega\) itself (in particular, it
contains all infinite subsets of some infinite
\(X\subseteq\mathbb{N}\)). Equivalently, any Borel coloring of
\([\mathbb{N}]^\omega\) has an infinite monochromatic set
(\href{https://mathoverflow.net/questions/67483/is-there-ramsey-theorem-for-infinitary-tuples\#:\~:text=In\%20contrast\%2C\%20Galvin\%20and\%20Prikry,Bbb\%7BN}{co.combinatorics -
Is there Ramsey Theorem for infinitary tuples? - MathOverflow}). Ali
Enayat's MathOverflow answer summarizes: \emph{“for Borel colorings of
\([\mathbb{N}]^\omega\), an infinite monochromatic subset always
exists”}
(\href{https://mathoverflow.net/questions/67483/is-there-ramsey-theorem-for-infinitary-tuples\#:\~:text=In\%20contrast\%2C\%20Galvin\%20and\%20Prikry,Bbb\%7BN}{co.combinatorics -
Is there Ramsey Theorem for infinitary tuples? - MathOverflow}).
Silver (1970) showed the same conclusion holds when the coloring is
merely analytic. This is a descriptive version of Ramsey's theorem for
infinite subsets.

\item \textbf{Silver's theorem (dichotomy):} Let \(E\) be a Borel (even
co-analytic) equivalence relation on a Polish space \(X\). Silver's
dichotomy asserts that either \(E\) has countably many equivalence
classes or there are continuum many. More strongly, \emph{if \(E\) is
co-analytic, then either \(E\) has only countably many classes, or
there is a perfect set of reals which are pairwise inequivalent under
\(E\)}
(\href{https://en.wikipedia.org/wiki/Silver\%27s\_dichotomy\#:\~:text=A\%20relation\%20is\%20said\%20to,2}{Silver's
dichotomy - Wikipedia}). In the latter case \(E\) has continuum-many
classes. Thus no intermediate cardinalities occur for definable
equivalence relations. In effect, one gets a perfect homogeneous set
for the “not \(E\)” relation, analogous to Galvin--Prikry.

\item \textbf{Other partition results:} Kechris's chapter also discusses related
theorems, such as Mycielski's theorem, which guarantees a perfect
independent set in certain situations (for example, if \(E\) is an
equivalence relation all of whose classes are meager, one can find a
perfect set of mutually inequivalent points). There are also
Ramsey-type results for Borel graphs and trees. All these theorems
typically use the perfect set property and determinacy of appropriate
games to build perfect homogeneous sets.

\item \textbf{Relevance:} Partition theorems like Silver's and Galvin--Prikry's are
cornerstones of modern DST and invariant descriptive set theory. They
imply that any definable attempt to “color” or classify a perfect
Polish space must either fail to distinguish continuum many points
(producing a perfect homogeneous set) or be essentially countable.
This dichotomy underpins many classification results: for instance,
\textbf{Silver's dichotomy} implies that any Borel or analytic equivalence
relation \(E\) either has only countably many or continuum-many
classes, ruling out a medium-size classification. These theorems also
frequently combine with other principles (like AD or large cardinals)
to yield structural insights about higher-level sets. In summary,
partition theorems ensure that definable partitions on Polish spaces
either admit a perfect homogeneous substructure or collapse to a small
case, greatly constraining the possible complexity of Borel and
analytic relations
(\href{https://en.wikipedia.org/wiki/Silver\%27s\_dichotomy\#:\~:text=A\%20relation\%20is\%20said\%20to,2}{Silver's dichotomy - Wikipedia})
(\href{https://mathoverflow.net/questions/67483/is-there-ramsey-theorem-for-infinitary-tuples\#:\~:text=In\%20contrast\%2C\%20Galvin\%20and\%20Prikry,Bbb\%7BN}{co.combinatorics - Is there Ramsey Theorem for infinitary tuples? - MathOverflow}).
\end{itemize}
\section{Sources}
\label{sec:org6eabc18}
\begin{itemize}
\item Kechris, \emph{Classical Descriptive Set Theory}, Chapters 8, 12,
\end{itemize}
18--19; lecture notes from Kechris's class (where available);
plus references such as
\begin{itemize}
\item (\href{https://en.wikipedia.org/wiki/Standard\_Borel\_space\#:\~:text=A\%20measurable\%20space\%20Image\%3A\%20,algebra.\%5B\%201}{Standard Borel space - Wikipedia})
\item (\href{https://en.wikipedia.org/wiki/Choquet\_game\#:\~:text=All\%20nonempty\%20complete\%20metric\%20spaces,displaystyle}{Choquet game - Wikipedia})
\item (\href{https://math.stackexchange.com/questions/509326/borel-hierarchy-doesnt-collapse-before-omega-1\#:\~:text=,0\_\%7B\%5Cxi\%2B1}{descriptive set theory - Borel hierarchy doesn't ``collapse'' before \(\omega_1\) - Mathematics Stack Exchange})
\item (\href{https://en.wikipedia.org/wiki/Baire\_function\#:\~:text=,Baire\%20class\%20less\%20than\%20\%CE\%B1}{Baire function - Wikipedia})
\item (\href{https://en.wikipedia.org/wiki/Baire\_function\#:\~:text=Baire\%20class\%20of\%20a\%20countable,not\%20in\%20any\%20Baire\%20class}{Baire function - Wikipedia})
\item (\href{https://link.springer.com/content/pdf/10.1007/978-1-4612-4190-4\_18\#:\~:text=Given\%20two\%20sets\%20X\%2C\%20Y,a\%20uniformizing\%20function\%20for\%20P}{Uniformization Theorems | SpringerLink})
\item (\href{https://en.wikipedia.org/wiki/Silver\%27s\_dichotomy\#:\~:text=A\%20relation\%20is\%20said\%20to,2}{Silver's dichotomy - Wikipedia})
\item (\href{https://mathoverflow.net/questions/67483/is-there-ramsey-theorem-for-infinitary-tuples\#:\~:text=In\%20contrast\%2C\%20Galvin\%20and\%20Prikry,Bbb\%7BN}{co.combinatorics - Is there Ramsey Theorem for infinitary tuples? - MathOverflow})
\end{itemize}
\end{document}
